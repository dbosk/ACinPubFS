\section{\Acl{ANOBE}}\label{ANOBE}

The main idea of \ac{ANOBE} is to distribute a key \(k\) (i.e.\ 
key-encapsulation) to a subset of users, such that they can decrypt the 
broadcast message encrypted with \(k\) but no one else can.
The users who receive \(k\) should not be able to figure out who else received 
\(k\) and who did not.

We will now describe how \ac{ANOBE} works.
An overview of the encryption function is given in \cref{EncANOBE} and an 
overview of the decryption function in \cref{DecANOBE}.

\subsection{Encryption}

We must first generate a signature-verification key-pair \((\SignKey{k}, 
  \VerifKey{k})\), then we choose a random permutation \(\pi\colon S\to S\).
Next we must encrypt the key and the verification key \((k, \VerifKey{k})\) for 
every user \(u_i\in S\) in the recipient set \(S\subseteq U\) under their 
respective public key, \(c_{u_i} = \Enc[\PubKey{u_i}]{k, \VerifKey{k}}\).
We let the \ac{ANOBE} ciphertext be the tuple \((\VerifKey{k}, C, \sigma_k)\), 
where
\(C = ( c_{\pi(u_1)}, \ldots, c_{\pi(u_{|S|})})\) and
\(\sigma_k = \Sign[\SignKey{k}]{ C }\).
Note that the signatures does not authenticate the sender, it is used to verify 
correct or incorrect decryption.
We will get back to the details about this shortly.

\begin{frame}
  \begin{figure}
    \framebox{\begin{minipage}{0.96\textwidth}
    \begin{algorithmic}
      \Function{$\EncOp^{\mathrm{ANOBE}}_S$}{$k$}
      \Comment{%
        Recipient set $S$,
        $k$ to be encrypted.
      }
        \State{%
          $(\SignKey{k}, \VerifKey{k})\gets{\Gen{1^\lambda}}$
        }
        \Comment{%
          Signature-verification key-pair\mode<article>{, security parameter 
            $\lambda$}
        }
        \State{%
          Choose a random permutation \(\pi\colon S\to S\).
        }

        \For{$u_i \in S$}
          \State{%
            $c_{u_i}\gets{ \Enc[\PubKey{u_i}]{k, \VerifKey{k}} }$
          }
        \EndFor{}

        \State{%
          $C\gets{( c_{\pi(u_1)}, \ldots, c_{\pi(u_{|S|})} )}$
        }
        \Comment{%
          $\forall i\in\{ 1, \ldots, |S|\}\colon u_i\in S$
        }
        \State{%
          $\sigma\gets{ \Sign[\SignKey{k}]{C} }$
        }
        \State{%
          \Return{$(\VerifKey{k}, C, \sigma_k)$}
        }
      \EndFunction{}
    \end{algorithmic}
    \end{minipage}}
    \caption{%
      An algorithmic overview of the encryption algorithm in the \ac{ANOBE} 
      scheme.
    }\label{EncANOBE}
  \end{figure}
\end{frame}

For \ac{ANOBE} to be anonymous, we require an encryption scheme which is 
key-private~\cite{KeyPrivacy}.
This property ensures that, given a ciphertext, an eavesdropper cannot tell 
which public key, out of a set of public keys, was used to generate the 
ciphertext.
This also means that a user \(u\in U\) must try to decrypt the ciphertexts to 
decide whether \(u\in S\) or \(u\notin S\).
For this reason we also need the encryption scheme to be 
robust~\cite{RobustEncryption}.
Robust encryption ensures that we cannot create a ciphertext which is valid for 
two different recipients:
if we have a set of possible recipients \(R\), we encrypt a message for one of 
them, \(r\in R\), then all \(r^\prime\neq r\) must decrypt the ciphertext to an 
invalid message under their keys.

\subsection{Decryption}

We now have data which we parse as the tuple \((\VerifKey{k}, C, \sigma_k)\).
If \(\Verify[\VerifKey{k}]{ C, \sigma_k } = 0\), then we return \(\bot\) as the 
verification failed.
For each \(c_i\) (from \(C\)):
Compute \(M = \Dec[\PriKey{u}]{ c_i }\).
If \(M \neq \bot\) and \(M = (k, \VerifKey{k})\), then return \(k\).
Otherwise, try the next \(c_i\).
If there are no more \(c_i\) to try, then return \(\bot\).

\begin{frame}
  \begin{figure}
    \framebox{\begin{minipage}{0.96\textwidth}
    \begin{algorithmic}
      \Function{$\DecOp^{\mathrm{ANOBE}}_{\PriKey{u}}$}{$\VerifKey{k}, C, 
        \sigma_k$}
      \Comment{%
        Private key \(\PriKey{u}\),
        ciphertext $(\VerifKey{k}, C, \sigma_k)$.
      }

        \If{$\Verify[\VerifKey{k}]{ C, \sigma_k } = 0$}
          \State{%
            \Return{$\bot$}
          }
        \EndIf{}

        \For{$c_i\in C$}
          \Comment{%
            For each sub-ciphertext, or use tag-hints.
          }
          \State{%
            $M\gets{\Dec[\PriKey{u}]{c_i}}$
          }
          \Comment{%
            Try to decrypt
          }
          \If{$M = \bot$}
            \State{\Return{$\bot$}}
          \ElsIf{$M = (k, \VerifKey{k})$}
            \State{\Return{k}}
          \EndIf{}
        \EndFor{}
        \State{\Return{$\bot$}}
      \EndFunction{}
    \end{algorithmic}
    \end{minipage}}
    \caption{%
      An algorithmic overview of the decryption algorithm in the \ac{ANOBE} 
      scheme.
    }\label{DecANOBE}
  \end{figure}
\end{frame}

To decrypt an \ac{ANOBE} ciphertext, we need a trial-and-error decryption 
procedure to decide if the ciphertext was indeed intended for us.
This is costly as it makes the decryption function complexity \(O(|S|)\).
\citet{ANOBE} presented a tag-hint system along with their \ac{ANOBE} scheme.
The tag-hint system reduced the complexity back to \(O(1)\).
Each recipient needs a tag-hint public-private key-pair.
The system works by letting the recipient do a constant-time computation on the 
received message and the private tag-hint key which gives a value.
If this value is not found among the sub-ciphertexts, then the recipient was 
not among the designated recipients.

