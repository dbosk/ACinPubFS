\subsection{\Iacl{ANOBE} Scheme}\label{ANOBE}

We will now describe the \ac{ANOBE} scheme by \citet{ANOBE}.
We will use \(\ANOBE\) to denote this scheme.
The algorithms work as follows.
\(\ANOBE[Setup]\) generates a master public key \(MPK = (\S*, \{\PubKey{i}\}_{i\in
    U})\) and the master secret key \(MSK = \{\PriKey{i}\}_{i\in U}\), where 
\((\PubKey{i}, \PriKey{i})\rgets \E*[Keygen][1^\lambda]\).
\(\ANOBE[Keygen][MPK, MSK, i]\) simply returns \(\PriKey{i}\) from \(MSK\).
An overview of the encryption function is given in \cref{EncANOBE} and an 
overview of the decryption function in \cref{DecANOBE}.

\subsubsection{Encryption}

We must first generate a one-time signature key-pair \((\SignKey{}, 
  \VerifKey{})\), then we choose a random permutation \(\pi\colon R\to R\).
Next we must encrypt the message \(m\) and the verification key \((m, 
  \VerifKey{})\) for every user \(i\in R\) in the recipient set \(R\subseteq 
  U\) under their respective public key, \(c_{i} = \E*[Enc][\PubKey{i}, 
  m\concat \VerifKey{}]\).
We let the \ac{ANOBE} ciphertext be the tuple \((\VerifKey{}, C, \sigma)\), 
where
\(C = ( c_{\pi(1)}, \ldots, c_{\pi({|S|})})\) and
\(\sigma = \S*[Sign][\SignKey{}, C ]\).
Note that the signatures does not authenticate the sender, it is used to verify 
correct or incorrect decryption.
We will get back to the details about this shortly.

\begin{frame}
  \begin{figure}
    \framebox{\begin{minipage}{0.96\textwidth}
    \begin{algorithmic}
      \Function{$\ANOBE[Enc]$}{$MPK, m, R$}
      \Comment{%
        Recipient set $R$,
        $m$ to be encrypted.
      }
        \State{%
          $(\SignKey{}, \VerifKey{})\rgets{\S*[Keygen][1^\lambda]}$
        }
        \Comment{%
          Signature key-pair, security parameter $\lambda$
        }
        \State{%
          Choose a random permutation \(\pi\colon R\to R\).
        }

        \For{$i \in R$}
          \State{%
            $c_{i}\gets{ \E*[Enc][\PubKey{i}, m\concat \VerifKey{}] }$
          }
        \EndFor{}

        \State{%
          $C\gets{( c_{\pi(1)}, \ldots, c_{\pi({|S|})} )}$
        }
        \State{%
          $\sigma\gets{ \S*[Sign][\SignKey{}, C] }$
        }
        \State{%
          \Return{$(\VerifKey{}, C, \sigma)$}
        }
      \EndFunction{}
    \end{algorithmic}
    \end{minipage}}
    \caption{%
      An algorithmic overview of the encryption algorithm in the \ac{ANOBE} 
      scheme.
    }\label{EncANOBE}
  \end{figure}
\end{frame}

\subsubsection{Decryption}

We now have data which we parse as the tuple \((\VerifKey{}, C, \sigma)\).
If \(\S*[Verify][\VerifKey{}, C, \sigma ] = 0\), then we return \(\bot\) as the 
verification failed.
For each \(c\) in \(C\):
Compute \(M = \E*[Dec][\PriKey{}, c ]\).
If \(M \neq \bot\) and \(M = (m, \VerifKey{})\), then return \(m\).
Otherwise, try the next \(c\).
If there are no more \(c\) to try, then return \(\bot\).

\begin{frame}
  \begin{figure}
    \framebox{\begin{minipage}{0.96\textwidth}
    \begin{algorithmic}
      \Function{$\ANOBE[Dec]$}{$MPK, \PriKey{}, C_{\ANOBE}$}
%      \Comment{%
%        Private key \(\PriKey{}\),
%        ciphertext $C_{\ANOBE} = (\VerifKey{}, C, \sigma)$.
%      }

        \If{$\S*[Verify][\VerifKey{}, C, \sigma ] = 0$}
          \State{%
            \Return{$\bot$}
          }
        \EndIf{}

        \For{$c\in C$}
          \State{%
            $M\gets{\E*[Dec][\PriKey{}, c]}$
          }
          \Comment{%
            Try to decrypt
          }
          \If{$M = \bot$}
            \State{\Return{$\bot$}}
          \ElsIf{$M = (m, \VerifKey{})$}
            \State{\Return{m}}
          \EndIf{}
        \EndFor{}
        \State{\Return{$\bot$}}
      \EndFunction{}
    \end{algorithmic}
    \end{minipage}}
    \caption{%
      An algorithmic overview of the decryption algorithm in the \ac{ANOBE} 
      scheme.
    }\label{DecANOBE}
  \end{figure}
\end{frame}

To decrypt an \ac{ANOBE} ciphertext, we need a trial-and-error decryption 
procedure to decide if the ciphertext was indeed intended for us.
This is costly as it makes the decryption function complexity \(O(|S|)\).
\citet{ANOBE} presented a tag-hint system along with their \ac{ANOBE} scheme.
The tag-hint system reduced the complexity back to \(O(1)\).
As this is not relevant for our discussion, we refer the reader 
to~\cite{ANOBE}.

