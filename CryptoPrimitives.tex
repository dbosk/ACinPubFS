\subsection{Cryptographic Primitives}
\label{CryptoPrimitives}

We will use mainly standard public-key cryptography.
We need a public-key encryption scheme and two signature schemes.
Although we can use shared-key mechanisms in some places, we will maintain 
a public-key notation throughout this paper and merely point out in the places 
where a shared-key scheme would be possible.

We will use a public-key encryption scheme \(\E = (\E[Keygen], \E[Enc], 
  \E[Dec])\).
This scheme must be semantically secure under chosen-ciphertext attacks 
(IND-CCA) and key-private (AI-CCA)~\cite{KeyPrivacy}, i.e.\ the ciphertext does 
not leak under which key it was created.
We also need an unforgeable signature scheme \(\S = (\S[Keygen], \S[Sign], 
  \S[Verify])\).

We additionally need \iac{BE} scheme \(\BE = (\BE[Setup],\allowbreak 
  \BE[Keygen],\allowbreak \BE[Enc],\allowbreak \BE[Dec])\) which is anonymous 
and semantically secure under chosen-ciphertext attacks 
(ANO-IND-CCA)~\cite{ANOBE}.
Let \(U = \{1, \ldots, n\}\) be the universe of users.
\(\BE[Setuo][1^\lambda, n]\) generates a master public key \(MPK\) and 
master secret key \(MSK\) for \(n\) users.
\(\BE[Keygen][MPK, MSK, i]\) generates the secret key \(\PriKey{i}\) for 
user \(i\in U\).
\(\BE[Enc][MPK, m, R]\) will encrypt a message \(m\) to a ciphertext \(c\) 
for the set of users \(R\subseteq U\) and \(\BE[Dec][MPK, \PriKey{i}, c]\) 
will return \(m\) if \(i\in R\).
Formally we define ANO-IND-CCA by the following game.

% XXX Rewrite ANO-IND-CCA to original
\begin{definition}[ANO-IND-CCA~\cite{ANOBE}]\label{ANO-IND-CCA}
  Let \(U = \{1, \ldots, n\}\) be the universe of users and \(\BE\) be \iac{BE} 
  scheme.
  \Iac{BE} scheme is anonymous and adaptively IND-CCA (ANO-IND-CCA) if the 
  adversary has negligible advantage in winning the following game.
  \begin{description}
    \item[Setup] The challenger runs \(\BE[Setup][1^\lambda, n]\), where 
      \(\lambda\) is the security parameter.
      This generates a master public key which is given to the adversary 
      \(\A\).

    \item[Phase 1] The adversary \(\A\) may corrupt any \(i\in U\), i.e.\ 
      request its secret key \(k_i = \BE[Keygen][MPK, MSK, i]\) through an 
      oracle.
      Additionally \(\A\) has access to a decryption oracle to decrypt 
      arbitrary ciphertexts for any \(i\in U\), upon request \((c, i)\) the 
      oracle will return \(\BE[Dec][MPK, k_i, c]\).

    \item[Challenge] The adversary chooses messages \(m_0, m_1\) such that 
      \(|m_0| = |m_1|\) and recipient sets \(R_0, R_1\) such that \(|R_0| 
        = |R_1|\).
      For all corrupted \(i\in U\) we have \(i\notin R_0\cup R_1\setminus 
        (R_0\cap R_1)\).
      If there exists an \(i\in R_0\cap R_1\) we require \(m_0 = m_1\).
      Then \(\A\) gives them to the challenger.
      The challenger randomly chooses \(b\in \{0, 1\}\) and runs \(c^*\gets 
        \BE[Enc][MPK, m_b, R_b]\) and gives \(c^*\) to the adversary.

    \item[Phase 2] The adversary may continue to corrupt \(i\notin R_0\cup 
        R_1\setminus (R_0\cap R_1)\).
      \(\A\) may corrupt \(i\in R_0\cap R_1\) only if \(m_0 = m_1\).
      The adversary is not allowed to use the decryption oracle on the 
      challenge ciphertext \(c^*\).

    \item[Guess] \(\A\) outputs a bit \(\hat{b}\) and wins the game if 
      \(\hat{b} = b\).

  \end{description}
  We define the adversary's advantage \(\Adv{\text{ANO-IND-CCA}}{\A, 
      \Cop}[1^\lambda] = |\Pr[ \hat{b} = b ] - \frac{1}{2}|\).
\end{definition}

Throughout \(\BE\) can be any \ac{BE} scheme with the ANO-IND-CCA property.
However, we will use the \ac{ANOBE} scheme by \citet{ANOBE} as an example in 
our discussion.
For our description of this scheme, which follows, we need slight variants of 
the above encryption (\(\E\)) and signature (\(\S\)) schemes.
We need an encryption scheme \(\E* = (\E*[Keygen], \E*[Enc], \E*[Dec])\) which 
is in addition to key-private also robust (ROB-CCA)~\cite{RobustEncryption}.
For the signature scheme \(\S* = (\S*[Keygen], \S*[Sign], \S*[Verify])\), we 
only need it to be a strongly-unforgeable one-time signature scheme.
