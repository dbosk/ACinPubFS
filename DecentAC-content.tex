\title{%
  Privacy-Preserving Access Control
  in Decentralized Storage
  for Online Social Networks
  % XXX Can we generalize Online Social Networks?
}
\author{%
  Daniel Bosk \and
  Sonja Buchegger
}
\institute{%
  School of Computer Science and Communication\\
  KTH Royal Institute of Technology, Stockholm, Sweden\\
  \email{\{dbosk,buc\}@kth.se}%
}
\date{IFIP Summer School\\Edinburgh, 20th August 2015}

\maketitle

\mode* % required for slides to compile without non-frame text

\begin{abstract}
  Online social networks collect and store large amounts of private data.
  Trusting third parties with too much data is a privacy risk.
  For this reason \acp{DOSN} were proposed.
  The current research on access control in \acp{DOSN} has been focused on 
  efficiently achieving confidentiality for data.
  In this paper, we focus on achieving privacy-preserving access control 
  mechanisms for the \ac{DOSN} setting, i.e.~hidden policies, 
  hidden credentials and hidden decisions.
  We design an access-control scheme for two models of communication, the Pull 
  Model and the Push Model.
  These models yield different privacy properties and requirements on the 
  underlying system.
  The complexity of the two approaches are asymptotically the same and differ 
  only by constant factors.

  \keywords{%
    Decentralized Online Social Networks,
    Access Control,
    Decentralized Storage,
    Anonymous Broadcast Encryption,
    Hidden Policies,
    Hidden Credentials,
    Hidden Decisions
  }
\end{abstract}

% Since this a solution template for a generic talk, very little can
% be said about how it should be structured. However, the talk length
% of between 15min and 45min and the theme suggest that you stick to
% the following rules:
% - Exactly two or three sections (other than the summary).
% - At *most* three subsections per section.
% - Talk about 30s to 2min per frame. So there should be between about
%   15 and 30 frames, all told.

% 1) what is the concrete setting that we consider (entities and their props),
% 2) what overall functionality and security properties do we want,
% 3) how are these properties realized (protocols, composition)?
%
% - Good synonyms for "looking at":
%   - investigating, exploring, evaluating, researching, additionally;
%   - if in relation to something else: contrasting, comparing.

\acresetall{}
\section{Introduction}\label{Introduction}
% XXX Rewrite the introduction

Alice and her friends want to communicate asynchronously.
To do this they want to use a publicly available file system and write their 
messages to different files, which the other party later can read.
We are interested in enforcing access-control policies in such a public file 
system which does not have any built-in access control mechanisms.
Our approach is to introduce a layer of encryption as a logical reference 
monitor.
Beyond the expected confidentiality, Alice wants some stronger privacy 
properties as well:
\begin{itemize}
  \item \dots
\end{itemize}

We will assume a simple file system with no built-in access control.
This system is discussed and defined in \cref{FileSystem}.
We will continue to define an ideal communication functionality and two models 
implementing that functionality in \cref{PushAndPull}.
Then we will discuss the high-level properties of the two models in 
\cref{PushAnalysis,PullAnalysis}.
Finally we prove in \cref{SecurityProof} that one of the protocols securely 
realizes the ideal communication model.


% XXX Formalize the setting, the system model:
%  - A file system: read, write, append.
%  - Adversary is storage node, operates file system.
%  - Adversary participates as user.

\section{A General File System}\label{FileSystem}

We will model our system as an abstract file system with the operations create, 
read and append.
The operations are defined as we intuitively expect: we can create an object, 
then read it and also append to it.
The file system itself provides no access control.

\begin{definition}[Public File System]\label{FS}
  A \emph{file} \(f = (i, m)\) consists of the \emph{identifier} \(i\) and the 
  \emph{content} \(m\).
  We define the set of files \(F = \{(i, m_i)\}\) together with the following 
  operations to be the \emph{public file system} \(\FS\):
  \begin{itemize}
    \item \(\fscreate[i][m_i]\) will set \(F\gets F\cup \{(i, m_i)\}\) if 
      \(\{(i^\prime, m^\prime)\in F \mid i^\prime = i\} = \emptyset\).
      Otherwise this operation has no effect.

    \item \(\fsread[i] = m_i\) if \((i, m_i)\in F\), otherwise 
      \(\fsread{i^\prime} = \bot\).

    \item \(\fsappend[i][m^\prime]\) will set \((i, m)\gets (i, m\concat 
        m^\prime)\) for \((i, m)\in F\).
      If \(\{(i^\prime, m^\prime)\in F\mid i^\prime = i\} = \emptyset\) this 
      operation has no effect.
  \end{itemize}
\end{definition}

We can see in the definition that anyone can read any object in the file 
system.
Anyone can also create new files and append to existing files.

% XXX Formalize assumptions:
%  - Storage node cannot tell the origin of two requests apart.
%  - Eve controls the entire storage system?
%    - Here she can do an exhaustive search.
%  - Eve only controls part of the network?
%    - Can Eve do an exhaustive search in the storage?
%
%Assuming that Eve controls the entire storage system should be more 
%interesting as it's a stronger adversary.  There shouldn't be any need 
%for the weaker adversary who controls only a part of the storage system.  
%We'll have to remove some randomness re-use from ANOBE for the 
%implementation, but that should suffice.

We will let the adversary Eve operate the file system \(\FS\) defined in 
\cref{FS}.
By this we mean that Eve has access to the internal state of \(\FS\) directly, 
i.e.\ the set of files \(F\).
She can thus read all files (same as everyone else), but she can also do 
arbitrary searches since she does not have to use \(\fsread{\cdot}\).
She can also modify the files (including deleting them).

To be consistent with \cref{FS}, Eve cannot distinguish between Alice's and 
Bob's requests to the operations of \(\FS\) --- in the definition there is 
nothing to identify who used any of the operations.
But she can record the times at which they occurred with her own clock, since 
she executes them.


% XXX Formalize what we want to do:
%  - Publisher makes a message m asynchronously available to n recipients 
%    while hiding the recipient set:
%    - Pull model: publish m as object o.
%    - Push model: publish m as objects o_1, ..., o_n.
%  - Each recipient reads m:
%    - Pull model: read object o.
%    - Push model: read object o_i.
%  - The publisher should be unidentifiable for a non-recipient.
%
% XXX Top-down approach:
%  - First, analyse high-level properties in pull and push models.
%    - Motivation for hidden policies, credentials, decisions.
%    - Alice removes Bob from recipient set: Bob shouldn't distinguish this or 
%      if Alice stopped posting to all.
%    - Bob shouldn't be able to monitor Alice's activities unrelated to Bob.
%    - Eve knows Alice's and Bob's inboxes, for any message m in Alice's, she 
%      shouldn't be able to tell if it's also in Bob's.
%  - Second, suggest an example of possible implementation using (modified) 
%    ANOBE.
%  - Third, complexity for solution.
%
\section{Two Communication Protocols}\label{PushAndPull}

There are several ways Alice can implement the message passing with her 
friends.
We will start with a general outline of the model of communication Alice will 
use, then we will proceed to the details of two alternative protocols that 
implements the model.

% XXX Shall we have an adversary interface for C?
\begin{definition}[Communication model]\label{CommunicationModel}
  Let \(\Cop\) be a \emph{\((p, S)\)-communication model} with 
  a \emph{publisher} \(p\) and a \emph{set of subscribers} \(S\).
  Then we have the following operations defined on \(\Cop\):
  \begin{itemize}
    \item Before start the publisher runs \(\csetup{p}[1^\lambda]\) and each 
      subscriber \(s\in S\) runs \(\csetup{s}[1^\lambda]\), where \(\lambda\) 
      is the security parameter.
    \item The publisher \(p\) uses \(\cpub{p}[R][m]\) to publish the message 
      \(m\) to the designated \emph{recipient set} \(R\) by making it available 
      to all subscribers \(s\in R\subseteq S\).
    \item Each subscriber \(s\in S\) uses \(\cget{s}[]\) to get the set \(M\) 
      of published messages \(m\in M\) for which \(s\) was in the recipient 
      set.
  \end{itemize}
\end{definition}

In our desired scenario, Alice acts as a publisher and her friends as the 
subscribers.
We note that no subscriber \(s\in S\) can use \(\cpub{s}[R][m]\) to publish 
a message \(m\).
The communication model is strictly unidirectional from publisher to 
subscribers and they cannot access the publisher's interface.
In the same sense, no subscriber \(s^\prime\in S\) can use \(\cget{s}\) for 
\(s^\prime\neq s\).
It follows too that no subscriber \(s_i\) can learn which \(s_1, \ldots, s_n\in 
  R\) beyond that \(i\in \{1, \ldots, n\}\) and \(s_i\in R\) for a given 
message \(m\).
More formally this is captured in the following security game, which is adapted 
from Def.\ 2 of~\cite{ANOBE} (ANO-IND-CCA).

\begin{definition}[ANO-IND-CCA]\label{ANO-IND-CCA}
  We define the ANO-IND-CCA game as follows:
  \begin{description}
    \item[Setup] The challenger runs \(\csetup{s}[1^\lambda]\) for all \(s\in 
        S\) and \(\csetup{p}[1^\lambda]\).
    \item[Phase 1] The adversary \(\A\) may corrupt any \(s\in S\) and take 
      control over their \(\cget{s}\) interface.
    \item[Challenge] The adversary chooses messages \(m_0, m_1\) such that 
      \(|m_0| = |m_1|\) and recipient sets \(R_0, R_1\) such that \(|R_0| 
        = |R_1|\) and for all corrupted \(s\in S\) we have \(s\notin S_0\cup 
        S_1\setminus (S_0\cap S_1)\).
      If \(s\in S_0\cap S_1\) we require \(m_0 = m_1\).
      Then \(\A\) gives them to the challenger.
      The challenger randomly chooses \(b\in \{0, 1\}\) and executes 
      \(\cpub{p}[R_b][m_b]\).
    \item[Phase 2] The adversary may continue to corrupt \(s\notin S_0\cup 
        S_1\setminus (S_0\cap S_1)\).
      \(\A\) may corrupt \(s\in S_0\cap S_1\) only if \(m_0 = m_1\).
    \item[Guess] \(\A\) outputs a bit \(\hat{b}\) and wins the game if 
      \(\hat{b} = b\).
  \end{description}
  We define the adversary's advantage \(\Adv{\text{ANO-IND-CCA}}{\A, 
      \Cop}[1^\lambda] = |\Pr[ \hat{b} = b ] - \frac{1}{2}|\).
\end{definition}

% XXX Formalize other required (more high-level) properties:
%  - Alice removes Bob as a friend: Bob shouldn't distinguish this or if 
%    Alice stopped posting to all.
%  - Bob shouldn't be able to monitor Alice's activities.
%  - Eve knows Alice's and Bob's inboxes, for any message m in Alice's, 
%    she shouldn't be able to tell if it's also in Bob's.

However, there are some desirable privacy properties that are not captured by 
ANO-IND-CCA\@.
Bob might become jealous if he realizes that Alice no longer includes him in 
the recipient set for her messages.
As such Bob should only be able to determine that Alice publishes content by 
being in the recipient set himself.
We capture this in the following game.

\begin{definition}[Jealous Bob]\label{JealousBob}
  \dots
\end{definition}

There are several ways to implement the message-passing protocol for the 
communication model in \cref{CommunicationModel}.
We will focus on two alternative protocols which uses \(\FS\) from \cref{FS} 
above, these are the Push Model and the Pull Model.
The Push Model is analogous to the subscription of magazines:
a subscriber contacts the publisher and signs a subscription, whenever the 
publisher issues a new magazine it sends a copy to the subscribers mailbox.
The Pull Model the converse of the Push Model.
It is analogous to the selling of magazines in kiosks:
the publisher issues magazines and the \enquote{subscribers} come to the kiosk 
and buy them whenever they want.
We note that essentially all \ac{BE} schemes, including \ac{ANOBE}, are 
designed for the Pull Model.
It is therefore natural to start our analysis with the Pull Model.

\subsection{Analysis of the Pull Model}
\label{PullAnalysis}

In this model, each publisher has an index file in the file system.
This is simply a file object with a randomly chosen identifier.
The publisher adds new publications to the index while subscribers pull new 
content from the index.
In our analysis we will use the following definition for the Pull Model.

\begin{definition}[Pull Model]\label{PullModel}
  Let \(\mathcal{E} = (\KeygenOp, \EncOp, \DecOp)\) be an authenticated 
  broadcast-encryption scheme and \(\FS\) be a public file system.
  We denote by \(\PullOp\) the \emph{push model protocol} implementing a \((p, 
    S)\)-communication model through the operations
  \begin{itemize}
    \item \(\psetup{p}[\cdot]\) and \(\psetup{s}[\cdot]\),
    \item \(\ppub{p}[\cdot][\cdot]\), and
    \item \(\pget{s}[\cdot]\)
  \end{itemize}
  as in \cref{PullFunctions}.
\end{definition}

\begin{figure}%[p]
  \framebox{%
  \begin{minipage}[t]{0.48\textwidth}
    \begin{algorithmic}
      \Function{$\psetup{p}$}{$1^\lambda$}
        \State{$(\SignKey{p}, \VerifKey{p})\rgets \Keygen{1^\lambda}$}
        \State{$i_p\rgets \{0, 1\}^{\lambda}$}
        \State{$\forall s\in S$: give $(\VerifKey{p}, i_p)$ to $\Pull{s}$.}
      \EndFunction{}

      \Statex{}

      \Function{$\ppub{p}$}{$R, m$}
        \State{$\sigma \gets \Enc[\{\PubKey{s}\}_{s\in R}, \SignKey{p}]{m}$}
        \State{$\fsappend[i_p][\sigma]$}
      \EndFunction{}
    \end{algorithmic}
  \end{minipage}
  \begin{minipage}[t]{0.48\textwidth}
    \begin{algorithmic}
      \Function{$\psetup{s}$}{$1^\lambda$}
        \State{$(\PubKey{s}, \PriKey{s})\rgets \Keygen{1^\lambda}$}
        \State{Give $\PubKey{s}$ to $\Push{p}$.}
      \EndFunction{}

      \Statex{}

      \Function{$\pget{s}$}{}
        \State{$C = \fsread[i_p]$}
        \State{$M\gets \emptyset$}
        \For{$\sigma\in C$}
          \State{$m_\sigma = \Dec[\PriKey{s}, \VerifKey{p}]{\sigma}$}
          \State{$M\gets M\cup \{m\}$}
        \EndFor{}
        \State{\Return{$M$}}
      \EndFunction{}
    \end{algorithmic}
  \end{minipage}
  }
  \caption{%
    Functions implementing the communication model for the Pull Model protocol.
    The publisher's interface is to the left and the subscribers' to the right.
  }\label{PullFunctions}
\end{figure}

When Alice executes \(\pnsetup{p}\) the \(\KeygenOp\) algorithm generates 
a signature-verification key-pair \((\SignKey{p}, \VerifKey{p})\).\footnote{%
  Since the scheme \(\mathcal{E}\) is an authenticated \ac{BE} scheme, the key 
  generation yields a proper broadcast keys for encryption and decryption in 
  addition to a signature-verification key-pair --- in Alice's case (conversely 
  for her friends) we are only interested in the authentication part, so we 
  simply discard her encryption keys in the definition.
}
She also randomly chooses an identifier.
Then the verification key and identifier are both given to all her friends.
Each of her friends \(s\in S\), when they execute \(\pnsetup{s}\) the 
\(\KeygenOp\) algorithm generates a public-private key-pair.\footnote{%
  Actually this is ju the notation we use.
  It might just as well be Alice's execution that generates the broadcast keys 
  and distributes them to her friends, while her friends' \(\psetup{s}\) does 
  nothing.
}
They give their respective public key to Alice.

When Alice wants to publish a message \(m\) to the recipient set \(R\subseteq 
  S\), she runs \(\ppub{p}[R][m]\).
This operation simply creates an authenticated broadcast ciphertext 
\(\sigma\gets \Enc[\{\PubKey{s}\}_{s\in R}, \SignKey{p}]{m}\) by combining the 
broadcast key \(\{\PubKey{s}\}_{s\in R}\) and the signature key 
\(\SignKey{p}\).
She appends the ciphertext to her index file.
Each friend can then use \(\pnget{s}\) to retrieve the ciphertexts from the 
file system and decrypt them.

As we mentioned earlier (and in \cref{PullModel}), \ac{BE} schemes are designed 
for the Pull Model.
We know from the analysis by \citet{ANOBE} that Eve has negligible advantage in 
ANO-IND-CCA (\cref{ANO-IND-CCA}) when the \ac{ANOBE} scheme is used.
However, \ac{ANOBE} used in the Pull Model cannot yield negligible advantage to
jealous Bob (\cref{JealousBob}).
Bob can read Alice's index, conclude that there are entries which he cannot 
decrypt and thus he can become jealous.
(In \cref{PullFunctions} we see that Bob can count the number of \(\bot\) in 
the output of \(\pget{\cdot}\).)
To prevent this Alice needs a unique index per friend, so Bob has his own 
index.
This actually reduces the problem to the Push Model.

\subsection{Analysis of the Push Model}
\label{PushAnalysis}

The idea of the Push Model is for each subscriber to have an inbox in the file 
system --- as opposed to the Pull Model, where the publisher has an 
\enquote{outbox}, namely the index.
This is simply a file object with a randomly chosen identifier.
The publisher then puts all published material in the inbox of each subscriber.
We base our analysis on the following definition.

\begin{definition}[Push Model]\label{PushModel}
  Let \(\mathcal{E} = (\KeygenOp, \EncOp, \DecOp)\) be an authenticated 
  encryption scheme and \(\FSop\) be a public file system.
  We denote by \(\PushOp\) the \emph{push model protocol} implementing a \((p, 
    S)\)-communication model through the operations
  \begin{itemize}
    \item \(\pnsetup{p}[\cdot]\) and \(\pnsetup{s}[\cdot]\),
    \item \(\pnpub{p}[\cdot][\cdot]\), and
    \item \(\pnget{s}[\cdot]\)
  \end{itemize}
  as in \cref{PushFunctions}.
\end{definition}

\begin{figure}%[p]
  \framebox{%
  \begin{minipage}[t]{0.48\textwidth}
    \begin{algorithmic}
      \Function{$\pnsetup{p}$}{$1^\lambda$}
        \State{$(\SignKey{p}, \VerifKey{p})\rgets \Keygen{1^\lambda}$}
        \State{$\forall s\in S$: give $\VerifKey{p}$ to $\Push{s}$.}
      \EndFunction{}

      \Statex{}

      \Function{$\pnpub{p}$}{$R, m$}
        \For{$s\in R$}
          \State{$\sigma_s\gets \Enc[\PubKey{s}, \SignKey{p}]{m}$}
          \State{$\fsappend[i_s][\sigma_s]$}
        \EndFor{}
      \EndFunction{}
    \end{algorithmic}
  \end{minipage}
  \begin{minipage}[t]{0.48\textwidth}
    \begin{algorithmic}
      \Function{$\pnsetup{s}$}{$1^\lambda$}
        \State{$(\PubKey{s}, \PriKey{s})\rgets \Keygen{1^\lambda}$}
        \State{$i_s\rgets \{0, 1\}^{\lambda}$}
        \State{Give $(\PubKey{s}, i_s)$ to $\Push{p}$.}
      \EndFunction{}

      \Statex{}

      \Function{$\pnget{s}$}{}
        \State{$C = \fsread[i_s]$}
        \State{$M\gets \emptyset$}
        \For{$\sigma\in C$}
          \State{$m_\sigma\gets \Dec[\PriKey{s}, \VerifKey{p}]{\sigma}$}
          \State{$M\gets M\cup \{m\}$}
        \EndFor{}
        \State{\Return{$M$}}
      \EndFunction{}
    \end{algorithmic}
  \end{minipage}
  }
  \caption{%
    Functions implementing the communication model for the Push Model protocol.
    The publisher's interface is to the left and the subscribers' to the right.
  }\label{PushFunctions}
\end{figure}

When Alice executes \(\pnsetup{p}\) the \(\KeygenOp\) algorithm generates 
a signature-verification key-pair \((\SignKey{p}, \VerifKey{p})\) and gives the 
verification key to all her friends.\footnote{%
  Since the scheme \(\mathcal{E}\) is an authenticated encryption scheme, the 
  key generation yields a public-private key-pair and a signature-verification 
  key-pair --- in Alice's case (conversely for her friends) we are only 
  interested in the authentication part, so we simply discard her encryption 
  keys in the definition.
}
Each of her friends \(s\in S\), when they execute \(\pnsetup{s}\) the 
\(\KeygenOp\) algorithm generates a public-private key-pair.
Additionally they randomly choose a string as an identifier.
They give the public key and the identifier to Alice.

When Alice wants to send a message \(m\) to a subset \(R\subseteq S\) of her 
friends, she uses \(\pnpub{p}\) to create the authenticated ciphertexts 
\(\sigma_s\gets \Enc[\PubKey{s}, \SignKey{p}]{m}\) for each friend \(s\in R\).
The encryption scheme simply combines the recipients public key \(\PubKey{s}\) 
and Alice's signature key \(\SignKey{p}\).
Then she uses \(\FS\) to append the ciphertext \(\sigma_s\) to the file with 
identifier \(i_s\).

The operation \(\pget{s}\) works exactly the same as for the Pull Model, it 
just uses a different file.

Note that the authenticated encryption scheme \(\mathcal{E}\) can be 
a symmetric-key scheme although we use the notation of public-key cryptography 
in our abstraction.
We can simply let \(\PubKey{} = \PriKey{}\) and \(\SignKey{} = \VerifKey{}\) to 
achieve this.
We have thus not yet limited ourselves to any properties of \(\mathcal{E}\) 
(beyond authentication and confidentiality).

% XXX Write high-level analysis of push model
Some high-level analysis \dots

Eve can correlate all inboxes with by observing equal ciphertexts.

\section{Security Proof for the Push Model}

\dots


\section{Related Work}

\dots

% XXX Formalize hidden policies, credentials and decisions --- but how?
%  - Hidden policies: policy-hiding ciphertext, how to formalize?
%    - Given two equally-sized ciphertexts the adversary cannot tell them 
%      apart?
%    - What about the size of the recipient set?  Padding and dummy entries 
%      should work?
%    - Probably the definition in [ANOBE] can help?
%    - Timing-attacks on the push model?  Alice writes n same-sized 
%      objects to n places within a short timespan.  How to formalize?
%  - Hidden credentials: the keys are by definition hidden, how to 
%    formalize?  In identity-based access control, is the identity itself 
%    the credential or whatever authenticates the identity?  Or both?
%  - Hidden decisions: the adversary should not be able to guess the 
%    outcome, an IND-style property.  This shouldn't be any problem to 
%    turn into a security game.

%\citet{TowardsPPACwHPHCHD} identified three desirable and (conjectured 
%sufficient) properties for a privacy-preserving \ac{AC} mechanism:
%\begin{inparablank}
%\item hidden policies,
%\item hidden credentials, and
%\item hidden decisions.
%\end{inparablank}
%The work in~\cite{TowardsPPACwHPHCHD} focused on \ac{FHE} and is thus not 
%directly feasible for our purposes.
%However, the properties are still relevant in our setting.
%
%Hidden policies means that the access policy remains hidden from anyone but 
%the owner and the subjects learn at most if they have access or not.
%So the subjects cannot learn which other subjects can access the same object.
%We mentioned in \cref{Introduction} that we have a logical reference monitor 
%based solely on cryptography and that, due to this fact, everyone can read all 
%the ciphertexts in the system.
%This means that the ciphertexts must not reveal the policy for the object 
%(i.e.\ who can decrypt).
%Furthermore, if the storage node is also a subscriber, then by observing who 
%accesses an object reveals at least parts of the policy.
%
%Let us briefly illustrate why the hidden policies property is important.
%Bob is not allowed access to an object, but he can see that Alice is.
%Then Bob can go to Alice and either ask her about it, or otherwise force her to 
%reveal it to him --- e.g.\ by stealing her keys.
%Even if Bob was allowed access to the object it is important that he does not 
%learn the policy.
%If Bob knows that Alice also has access to the object, then he can reveal the 
%object's contents to Eve and tell Eve to blame Alice for leaking the data to 
%her.
%But if Bob did not know who else has access, then he might be the only one and 
%thus less prone to leak to Eve as he is the only one to blame.
%
%Hidden credentials means that the subject never has to reveal the access 
%credentials to anyone.
%In our case this is a cryptographic key, and as a consequence we allow the 
%subject to anonymously read the ciphertext from the storage node.
%This means that the storage node cannot track which subjects are requesting 
%access to which objects.
%If users are not anonymized, then the storage node can approximate the 
%credential (the subject's identity).
%
%Hidden decisions means that no-one but the subject must learn the outcome of an 
%access request.
%This means that no-one should learn whether or not a subject could decrypt the 
%ciphertext or not.
%However, if everyone only requests ciphertexts that they know they can decrypt 
%(which is the most efficient strategy), then the storage node can easily guess 
%the decision.
%This decision together with non-anonymized users allows the storage node to 
%infer parts of the policy, hence breaking the hidden policy property.
%% XXX Is anonymization sufficient for hidden decisions too?
%For this reason, in addition to anonymization, subjects must also request 
%ciphertexts they cannot decrypt.
%This can be done either by dummy requests (i.e.\ requesting objects at random) 
%or by \ac{PIR} techniques, such as \ac{OT}.

%\subsection{The Push and Pull Models}
%
%We achieve hidden credentials in the Pull Model: a user's private key serves as 
%a credential to allow access.
%As all users can read any object and every user is indistinguishable, there is 
%no need for a user to show any type of credential to the storage node and the 
%storage node cannot identify any of the users.
%Thus the access credentials are hidden.
%
%The hidden decisions property is also achieved in the Pull Model: a user 
%requests a ciphertext, the storage node does not learn whether the user can 
%decrypt it or not.
%Thus the access decisions are hidden.
%However, if all users only download ciphertexts they can decrypt, then this 
%will not hold:
%if the storage-node operator always guesses that the decision was 
%\enquote{allowed}, then he will be correct in the majority of the cases.
%Thus subscribers must either do dummy requests randomly or use \iac{PIR} scheme 
%when reading the object from the storage node.
%
%The last property, hidden policies, states that no user shall learn anything 
%about who else is authorized or unauthorized to access a certain object.
%Due to the anonymity property of the \ac{ANOBE} scheme, the ciphertexts do not 
%reveal the intended recipient and, consequently, not the access policy.
%
%However, there is some meta-information that can be inferred in the Pull Model.
%First, we reveal the cardinality of the recipient set to everyone.
%Since everyone can read the \ac{TOC}, they can also read the number of 
%sub-ciphertexts contained in the ciphertext \(c_{\Key{m}}\).
%One way we can remedy this is by a padding scheme, or dummy entries, in the 
%\ac{ANOBE} ciphertext.
%We can always use \(n\) recipient slots.
%If the cardinality \(|S|\leq n\) of the recipient set is smaller than \(n\), 
%then we add dummy entries to get \(n\) ciphertexts.
%If, on the other hand, the cardinality \(|S| > n\) is strictly larger than 
%\(n\), then we use two indistinguishable \ac{ANOBE} ciphertexts containing 
%\(2n\) entries together.
%Since Eve cannot distinguish whether the two ciphertexts are two parts of one 
%object or two separate objects, we do not reveal the cardinality of the 
%recipient set.
%
%Second, assume that we are friends with Alice, so we can read and decrypt 
%entries in her \ac{TOC} \(i_A\).
%If Alice decides to remove us as a friend, then we will notice that we can no 
%longer decrypt the entries showing up in her \ac{TOC}.
%Hence we can infer that Alice has removed us from her list of friends.
%Alice can then argue that she is posting to several groups in the same 
%\ac{TOC}, and that she just stopped posting to our group, but this allows us to 
%monitor Alice's communication activities.
%
%Finally, the storage contracts that Alice manages to negotiate, as in 
%\cite{DataAvailability}, for the object \(o_m\) determines the privacy Alice 
%enjoys as a publisher.
%
%We make the same arguments as for the Pull Model that the Push Model also 
%achieves hidden credentials and hidden decisions.
%The argument for the hidden policy property is also similar, but here the users 
%cannot even see more than their own inboxes.
%Thus we do not need to use the padding scheme.
%However, if Bob uses the same inbox for both Alice and Eve, then Eve can infer, 
%with negligible probability of error, from the verification key 
%\(\VerifKey{m}\) that the same message resides in both their inboxes.
%Hence, Eve can learn part of the access policy.
%If Eve can make an exhaustive search of all objects in the system, then she can 
%learn parts of the policy even if every user uses unique inboxes for each 
%friend.
%But unless she knows who owns a particular inbox, she learns only the 
%cardinality.
%
%There are several advantages regarding the Push Model.
%First, if a user has a different inbox for every friend, then no friend can 
%infer activity by monitoring a user's inbox.
%This also means that if we stop receiving messages from a friend, then we 
%cannot distinguish between that friend stopping posting entirely or just 
%stopping posting to us.
%
%Second, in the Push Model, if we do not rely on randomness 
%re-use~\cite{RandomnessReuse} for efficiency and use unique 
%signature-verification keys per-recipient instead of per-message, we no longer 
%reveal the cardinality of the recipient set --- not even in the case of 
%exhaustive search.
%As stated above, we do not know the names of the other inboxes, nothing reveals 
%even if there is more than our inbox who received the message.
%Even if Eve would control the majority of the storage nodes, she cannot 
%distinguish the origin of two requests and, thus, she cannot relate all 
%requests to a single event --- except possibly by a timing attack, but that can 
%be remedied in the mix-net used for anonymity.
%
%
%
%
%\section<article>{Cryptographic Primitives}\label{BE}\label{Prerequisites}
%% XXX Rewrite BE and ANOBE sections with Alice and Bob instead of u
%
%The purpose of \ac{BE} is to develop methods to efficiently transmit data to 
%dynamically changing target audiences \(S\subseteq U\) who are allowed to read 
%the data, whereas the remaining users \(U\setminus S\) are not.
%This scenario fits our use-case well.
%However, most research in \ac{BE} has been focused on efficiency and not 
%privacy.
%Thus schemes usually require that the set \(S\) is revealed as it is required 
%for the members \(s\in S\) to know \(S\) for decryption.
%\citet{PseudonymousBE} adapted a \ac{BE} scheme using pseudonyms, but 
%pseudonyms yield a limited anonymity property~\cite{PseudonymousBE}.
%Due to this limitation of \ac{BE} schemes, \citet{ANOBE} developed \ac{ANOBE}, 
%where the goal was to allow decryption without knowing \(S\).
%% XXX Why is ANOBE expensive?
%The complexity of the \ac{ANOBE} scheme is linear in the size of the recipient 
%set, i.e.~\(O(|S|)\), whereas constant-size ciphertexts\footnote{%
%  This does not include the description of the set \(S\), which is needed for 
%  decryption.
%  So we should add another \(|S|\) bits per ciphertext (if we do not use any 
%  type of coding scheme to reduce it further).
%}, \(O(1)\), has been achieved for non-anonymous \ac{BE}.
%
%\subsection{Anonymous Broadcast Encryption}\label{ANOBE}
%
%\mode<presentation>{%
%\begin{frame}<presentation>{Anonymous Broadcast Encryption}
%  \begin{figure}
%   \begin{tikzpicture}[scale=0.75]
%      \draw (0,0) circle [radius=1];
%      \draw (0,0) node []{$S$};
%      \draw (-3,-2) rectangle (3,2);
%      \draw (-2,0) node []{$U$};
%    \end{tikzpicture}
%    \mode<article>{%
%    \caption{%
%      The subset of users \(S\subseteq U\).
%    }\label{fig:SubsetS}
%    }
%  \end{figure}
%  \begin{itemize}
%    \item A \ac{BE} scheme wants to distribute a key \(k\) to a subset of users 
%      \(S\subseteq U\).
%
%    \item This results in a ciphertext \(c_k\) and a description of \(S\).
%
%      \pause{}
%
%    \item \acf{ANOBE}~\cite{ANOBE} does the same without the description of 
%      \(S\).
%
%    \item By revealing \(S\) we reveal the policy.
%  \end{itemize}
%\end{frame}
%}
%
%The main idea of \ac{ANOBE} is to distribute a key \(k\) to a subset of users, 
%such that they can decrypt the broadcast message encrypted with \(k\) but no 
%one else can.
%The users who receive \(k\) should not be able to figure out who else received 
%\(k\) and who did not.
%This is a suitable mechanism for our system model as publishing a message to 
%all friends is basically a broadcast to all or a subset of friends in the 
%network.
%The \ac{ANOBE} scheme was specifically designed to hide the recipient set.
%This is an important property to achieve hidden policies as outlined in 
%\cref{HiddenPCD}.
%
%% XXX Review details on ANOBE
%% - Is ANOBE perfect as it is or does it need changes for the context?
%We will now describe how \ac{ANOBE} works.
%An overview of the encryption function is given in Fig.~\ref{fig:EncANOBE} and 
%an overview of the decryption function in Fig.~\ref{fig:DecANOBE}.
%
%\subsubsection<article>{Encryption}
%
%We must first generate a signature-verification key-pair \((\SignKey{k}, 
%  \VerifKey{k})\), then we choose a random permutation \(\pi\colon S\to S\).
%Next we must encrypt the key and the verification key \((k, \VerifKey{k})\) for 
%every user \(u_i\in S\) in the recipient set \(S\subseteq U\) under their 
%respective public key, \(c_{u_i} = \Enc[\PubKey{u_i}]{k, \VerifKey{k}}\).
%We let the \ac{ANOBE} ciphertext be the tuple \((\VerifKey{k}, C, \sigma_k)\), 
%where
%\(C = ( c_{\pi(u_1)}, \ldots, c_{\pi(u_{|S|})})\) and
%\(\sigma_k = \Sign[\SignKey{k}]{ C }\).
%Note that the signatures does not authenticate the sender, it is used to verify 
%correct or incorrect decryption.
%We will get back to the details about this shortly.
%
%\begin{frame}
%  \mode<presentation>{\frametitle{ANOBE Encryption}}
%  \begin{figure}
%    \begin{algorithmic}[1]
%      \Function{$\EncOp^{\mathrm{ANOBE}}_S$}{$k$}
%      \Comment{%
%        Recipient set $S$,
%        $k$ to be encrypted.
%      }
%        \State{%
%          $(\SignKey{k}, \VerifKey{k})\gets{\Gen{\lambda}}$
%        }
%        \Comment{%
%          Signature-verification key-pair\mode<article>{, security parameter 
%            $\lambda$}
%        }
%        \State{%
%          Choose a random permutation \(\pi\colon S\to S\).
%        }
%
%        \pause{}
%
%        \For{$u_i \in S$}
%          \State{%
%            $c_{u_i}\gets{ \Enc[\PubKey{u_i}]{k, \VerifKey{k}} }$
%          }
%        \EndFor{}
%
%        \pause{}
%
%        \State{%
%          $C\gets{( c_{\pi(u_1)}, \ldots, c_{\pi(u_{|S|})} )}$
%        }
%        \Comment{%
%          $\forall i\in\{ 1, \ldots, |S|\}\colon u_i\in S$
%        }
%        \State{%
%          $\sigma\gets{ \Sign[\SignKey{k}]{C} }$
%        }
%        \State{%
%          \Return{$(\VerifKey{k}, C, \sigma_k)$}
%        }
%      \EndFunction{}
%    \end{algorithmic}
%    \mode<article>{%
%    \caption{%
%      An algorithmic overview of the encryption algorithm in the \ac{ANOBE} 
%      scheme.
%    }\label{fig:EncANOBE}
%    }
%  \end{figure}
%\end{frame}
%
%\begin{frame}<presentation>{ANOBE Requirements}
%  \begin{itemize}
%    \item The encryption scheme \(\EncOp\) must be 
%      key-private~\cite{KeyPrivacy}.
%
%    \item Key-privacy ensures that a ciphertext does not reveal for which 
%      public key is was created.
%
%      \pause{}
%
%    \item The encryption scheme \(\EncOp\) further requires 
%      robustness~\cite{RobustEncryption}.
%
%    \item Robustness ensures that we cannot create one ciphertext valid for two 
%      different public keys.
%  \end{itemize}
%\end{frame}
%
%For \ac{ANOBE} to be anonymous, we require an encryption scheme which is 
%key-private~\cite{KeyPrivacy}.
%This property ensures that, given a ciphertext, an eavesdropper cannot tell 
%which public key, out of a set of public keys, was used to generate the 
%ciphertext.
%This also means that a user \(u\in U\) must try to decrypt the ciphertexts to 
%decide whether \(u\in S\) or \(u\notin S\).
%For this reason we also need the encryption scheme to be 
%robust~\cite{RobustEncryption}.
%Robust encryption ensures that we cannot create a ciphertext which is valid for 
%two different recipients:
%if we have a set of possible recipients \(R\), we encrypt a message for one of 
%them, \(r\in R\), then all \(r^\prime\neq r\) must decrypt the ciphertext to an 
%invalid message under their keys.
%
%\subsubsection<article>{Decryption}
%
%We now have data which we parse as the tuple \((\VerifKey{k}, C, \sigma_k)\).
%If \(\Verify[\VerifKey{k}]{ C, \sigma_k } = 0\), then we return \(\bot\) as the 
%verification failed.
%For each \(c_i\) (from \(C\)):
%Compute \(M = \Dec[\PriKey{u}]{ c_i }\).
%If \(M \neq \bot\) and \(M = (k, \VerifKey{k})\), then return \(k\).
%Otherwise, try the next \(c_i\).
%If there are no more \(c_i\) to try, then return \(\bot\).
%
%\begin{frame}
%  \mode<presentation>{\frametitle{ANOBE Decryption}}
%  \begin{figure}
%    \begin{algorithmic}[1]
%      \Function{$\DecOp^{\mathrm{ANOBE}}_{\PriKey{u}}$}{$\VerifKey{k}, C, 
%        \sigma_k$}
%      \Comment{%
%        Private key \(\PriKey{u}\),
%        ciphertext $(\VerifKey{k}, C, \sigma_k)$.
%      }
%        \pause{}
%
%        \If{$\Verify[\VerifKey{k}]{ C, \sigma_k } = 0$}
%          \State{%
%            \Return{$\bot$}
%          }
%        \EndIf{}
%
%        \pause{}
%
%        \For{$c_i\in C$}
%          \Comment{%
%            For each sub-ciphertext, or use tag-hints.
%          }
%          \State{%
%            $M\gets{\Dec[\PriKey{u}]{c_i}}$
%          }
%          \Comment{%
%            Try to decrypt
%          }
%          \If{$M = \bot$}
%            \State{\Return{$\bot$}}
%          \ElsIf{$M = (k, \VerifKey{k})$}
%            \State{\Return{k}}
%          \EndIf{}
%        \EndFor{}
%        \State{\Return{$\bot$}}
%      \EndFunction{}
%    \end{algorithmic}
%    \mode<article>{%
%    \caption{%
%      An algorithmic overview of the decryption algorithm in the \ac{ANOBE} 
%      scheme.
%    }\label{fig:DecANOBE}
%    }
%  \end{figure}
%\end{frame}
%
%%\begin{frame}<presentation>{ANOBE Decryption}
%%  \begin{itemize}
%%    \item Instead of trial-and-error decryption, we can use the tag-hint system 
%%      from~\cite{ANOBE}.
%%
%%    \item Apply a function on the \ac{ANOBE} ciphertext and our private key, 
%%      the result reveals the sub-ciphertext we can decrypt.
%%
%%  \end{itemize}
%%\end{frame}
%
%To decrypt an \ac{ANOBE} ciphertext, we need a trial-and-error decryption 
%procedure to decide if the ciphertext was indeed intended for us.
%This is costly as it makes the decryption function complexity \(O(|S|)\).
%\citet{ANOBE} presented a tag-hint system along with their \ac{ANOBE} scheme.
%The tag-hint system reduced the complexity back to \(O(1)\).
%Each recipient needs a tag-hint public-private key-pair.
%The system works by letting the recipient do a constant-time computation on the 
%received message and the private tag-hint key which gives a value.
%If this value is not found among the sub-ciphertexts, then the recipient was 
%not among the designated recipients.
%
%\subsection{Other Primitives}
%
%There are other primitives which are promising too.
%If we would relax our requirement for hidden policies, then the outsider 
%\ac{ANOBE}~\cite{oANOBE} scheme by \citeauthor{oANOBE} could also be used.
%The problem with our current situation is that not even subjects with reading 
%rights are allowed to learn the policy, which would be the case with only 
%outsider anonymity.
%
%Yet other interesting primitives are various \ac{ABE} schemes.
%The requirements for this category of schemes is that they need to be 
%decentralized and policy hiding.
%There is some work done in this area, but to our knowledge, current \ac{ABE} 
%schemes are not policy hiding.
%% XXX Does key-privacy imply hidden policies?
%
%
%\section[Scheme Design]{Design of the \acl*{AC} Mechanism}\label{Design}
%% XXX Review the Design section
%% - Check that we recap what functional and security properties we want.
%
%We will now describe the \ac{AC} mechanism design.
%We will start by describing the mechanism for the Pull Model followed by 
%a security analysis and an evaluation of the algorithmic complexity.
%Then we will do the same for the Push Model.
%For both models Alice is the publisher and Bob is the subscriber.
%Alice wants to share a message \(m\) with (at least) Bob.
%She will encrypt\footnote{%
%  Alice uses an authenticated encryption scheme.
%} \(m\) under a shared key \(k_m\) and publish the resultant ciphertext 
%somewhere in the storage system.
%Then she will use the either the Pull or Push Model to distribute the key 
%\(k_m\) to Bob and all other subscribers.
%I.e.\ we will use key encapsulation.
%Without loss of generality, we will assume that the adversary Eve operates the 
%storage node in our descriptions below.
%
%\subsection{\Acl*{AC} in the Pull Model}\label{PullModelAC}
%
%In this model, Alice distributes the link and key to the object by posting them 
%to her \ac{TOC} (index).
%The protocol is illustrated in \cref{PullModelSetup,PullModelPublication}, the 
%details follow.
%
%\begin{frame}
%  \mode<presentation>{\frametitle{Set-up in Pull Model}}
%\begin{figure}
%  \centering
%  \begin{sequencediagram}
%    \newinst{A}{Alice}
%    \newinst[1]{B}{Bob}
%
%    \mess{A}{}{B}
%    \node[anchor=east] at (mess from) {%
%      $\PubKey{A}, \VerifKey{A}$
%    };
%
%    \mess{B}{}{A}
%    \node[anchor=west] at (mess from) {%
%      $\PubKey{B}, \VerifKey{B}$
%    };
%
%    \mess{A}{}{B}
%    \node[anchor=east] at (mess from) {%
%      \shortstack{$c = \Enc[\PubKey{B}]{ i_A }$,\\
%        $\sigma = \Sign[\SignKey{A}]{ c }$}
%    };
%  \end{sequencediagram}
%  \mode<article>{%
%    \caption{%
%      The set-up phase in the Pull Model.
%    }\label{PullModelSetup}
%  }
%\end{figure}
%\end{frame}
%
%\begin{frame}
%  \mode<presentation>{\frametitle{Publication in Pull Model}}
%\begin{figure}
%  \centering
%  \begin{sequencediagram}
%    \newinst{A}{Alice}
%    \newinst[1]{E}{Eve}
%    \newinst[1]{B}{Bob}
%
%    \mess{A}{write}{E}
%    \node[anchor=east] at (mess from) {%
%      $c_m = \AuthEnc[\Key{m}]{ m }$
%    };
%    \node[anchor=west] at (mess to) {%
%      $o_m$
%    };
%
%    \mess{A}{append}{E}
%    \node[anchor=east] at (mess from) {%
%      $( \VerifKey{m}, \{c_{u_{\pi(j)}}\}_j, 
%      \sigma_{\{c_{u_{\pi(j)}}\}_{u_j\in S}} )$
%    };
%    \node[anchor=west] at (mess to) {%
%      $i_A$
%    };
%
%    \mess{B}{read}{E}
%    \node[anchor=west] at (mess from) {%
%      $i_A$
%    };
%
%    \mess{E}{}{B}
%    \node[anchor=east] at (mess from) {%
%      $( \VerifKey{m}^\prime, \{c_{u_{\pi(j)}}^\prime\}_{u_j\in S}, 
%      \sigma_{\{c_{u_{\pi(j)}}\}_j}^\prime )$
%    };
%
%    \mess{B}{read}{E}
%    \node[anchor=west] at (mess from) {%
%      $o_m^\prime$
%    };
%
%    \mess{E}{}{B}
%    \node[anchor=east] at (mess from) {%
%      $c_m^\prime$
%    };
%
%  \end{sequencediagram}
%  \mode<article>{%
%    \caption{%
%      The publication phase in the Pull Model.
%      Eve operates all the storage nodes, thus she will return possibly 
%      modified data, indicated by the primes (\(\prime\)).
%    }\label{PullModelPublication}
%  }
%\end{figure}
%\end{frame}
%
%\subsubsection<article>{Set-up}
%
%We assume that Alice and Bob already have securely exchanged public keys and 
%signature-verification keys,
%so let \((\PubKey{A}, \PriKey{A})\) denote Alice's public and private key, 
%respectively, and let \((\PubKey{B}, \PriKey{B})\) be Bob's.
%Let \((\SignKey{A}, \VerifKey{A})\) denote Alice's signing key and verification 
%key, respectively, and let \((\SignKey{B}, \VerifKey{B})\) be Bob's.
%When Alice wants to add Bob as a friend, she sends the name of her \ac{TOC} 
%object \(i_A\) to Bob:
%she computes the ciphertext \(c = \Enc[\PubKey{B}]{ i_A }\) and the signature 
%\(\sigma = \Sign[\SignKey{A}]{ c }\), then she sends \((c, \sigma)\) to Bob.
%An overview is given in \cref{PullModelSetup}.
%
%\subsubsection<article>{Publication}
%
%When Alice wants to publish a message \(m\) to a subset \(S = \{u_1, \ldots, 
%  u_l\}\subseteq U\) of her friends she does the following.
%First Alice creates a secret key \(\Key{m}\) and a one-time 
%signature-verification key-pair \(\SignKey{m}, \VerifKey{m}\).
%
%Next, Alice computes \(c_m = \AuthEnc[\Key{m}]{ m }\), where \(\AuthEncOp\) is 
%an authenticated encryption scheme, and she creates a new object with the name 
%\(o_m\) which contains \(c_m\).
%
%After that, she computes \(c_{u_j} = \Enc[\Key{u_j}]{ k_m, v_m, o_m }\) for 
%\(1\leq j\leq l\).
%She then computes the ciphertext \(c_{\Key{m}} = (\VerifKey{m}, c_{u_{\pi(1)}}, 
%  \ldots, c_{u_{\pi(l)}}, \sigma_{\Key{m}})\), where \(\sigma_{\Key{m}} 
%  = \Sign[\SignKey{m}]{ c_{u_{\pi(1)}}, \ldots, c_{u_{\pi(l)}} }\) is 
%a signature and \(\pi\colon \{1,\ldots,l\}\to \{1,\ldots,l\}\) is a random 
%permutation.
%% XXX Add the tag-hint system to the pull-model scheme
%
%Finally, Alice generates a signature \(\sigma_{c_{\Key{m}}} 
%  = \Sign[\SignKey{A}]{ c_{\Key{m}} }\) using her signing key \(\SignKey{A}\) 
%which must be shared with at least all her friends in \(S\).
%Then she appends \((c_{\Key{m}}, \sigma_{c_{\Key{m}}})\) to her \ac{TOC} object 
%\(o_{i_A}\).
%
%An overview is given in \cref{PullModelPublication}.
%
%\subsection{\Acl*{AC} in the Push Model}\label{sec:PushModelAC}
%% XXX Replace MAC with MA where applicable
%
%Now Alice wants to use the Push Model for sharing data with her friends.
%The Push Model is conceptually different from the Pull Model in that Alice will 
%publish the link to the object in all her friends' inboxes, instead of in her 
%own \ac{TOC}.
%This also allows us to use a \ac{MAC} instead of a digital signature for 
%authentication.
%(It is possible in the Pull Model too, but the \ac{MAC} key must be shared 
%between all recipients.)
%The benefit is that we remove the non-repudiation property, but this comes at 
%the cost of slightly more computations.
%The protocol is illustrated in \cref{PushModelSetup,PushModelPublication}, the 
%details follow.
%
%\begin{frame}
%  \mode<presentation>{\frametitle{Set-up in Push Model}}
%\begin{figure}
%  \centering
%  \begin{sequencediagram}
%
%    \newinst{A}{Alice}
%    \newinst[1]{B}{Bob}
%
%    \mess{A}{}{B}
%    \node[anchor=east] at (mess from) {%
%      $\PubKey{A}, \VerifKey{A}$
%    };
%
%    \mess{B}{}{A}
%    \node[anchor=west] at (mess from) {%
%      $\PubKey{B}, \VerifKey{B}$
%    };
%
%    \mess{A}{}{B}
%    \node[anchor=east] at (mess from) {%
%      \shortstack{%
%        $c_{\MACKey{B}} = \Enc[\PubKey{B}]{ \MACKey{B} }$, \\
%        $\sigma_{\MACKey{B}} = \Sign[\SignKey{A}]{ c_{\MACKey{B}} }$
%      }
%    };
%
%    \mess{B}{}{A}
%    \node[anchor=west] at (mess from) {%
%      \shortstack{$c_{i_B} = \Enc[\PubKey{A}]{ i_B }$,\\
%        $\sigma_{i_B} = \Sign[\SignKey{B}]{ c_{i_B} }$}
%    };
%
%  \end{sequencediagram}
%  \mode<article>{%
%    \caption{%
%      The set-up phase in the Push Model.
%      The difference from Fig.~\ref{PullModelSetup} is that Alice generates 
%      a \ac{MAC} key and Bob gives Alice the address of his inbox.
%    }\label{PushModelSetup}
%  }
%\end{figure}
%\end{frame}
%
%\begin{frame}
%  \mode<presentation>{\frametitle{Publication in Push Model}}
%\begin{figure}
%  \centering
%  \begin{sequencediagram}
%    \newinst{A}{Alice}
%    \newinst[1]{E}{Eve}
%    \newinst[1]{B}{Bob}
%
%    \mess{A}{write}{E}
%    \node[anchor=east] at (mess from) {%
%      $c_m = \AuthEnc[\Key{m}]{ m }$
%    };
%    \node[anchor=west] at (mess to) {%
%      $o_m$
%    };
%
%    \mess{A}{append}{E}
%    \node[anchor=east] at (mess from) {%
%      $( \VerifKey{m}, c_{B}, \mu_{c_B} )$
%    };
%    \node[anchor=west] at (mess to) {%
%      $i_B$
%    };
%
%    \mess{B}{read}{E}
%    \node[anchor=west] at (mess from) {%
%      $i_B$
%    };
%
%    \mess{E}{}{B}
%    \node[anchor=east] at (mess from) {%
%      $( \VerifKey{m}^\prime, c_{B}^\prime, \mu_{c_{B}^\prime} )$
%    };
%
%    \mess{B}{read}{E}
%    \node[anchor=west] at (mess from) {%
%      $o_m^\prime$
%    };
%
%    \mess{E}{}{B}
%    \node[anchor=east] at (mess from) {%
%      $c_m^\prime$
%    };
%
%  \end{sequencediagram}
%  \mode<article>{%
%    \caption{%
%      The publication phase in the Push Model.
%      Eve operates all the storage nodes, thus she will return possibly 
%      modified data, indicated by the primes (\(\prime\)).
%      The difference from Fig.~\ref{PullModelPublication} is that Alice 
%      appends to \(i_B\) instead of \(i_A\), she only includes \(c_B\) instead 
%      of \(\{c_i\}_{i\in S}\), and she uses \(\mu_{c_B}\) instead of 
%      \(\sigma_{\{c_i\}_{i\in S}}\).
%    }\label{PushModelPublication}
%  }
%\end{figure}
%\end{frame}
%
%\subsubsection<article>{Set-up}
%
%The set-up phase for the Push Model is similar to that of the Pull Model.
%When Alice wants to add Bob as a friend, she generates and sends a \ac{MAC} key 
%\(\MACKey{B}\) to Bob:
%she computes the ciphertext \(c_{\MACKey{B}} = \Enc[\PubKey{B}]{ \MACKey{B} }\) 
%and the signature \(\sigma_{\MACKey{B}} = \Sign[\SignKey{A}]{ c_{\MACKey{B}} 
%  }\), then she sends \((c_{\MACKey{B}}, \sigma_{\MACKey{B}})\) to Bob.
%Bob then replies with the name of his inbox object, \(i_B\):
%he computes \(c_{i_B} = \Enc[\PubKey{A}]{ i_B }\) and \(\sigma_{i_B} 
%  = \Sign[\SignKey{B}]{ c_{i_B} }\) and sends them to Alice.
%An overview is given in \cref{PushModelSetup}.
%
%\subsubsection<article>{Publication}
%
%Again, Alice wants to publish a message \(m\) to a subset \(S = \{u_1, \ldots, 
%  u_l\}\subseteq U\) of her friends.
%She does the same as she did in the Pull Model, except for the final step.
%Instead of signing with her signing key \(\SignKey{A}\), she generates 
%a \ac{MAC} \(\mu_{c_{\Key{m}}, u_j}\) for each inbox \(u_j\in S\) using the 
%respective \ac{MAC} key \(\MACKey{u_j}\): \(\mu_{c_{\Key{u_j}}} 
%  = \MAC[\MACKey{u_j}]{ c_{\Key{u_j}} }\).
%Finally, for each \(u_j\in S\) Alice appends \((c_{\Key{u_j}}, 
%  \mu_{c_{\Key{u_j}}})\) to the inbox of \(u_j\).
%An overview is given in \cref{PushModelPublication}.
%
%\subsection[Complexity]{Algorithmic Complexity}\label{AlgComplexity}
%% XXX Actually do some measurements
%% XXX Write an introduction to the section on complexities
%
%\subsubsection{Key Management, Encryption and Decryption}
%
%The performance is interesting to evaluate from two perspectives: the 
%publisher's (Alice in all examples) and the subscriber's (Bob in all examples).
%From the publisher's perspective, it is interesting to investigate the 
%complexity of key-storage size, communication complexity for publication and 
%time complexity for encryption of new material.
%From the subscriber's perspective, the complexity of key-storage size and the 
%time-complexity of aggregating the newest published messages are the most 
%interesting aspects.
%An overview of the results is presented in \cref{Complexities}.
%
%\begin{frame}
%  \mode<presentation>{\frametitle{Publisher's View}}
%\begin{table}
%  \centering
%  \mode<article>{%
%  \caption{%
%    The storage, communication and time complexities in the two models.
%    \(U\) is the set of friends of a user (the universal set), \(S\) is the set 
%    of recipients of a message.
%  }\label{Complexities}
%  }
%  \begin{tabular}{lrr}
%    Publisher
%    & \head{Pull Model}
%    & \head{Push Model} \\
%    
%    \toprule
%
%    Key-storage size
%    & \(2|U|\) & \(|U|\) \\
%
%    \pause{}%
%    Ciphertext size
%    & \(2|S|\) & \(|S|\) \\
%
%    Encryption
%    & \(2|S|\) & \(2|S|\) \\
%
%    \pause{}%
%    Communication
%    & \(1\) & \(|S|\) \\
%
%    \bottomrule
%
%  \end{tabular}
%  \begin{tabular}{lrr}
%    Subscriber
%    \mode<article>{& \head{Pull Model} & \head{Push Model}} \\
%    %& \head{Pull Model} & \head{Push Model} \\
%    
%    \toprule
%
%    Key-storage size
%    & \(2\) & \(1\) \\
%
%    \pause{}%
%    Ciphertext size
%    & \(2|S|\) & \(1\) \\
%
%    Decryption
%    & \(2\) & \(1\) \\
%
%    \pause{}%
%    Communication
%    & \(1\) & \(1\) \\
%
%    \bottomrule
%
%  \end{tabular}
%\end{table}
%\end{frame}
%
%Due to their differences, our two approaches described above have some 
%different complexity properties.
%But they also have some similarities, and we will start with those.
%The space complexity for the key management is the same for both the Pull and 
%Push Model.
%If we have \(|U|\) friends, then we need to exchange and store \(O(|U|)\) keys:
%we need one public key per friend.
%For the Pull Model, if we use the tag-hint system we also need an extra key for 
%that, hence \(O(2|U|)\) keys.
%
%%\begin{frame}<presentation>{Algorithmic Complexity}
%%  \begin{itemize}
%%    \item Space complexity for ciphertexts is \emph{linear} in the cardinality 
%%      of the recipient set.
%%
%%      \pause{}
%%
%%    \item The same for the time complexity for encryption: \emph{linear} in the 
%%      recipient set.
%%
%%    \item There is a higher constant factor for the Push Model, since we need 
%%      to compute the signatures per-recipient.
%%
%%      \pause{}
%%
%%    \item The decryption is constant-time.
%%
%%    \item It's slightly higher in the Pull Model, since we need to compute the 
%%      hint.
%%
%%    \item Decryption would be linear in the Pull Model without the tag-hint 
%%      system.
%%
%%  \end{itemize}
%%\end{frame}
%
%The space complexity for the ciphertexts are \(O(|S|)\) for both models, 
%although the one for the Pull Model is \(O(2|S|)\) here as well due to the 
%tag-hint system.
%The time complexity for encryption is also \(O(2|S|)\) for the Pull Model and 
%\(O(3|S|)\) for the Push Model.
%The reason is that in the Push Model we have to compute the signatures for the 
%ciphertexts separately, and we also have to compute the \acp{MAC} for all 
%ciphertexts.
%The time complexity for decryption on the other hand is constant time, 
%\(O(1)\), for the Push Model.
%For the Pull Model, we need \(O(|S|)\) trial-and-error decryption operations.
%If we add the \ac{ANOBE} tag-hint system, then we can achieve constant-time 
%decryption even here --- but it is still \(O(2)\) as opposed to \(O(1)\) for 
%the Push Model.
%This also increases the needed space for the ciphertexts in the Pull Model to 
%\(O(2|S|)\) for the publisher, as opposed to \(O(1)\) for the subscribers in 
%the Push Model.
%
%%\begin{frame}<presentation>{Algorithmic Complexity}
%%  \begin{itemize}
%%    \item The communication complexity is \emph{linear} in both models.
%%
%%      \pause{}
%%
%%    \item In the Pull Model the publisher needs only one connection.
%%
%%    \item In the Push Model the publisher needs \(|S|\) connections.
%%
%%  \end{itemize}
%%\end{frame}
%
%Finally, the communication complexity for the different models differ slightly.
%Asymptotically they are the same, \(O(|S|)\).
%But we get a slight overhead for the publisher in the Push Model.
%We need to make \(|S|\) connections instead of one: depending on the underlying 
%system, e.g.\ running Tor on a mobile device, this can be expensive.
%Further, we have to transfer the verification key and signature to every inbox 
%instead of just putting a copy in our \ac{TOC}.
%For the subscriber it is the opposite: in the Push Model the subscriber needs 
%one connection to the inbox, but in the Pull Model requires \(O(|U|)\) 
%connections to all the friends' \acp{TOC}.
%
%\subsubsection{Group Management}\label{GroupManagement}
%
%\mode<presentation>{%
%\begin{frame}<presentation>{Modifying Groups}
%  \begin{itemize}
%    \item Alice wants to give Bob access to some old material.
%
%      \pause{}
%
%    \item Then Alice has to encrypt the message keys and send them to Bob.
%
%  \end{itemize}
%\end{frame}
%}
%
%Our scheme is, as are general \ac{BE} schemes, designed to broadcast a message 
%to a dynamically changing group.
%This means that we can easily change the recipient group.
%However, when a new user, say Bob, is added to the group, we might want to give 
%Bob access to old messages.
%In our scheme we have to encrypt the message key \(k_m\) for every message 
%\(m\) that we want Bob to access.
%
%\begin{frame}<presentation>{Modifying Groups}
%  \begin{itemize}
%    \item Alice wants to revoke Eve's clearances.
%
%      \pause{}
%
%    \item Now Alice has to re-encrypt all objects Eve had access to.
%
%    \item In the Push Model, Alice also has to update other subscribers' 
%      inboxes.
%
%  \end{itemize}
%\end{frame}
%
%Similarly as for adding a user, when we remove a user, say Eve, from a group, 
%we might want to remove Eve's clearances for the objects.
%First, we can argue whether we should remove these or not.
%Since Eve has had access to an object, she might already keep a copy anyway.
%If we remove her clearance, then she learns that we have removed her from the 
%group.
%However, if we want to remove Eve's clearance, then we will have to re-encrypt 
%all objects she had access to.
%This is easier in the Pull Model, because in the Push Model we also need to 
%update the entries in all other users' inboxes.
%
%%\subsection{Extensions}\label{sec:Extensions}
%%% XXX Write about possible extensions to improve the scheme
%%% XXX Add that we can replace PKE with SKE in the Push Model
%%For performance reasons, we also look into a trade-off between using a robust, 
%%key-private IND-CCA2 \ac{PKE} scheme and a semantically secure symmetric 
%%encryption scheme in the \ac{ANOBE} construction.
%%The reason for this is that the symmetric operations are faster than the 
%%asymmetric ones.
%%This is an important factor for \acp{DOSN} during e.g.~news-feed aggregation 
%%when a user comes online.
%%In these situations we have to handle large amounts of data, which can lead to 
%%performance problems.
%%
%%% XXX Write about MAC for group deniability
%%If we replace the signatures by \acp{MAC}, then we can achieve group 
%%deniability: anyone in the group could have posted the message.
%
%
%\section<article>{Related Work}\label{RelatedWork}
%% XXX Adapt related work to be in the end
%
%There is some related work already done in this area, ranging from general 
%solutions to specific solutions in \ac{P2P} file-sharing systems or \acp{DOSN}.
%We summarize these below.
%
%The work of \citet{TowardsPPACwHPHCHD} is focused on a general solution for 
%privacy-preserving \ac{AC}.
%They build their solution on \ac{FHE}, and as such, the scheme is not yet 
%practically feasible.
%However, there is also a body of work which is practically feasible and is 
%applied in our setting.
%There are several proposals available for \acp{DOSN}, e.g.\ 
%DECENT~\cite{DECENT}, Cachet~\cite{Cachet} and Persona~\cite{Persona}.
%The \ac{AC} mechanisms in these proposals focus on providing confidentiality 
%for the data.
%E.g.\ Persona uses \ac{KP-ABE} to implement the \ac{AC} mechanism and
%unfortunately, this yields lacking privacy: as this is not policy-hiding, 
%anyone can read the \ac{AC} policies and see who may access what data.
%
%There is also a body of work done in storage for \ac{P2P} file-sharing systems, 
%e.g.~\cite{TunableACinP2P}.
%\citet{TunableACinP2P} aimed at the trade-off between privacy and search in 
%\ac{P2P} file-sharing systems.
%So it is not directly applicable in our use-case and neither does it fulfil the 
%hidden credentials requirement.
%
%%\citet{PPACTechniquesInDistSys} developed a variety of \ac{ABAC} schemes.
%%Although some of them can handle multiple authorities, unfortunately, they are 
%%too centralized for our purposes.
%
%\citet{PEAC} adapted \ac{PE} for the \ac{AC} mechanism in \acp{DOSN}.
%% XXX What properties did the PE scheme achieve?
%% - The scheme achieved the properties X.
%% XXX What are the problems with the PE scheme?
%% - More flexibility but less scalability?
%For the \ac{PE} scheme, the storage complexity for keys is exponential and it 
%requires \(O(2^g)\) keys, where \(g\) is the number of groups a user is 
%a member of.
%The key size is also bound to the expressibility of the policy, e.g.\ we can 
%estimate from the size of my key how many other individuals and groups are 
%accounted for in the policy.
%As this scheme reveals such meta-information, it is not entirely satisfactory.
%Also, it focused on the encryption scheme being policy hiding, i.e.\ that the 
%ciphertext does not reveal the policy, and not the combination with the other 
%properties.
%
%\Ac{ABE} and \ac{PE} schemes are still more efficient for extending the 
%audience of previous publications.
%In those schemes they simply add the correct attributes or predicates to the 
%new subscriber's key.
%In our \ac{ANOBE} based scheme we have to encrypt and send all keys for the 
%previously published material.
%But for dynamically changing the group, the \ac{BE} schemes are better, e.g.\ 
%to add or remove a specific recipient.
%The \ac{ABE} and \ac{PE} schemes must prepare attributes or predicates for each 
%user to be able to be as specific as the \ac{BE} schemes.
%Finally, however, when it comes to removing the reading rights of a subscriber, 
%all schemes are equally inefficient: they need to re-encrypt all previously 
%posted material.


\section{Conclusions}\label{Conclusions}

\dots

We do not have to consider the re-encryption problem when changing group 
constellations: this would violate the property that jealous Bob cannot infer 
a change in the access policy.

%% XXX Review the conclusions
%% - Ensure that the solution covers the needed properties.
%% XXX Add slides for the conclusions
%
%% XXX Improve the problem statement
%% - Clearly describe the problem before any solution
%% - What functional and security properties do we want?
%% - What properties do we need?
%%   - Do we need anonymity?
%%   - Can we make a trade-off which is good enough?
%
%% XXX Review the overview of our contributions and outline of the paper
%% - We have simplicity, more performance?
%% - What are the benefits of our scheme?
%% - What is the problem that this paper solves and how?
%% - The initial concern was performance, right?  But given the right library 
%%   this goes away, iirc.
%% - Needs more thinking.
%
%We first analysed the high-level requirements to provide hidden policies, 
%hidden credentials and hidden decisions in our setting.
%From these requirements we gave a high-level description of what the storage 
%system must be like.
%We then suggested two models for publication and subscription of content in 
%that system --- the Pull and Push Model --- and analysed what privacy 
%properties can be achieved and how in the different models and the requirements 
%for the cryptographic primitives used to implement it.
%
%From these findings we created an \ac{AC} mechanism for a decentralized storage 
%system, relying on minimal trust in other entities.
%We presented an example implementation using the \ac{ANOBE} primitive.
%Using this example we showed how to achieve different privacy properties and 
%estimated the algorithmic complexity of the two models.
%Asymptotically the two approaches have the same time and space complexities, 
%but they differ by constant factors.
%
%\mode<presentation>{%
%\subsection{Conclusions}
%\begin{frame}<presentation>{Conclusions}
%  \begin{itemize}
%    \item We focus on the access control mechanisms in a decentralized storage 
%      system for \acp{DOSN}.
%
%    \item We use only cryptographic mechanisms, we don't rely on any \acp{TTP} 
%      as reference monitors.
%
%    \item We achieve Hidden-Credentials, Hidden-Policies, and Hidden-Decisions 
%      for our scheme.
%
%    \item For this we use \ac{ANOBE}~\cite{ANOBE}.
%  \end{itemize}
%\end{frame}
%}
%
%In the Pull Model we achieve the three desired properties hidden credentials, 
%hidden decisions and hidden policies, but we reveal meta-information such as 
%our activity in the system and we are required to use a padding scheme to hide 
%the cardinality of the recipient set.
%Without the padding scheme we reveal this piece of information about the 
%policy.
%
%In the Push Model, we achieve the first two properties similarly as for the 
%Pull Model..
%For the hidden policy property we must be careful in the design of the system.
%If Alice re-uses an inbox for several friends, then those friends learn that 
%part of her access policy.
%But if we ensure that we use one inbox per friend and that it is 
%computationally infeasible to search through all objects or that we use unique 
%randomness and signature-verification keys, then we can have policy hiding in 
%the Push Model too.
%
%Although both models are asymptotically equally complex, in practice, the Push 
%Model is probably slightly more costly for the publisher while cheaper for the 
%subscriber.
%But this should be investigated in details, currently they are only theoretical 
%estimates.
%Also, the estimates are done using \ac{ANOBE} in the implementation.
%Another primitive might yield other results.
%However, in the Push Model, might be able to gain some performance by replacing 
%the robust and key-private \ac{PKE} with a symmetric cipher.
%But it is important that the symmetric cipher has properties corresponding to 
%robustness and key-privacy, as these properties are crucial to the security of 
%the \ac{ANOBE} scheme.
%
%\subsection{Future Work}
%% XXX Review the future work section
%
%\mode<presentation>{%
%\begin{frame}{Future Work}
%  \begin{itemize}
%    \item Measuring an actual implementation.
%
%      \pause{}
%
%    \item A stronger Eve:
%      \begin{itemize}
%        \item controlling the majority of the storage nodes,
%        \item actively malicious.
%      \end{itemize}
%
%      \pause{}
%
%    \item Stronger deniability properties?
%
%      \pause{}
%
%    \item Accountability properties?
%
%  \end{itemize}
%\end{frame}
%}
%
%We can see several interesting tracks to explore from this work.
%First, we would like to have stronger deniability properties in the scheme.
%In the Pull Model, any deniability relies on the indistinguishability of the 
%users, that a third party cannot distinguish between users.
%However, Bob can prove to the third party what Alice has said, since she has 
%signed her messages to provide authentication.
%This could have a negative effect on free speech.
%We should be able to fix this by using a \ac{MAC} scheme for authentication of 
%messages, then we no longer have the non-repudiation property.
%Since we use a \ac{MAC}, Bob cannot prove anything to the third party, since he 
%himself holds a copy of the \ac{MAC} key.
%In the Push Model, we can easily remove the non-repudiation property in this 
%way.
%But it must be extended to a larger group by sharing the \ac{MAC} key within 
%the group in the Pull Model.
%We would like to explore further the possibilities for Alice to deny her 
%activities in the system.
%
%The second track would be the opposite direction of deniability: namely, 
%accountability.
%One reason for this is that Bob might want to verify that Alice received the 
%same message, if Eve told him that she sent a copy to Alice as well.
%
%
%\mode<presentation>{%
%\begin{frame}
%  \begin{center}
%    Questions?
%  \end{center}
%\end{frame}
%}
%

\mode<article>{%
\subsubsection{Acknowledgements}

This work was inspired by work with Benjamin Greschbach and work of Oleksandr 
Bodriagov and Gunnar Kreitz.
We would like to thank the Swedish Foundation for Strategic Research for grant 
SSF FFL09-0086 and the Swedish Research Council for grant VR 2009-3793, which 
funded this work.
We would also like to thank the anonymous reviewers for valuable feedback.
}

\begin{frame}
\printbibliography{}
\end{frame}
