\title{%
  Privacy-Preserving Access Control
  %for Asynchronous Message Passing
  %in Decentralized Storage
  in Publicly Readable Storage Systems
  %for Asynchronous Message Passing
  %for Online Social Networks
}
\author{%
  Daniel Bosk \and
  Sonja Buchegger
}
\institute{%
  School of Computer Science and Communication\\
  KTH Royal Institute of Technology, Stockholm, Sweden\\
  \email{\{dbosk,buc\}@kth.se}%
}
\date{IFIP Summer School\\Edinburgh, 20th August 2015}

\maketitle

\mode* % required for slides to compile without non-frame text

\begin{abstract}
  In this paper, we focus on achieving privacy-preserving access control 
  mechanisms for decentralized storage, primarily intended for an asynchronous 
  message passing setting.
  We propose two modular constructions, one using a pull strategy and the other
  a push strategy for sharing data.
  These models yield different privacy properties and requirements on the 
  underlying system.
  We achieve hidden policies, hidden credentials and hidden decisions.
  We additionally achieve what could be called \enquote{hidden policy-updates}, 
  meaning that previously-authorized subjects cannot determine if they have 
  been excluded from future updates or not.

  \keywords{%
    Privacy,
    Access Control,
    Cloud Storage,
    Decentralized Storage,
    %Anonymous Broadcast Encryption,
    Hidden Policies,
    Hidden Credentials,
    %Hidden Decisions
  }
\end{abstract}

% Since this a solution template for a generic talk, very little can
% be said about how it should be structured. However, the talk length
% of between 15min and 45min and the theme suggest that you stick to
% the following rules:
% - Exactly two or three sections (other than the summary).
% - At *most* three subsections per section.
% - Talk about 30s to 2min per frame. So there should be between about
%   15 and 30 frames, all told.

% 1) what is the concrete setting that we consider (entities and their props),
% 2) what overall functionality and security properties do we want,
% 3) how are these properties realized (protocols, composition)?
%
% - Good synonyms for "looking at":
%   - investigating, exploring, evaluating, researching, additionally;
%   - if in relation to something else: contrasting, comparing.

\acresetall{}
\section{Introduction}\label{Introduction}
% XXX Rewrite the introduction

Alice and her friends want to communicate asynchronously.
To do this they want to use a publicly available file system and write their 
messages to different files, which the other party later can read.
We are interested in enforcing access-control policies in such a public file 
system which does not have any built-in access control mechanisms.
Our approach is to introduce a layer of encryption as a logical reference 
monitor.
Beyond the expected confidentiality, Alice wants some stronger privacy 
properties as well: her friends should not be able to monitor her activity, 
e.g.\ infer with whom else she communicates, the fact that she communicates 
with others.

We will assume a simple file system with no built-in access control.
This system is discussed and defined in \cref{FileSystem}.
We will continue to define an ideal communication functionality in 
\cref{IdealCommunication} after which we will model our constructions.
Then we give two constructions that implement the functionality using different 
strategies and analyse what properties their primitives must have in 
\cref{PullAnalysis,PushAnalysis}.
The motivation behind these two strategies is that one is optimized for Alice 
and the other for Alice's friends.
One requires potentially many connections for her friends, while the other 
requires this for Alice.
In some situations connections can be expensive, compare e.g.\ establishing 
many Tor~\cite{Tor} circuits to transfer little data and establishing one 
circuit to transfer more data.
We summarize the algorithmic complexities in \cref{AlgComplexity}.
Finally we compare our results to related work in \cref{RelatedWork} and 
conclude by summarizing the main contributions and future work in 
\cref{Conclusions}.


%%% The main contribution %%%%%%%%%%%%%%%%%%%%%%%%
% XXX Formalize the setting, the system model:
%  - A file system: read, write, append.
%  - Adversary is storage node, operates file system.
%  - Adversary participates as user.

\section{A General File System}\label{FileSystem}

We will model our system as an abstract file system with the operations create, 
read and append.
The operations are defined as we intuitively expect: we can create an object, 
then read it and also append to it.
The file system itself provides no access control.

\begin{definition}[Public file system]\label{FS}
  A \emph{file} \(f = (i, m)\) consists of the \emph{identifier} \(i\) and the 
  \emph{content} \(m\).
  We define the set of files \(F = \{(i, m_i)\}\) together with the following 
  operations to be the \emph{public file system} \(\FS\):
  \begin{itemize}
    \item \(\fswrite[i][m_i]\) will set \(F\gets F\cup \{(i, m_i)\}\) if 
      \(\{(i^\prime, m^\prime)\in F \mid i^\prime = i\} = \emptyset\), i.e.\ 
      the file \(i\) does not yet exist.
      Otherwise this operation will set \((i, m)\gets (i, m\concat m^\prime)\) 
      for \((i, m)\in F\).

    \item \(\fsread[i] = m_i\) if \((i, m_i)\in F\), otherwise 
      \(\fsread[i^\prime] = \bot\).

  \end{itemize}
\end{definition}

We can see in the definition that anyone can read any object in the file 
system.
Anyone can also create new files and append to existing files.

% XXX Formalize assumptions:
%  - Storage node cannot tell the origin of two requests apart.
%  - Eve controls the entire storage system?
%    - Here she can do an exhaustive search.
%  - Eve only controls part of the network?
%    - Can Eve do an exhaustive search in the storage?
%
%Assuming that Eve controls the entire storage system should be more 
%interesting as it's a stronger adversary.  There shouldn't be any need 
%for the weaker adversary who controls only a part of the storage system.  
%We'll have to remove some randomness re-use from ANOBE for the 
%implementation, but that should suffice.

We will let the adversary Eve operate the file system \(\FS\) defined in 
\cref{FS}.
By this we mean that Eve has access to the internal state of \(\FS\) directly, 
i.e.\ the set of files \(F\).
She can thus read all files (same as everyone else), but she can also do 
arbitrary searches since she does not have to use \(\fsread{\cdot}\).
She can also modify the files (including deleting them).

To be consistent with \cref{FS}, Eve cannot distinguish between Alice's and 
Bob's requests to the operations of \(\FS\) --- in the definition there is 
nothing to identify who used any of the operations.
But she can record the times at which they occurred with her own clock, since 
she executes them.



% XXX Formalize what we want to do:
%  - Publisher makes a message m asynchronously available to n recipients 
%    while hiding the recipient set:
%    - Pull model: publish m as object o.
%    - Push model: publish m as objects o_1, ..., o_n.
%  - Each recipient reads m:
%    - Pull model: read object o.
%    - Push model: read object o_i.
%  - The publisher should be unidentifiable for a non-recipient.
%
% XXX Top-down approach:
%  - First, analyse high-level properties in pull and push models.
%    - Motivation for hidden policies, credentials, decisions.
%    - Alice removes Bob from recipient set: Bob shouldn't distinguish this or 
%      if Alice stopped posting to all.
%    - Bob shouldn't be able to monitor Alice's activities unrelated to Bob.
%    - Eve knows Alice's and Bob's inboxes, for any message m in Alice's, she 
%      shouldn't be able to tell if it's also in Bob's.
%  - Second, suggest an example of possible implementation using (modified) 
%    ANOBE.
%  - Third, complexity for solution.
%
\section{The Ideal Communication Model}\label{IdealCommunication}

There are several ways Alice can implement the message passing with her 
friends.
We will start by presenting an ideal model of communication 
(\cref{CommunicationModel}) whose properties Alice wants to achieve.
Then we will proceed to the details of two alternative protocols 
(\cref{PullModel,PushModel}) that yields the properties of the ideal model.

% XXX Shall we have an adversary interface for C?
\begin{definition}[Communication model]\label{CommunicationModel}
  Let \(\C[p, S]\) be a \emph{\((p, S)\)-communication model} with 
  a \emph{publisher} \(p\) and a \emph{set of subscribers} \(S\).
  Then we have the following operations defined on \(\C[p, S]\):
  \begin{itemize}
    \item The publisher first runs \(\csetup[p, S]{p}[1^\lambda]\) and each 
      subscriber \(s\in S\) runs \(\csetup[p, S]{s}[1^\lambda]\), where 
      \(\lambda\) is the security parameter.
    \item The publisher \(p\) uses \(\cpub[p, S]{p}[R][m]\) to publish the 
      message \(m\) to the designated \emph{recipient set} \(R\) by making it 
      available to all recipients \(r\in R\subseteq S\).
    \item Each subscriber \(s\in S\) uses \(\cget[p, S]{s}[]\) to get the set 
      \(M\) of published messages \(m\in M\) for which \(s\) was in the 
      recipient set.
  \end{itemize}
\end{definition}

Whenever \(p, S\) are clear from the context, we will simply omit them, e.g.\ 
\(\csetup{p}\) and \(\csetup{s}\).

There are two ways the adversary can gain information.
The first is from corrupted subscribers and the second is from the public file 
system over which the communication model is implemented.
Hence the second is not visible in \cref{CommunicationModel}, but will be in 
\cref{PullModel,PushModel}.

In our desired scenario, Alice acts as a publisher and her friends as the 
subscribers.
We want the following properties:
\begin{description}
  \item[Message privacy]
    No subscriber \(s^\prime\in S\) can use \(\cget{s}\) for \(s^\prime\neq s\).
    I.e.\ no subscriber can read the messages of any other subscriber, we call 
    this property \emph{message privacy}.

  \item[Hidden policies]
    No subscriber \(s_i\) must learn which \(s_1, \ldots, s_n\in R\) beyond 
    that \(i\in \{1, \ldots, n\}\) and \(s_i\in R\) for a given message \(m\).
    I.e.\ no subscriber must know who else received a message, we call this 
    property \emph{hidden policy}.
    Thus if Alice publishes a message, none of her friends know who else 
    received it and none can read the others' received messages to check if 
    they have received the same message.

% XXX Formalize other required (more high-level) properties:
%  - Alice removes Bob as a friend: Bob shouldn't distinguish this or if 
%    Alice stopped posting to all.
%  - Bob shouldn't be able to monitor Alice's activities.
%  - Eve knows Alice's and Bob's inboxes, for any message m in Alice's, 
%    she shouldn't be able to tell if it's also in Bob's.

  \item[Hidden policy-updates]
    In addition to the hidden-policy property, we want something we call 
    \emph{hidden policy-updates}.
    Consider the following example:
    Bob might become jealous if he realizes that Alice no longer includes him 
    in the recipient set for her messages.
    As such Bob should only be able to determine that Alice publishes content 
    by being in the recipient set himself.
\end{description}

There are several ways to implement the message-passing protocol for the 
communication model in \cref{CommunicationModel}.
We will focus on two alternative protocols, one using the pull model and the 
other using the push model for communication.
The push model is analogous to the subscription of magazines:
a subscriber contacts the publisher and signs a subscription, whenever the 
publisher issues a new magazine it sends a copy to the subscriber's mailbox.
The pull model is the converse of the push model.
It is analogous to the selling of magazines in kiosks:
the publisher issues magazines and the \enquote{subscribers} come to the kiosk 
and buy them whenever they want.
We can see that our ideal communication model allows both a pull and push 
strategy for implementation.
We will describe and analyse the pull construction in \cref{PullAnalysis} and 
the push construction in \cref{PushAnalysis}.
But first we need to review our needed building blocks.

\section{Construction and Analysis of the Pull Model}
\label{PullAnalysis}

In this model, each publisher has an \enquote{outbox} file in the file system.
This is simply a file object with a randomly chosen identifier.
The publisher adds new publications to the outbox while subscribers pull new 
content from the outbox.
In our analysis we will use the following definition for the Pull Model.

\begin{definition}[Pull Model]\label{PullModel}
  Let \(\mathcal{E} = (\KeygenOp, \EncOp, \DecOp)\) be an authenticated 
  broadcast-encryption scheme and \(\FS\) be a public file system.
  We denote by \(\Pull[p, S]\) the \emph{pull model protocol} implementing 
  a \((p, S)\)-communication model through the operations
  \begin{itemize}
    \item \(\psetup[p, S]{p}[\cdot]\) and \(\psetup[p, S]{s}[\cdot]\) for each
      \(s\in S\),
    \item \(\ppub[p, S]{p}[\cdot][\cdot]\), and
    \item \(\pget[p, S]{s}[\cdot]\) for each \(s\in S\)
  \end{itemize}
  as in \cref{PullFunctions}.
\end{definition}

\begin{figure}%[p]
  \framebox{%
  \begin{minipage}[t]{0.48\textwidth}
    \begin{algorithmic}
      \Function{$\psetup[p, S]{p}$}{$1^\lambda$}
        \State{$(\SignKey{p}, \VerifKey{p})\rgets \Keygen{1^\lambda}$}
        \State{$i_p\rgets \{0, 1\}^{\lambda}$}
        \State{$\forall s\in S$: give $(\VerifKey{p}, i_p)$ to $\Pull[p, 
          S][s]$.}
      \EndFunction{}

      \Statex{}

      \Function{$\ppub[p, S]{p}$}{$R, m$}
        \State{$c\gets \Enc[\{\PubKey{s}\}_{s\in R}, \SignKey{p}]{m}$}
        \State{$\fsappend[i_p][c]$}
      \EndFunction{}
    \end{algorithmic}
  \end{minipage}
  \begin{minipage}[t]{0.48\textwidth}
    \begin{algorithmic}
      \Function{$\psetup[p, S]{s}$}{$1^\lambda$}
        \State{$(\PubKey{s}, \PriKey{s})\rgets \Keygen{1^\lambda}$}
        \State{Give $\PubKey{s}$ to $\Pull[p, S][p]$.}
      \EndFunction{}

      \Statex{}

      \Function{$\pget[p, S]{s}$}{}
        \State{$C = \fsread[i_p]$}
        \State{$M\gets \emptyset$}
        \For{$c\in C$}
          \State{$m_c = \Dec[\PriKey{s}, \VerifKey{p}]{c}$}
          \State{$M\gets M\cup \{m_c\}$}
        \EndFor{}
        \State{\Return{$M$}}
      \EndFunction{}
    \end{algorithmic}
  \end{minipage}
  }
  \caption{%
    Functions implementing the communication model for the Pull Model protocol.
    The publisher's interface is to the left and the subscribers' to the right.
  }\label{PullFunctions}
\end{figure}

When Alice executes \(\psetup{p}\) the \(\KeygenOp\) algorithm generates 
a signature-verification key-pair \((\SignKey{p}, \VerifKey{p})\).\footnote{%
  Since the scheme \(\mathcal{E}\) is an authenticated \ac{BE} scheme, the key 
  generation yields a proper broadcast keys for encryption and decryption in 
  addition to a signature-verification key-pair --- in Alice's case (conversely 
  for her friends) we are only interested in the authentication part, so we 
  simply discard her encryption keys in the definition.
}
She also randomly chooses an identifier.
Then the verification key and identifier are both given to all her friends.
Each of her friends \(s\in S\), when they execute \(\psetup{s}\) the 
\(\KeygenOp\) algorithm generates a public-private key-pair.\footnote{%
  Actually this is ju the notation we use.
  It might just as well be Alice's execution that generates the broadcast keys 
  and distributes them to her friends, while her friends' \(\psetup{s}\) does 
  nothing.
}
They give their respective public key to Alice.

When Alice wants to publish a message \(m\) to the recipient set \(R\subseteq 
  S\), she runs \(\ppub{p}[R][m]\).
This operation simply creates an authenticated broadcast ciphertext 
\(\sigma\gets \Enc[\{\PubKey{s}\}_{s\in R}, \SignKey{p}]{m}\) by combining the 
broadcast key \(\{\PubKey{s}\}_{s\in R}\) and the signature key 
\(\SignKey{p}\).
She appends the ciphertext to her outbox file.
Each friend can then use \(\pget{s}\) to retrieve the ciphertexts from the file 
system and decrypt them.

As we mentioned earlier (and in \cref{PullModel}), \ac{BE} schemes are designed 
for the Pull Model.
We wanted to ensure that no subscriber should be able to learn who else is 
a subscriber.
We know from the analysis by \citet{ANOBE} that Eve has negligible advantage in 
the game in \cref{ANO-IND-CCA} when the \ac{ANOBE} scheme is used in 
\cref{PullModel}.
(We will not restate their proof here, but refer to the full version of their 
paper~\cite{ANOBE}.)

However, \ac{ANOBE} used in the Pull Model cannot immediately yield negligible 
advantage to Jealous Bob.
Bob can read Alice's outbox, conclude that there are entries which he cannot 
decrypt and thus he can become jealous.
(In \cref{PullFunctions} we see that Bob can count the number of \(\bot\) in 
the output of \(\pget{\cdot}\).)
One approach to prevent this is that Alice uses a unique outbox per friend, so 
Bob has his own outbox.
This actually reduces the problem to the Push Model.
Unfortunately this is not entirely trivial, which we will see in our analysis 
in \cref{PushModel}.
We will instead analyse a simpler solution for the Pull Model.

\subsection{Changing Recipient Set in the Pull Model}
\label{ChangingPullRecipientSet}

We will now propose and analyse a solution to the problem of Jealous Bob.
This solution, however, is based on the intricacies of the \ac{ANOBE} 
construction, and thus require us to use \ac{ANOBE} in the Pull Model above.
(For convenience the \ac{ANOBE} scheme is described in \cref{ANOBE}.)

When Bob get jealous he goes to Eve to ask her help, this means that we must 
prevent Eve from learning how Alice changed her policy too.
When Alice wants to change her policy and remove Jealous Bob from the 
subscriber set \(S\), she will update the identifier of her outbox file.
Essentially she will divide \(S\) into two new sets \(S_0, S_1\subset S\) such 
that \(S = S_0\cup S_1\), where Jealous Bob will be alone in one of them.
Then she will generate two new outbox identifiers \(i_0, i_1\) --- one for the 
recipients in \(S_0\) and one those in \(S_1\).
She will then create a broadcast ciphertext which decrypts to \(i_0\) for 
recipients in \(S_0\) and to \(i_1\) for recipients in \(S_1\).
From them on Alice will publish to the two outbox files \(i_0\) and \(i_1\) 
separately.
If she no longer wants to post to Bob, she will simply fade out her 
publications in her new Bob-specific outbox.
We summarize this algorithm in \cref{SplitGroup}.

\begin{figure}
  \framebox{%
    \begin{minipage}{0.96\textwidth}
      \begin{algorithmic}
        \Function{$\SplitGroupOp$}{$S_0, S_1$}
          \Comment{%
            Recipient sets $S_0, S_1$ such that $S = S_0\cup S_1$ and $S_0\cap 
            S_1 = \emptyset$.
          }
          \State{%
            $i_0\rgets \{0, 1\}^\lambda$,
            $(\SignKey{0}, \VerifKey{0})\rgets \Keygen{1^\lambda}$
          }
          \Comment{Generate new outbox and key-pair.}
          \State{%
            $i_1\rgets \{0, 1\}^\lambda$,
            $(\SignKey{1}, \VerifKey{1})\rgets \Keygen{1^\lambda}$
          }
          \Comment{One per new group.}
          \State{%
            $(\SignKey{}, \VerifKey{})\gets{\Keygen{1^\lambda}}$
          }
          \Comment{%
            One-time signature-verification key-pair.
          }
          \State{%
            Choose a random permutation $\pi\colon S\to S$.
          }

          \For{$s \in S$}
            \If{$s\in S_0$}
              \State{%
                $c_{s}\gets{ \Enc[\PubKey{s}]{i_0, \VerifKey{0}, \VerifKey{}}}$
              }
            \Else\Comment{$s\in S_1$}
              \State{%
                $c_{s}\gets{ \Enc[\PubKey{s}]{i_1, \VerifKey{1}, \VerifKey{}}}$
              }
            \EndIf{}
          \EndFor{}

          \State{%
            $C\gets{\mleft( c_{\pi(s)}\mright)_{s\in S}}$
          }
          \Comment{Put all subciphertexts in random order.}
          \State{%
            $\sigma\gets{ \Sign[\SignKey{}]{C} }$
          }
          \State{%
            \Return{$(i_0, \SignKey{0}), (i_1, \SignKey{1}), (\VerifKey{}, C, 
              \sigma)$}
          }
        \EndFunction{}
      \end{algorithmic}
    \end{minipage}
  }
  \caption{%
    An algorithm splitting a subscriber set \(S\) into two new \(S_0, S_1\).
  }\label{SplitGroup}
\end{figure}

The property we want from this algorithm is that Bob cannot determine how the 
subscribers are divided.
Thus he cannot know whether he is removed or not.
We capture this in the following game.

\begin{definition}[Split-group indistinguishability]\label{SplitGroupSecurity}
  Let \(\mathcal{E} = (\SetupOp, \KeygenOp, \EncOp, \DecOp)\) be an \ac{ANOBE} 
  scheme.
  We define the security of the split-group algorithm in \cref{SplitGroup} as 
  follows:
  \begin{description}
    \item[Setup] The challenger runs \(\Setup{1^\lambda}\) of the \ac{ANOBE} 
      scheme.
      This generates the public and private keys for all recipients.
      The public keys are given to the adversary.

    \item[Phase 1] The adversary may corrupt recipients (request secret keys) 
      and request decryptions of ciphertexts for any recipient (without 
      corruption).

    \item[Challenge] The adversary gives the challenger a set of uncorrupted 
      recipients \(R = \{r_0, \ldots, r_n\}\).
      The challenger randomly chooses \(B\rgets (B_0, \ldots, B_n) \in \{0, 
        1\}^n\) and constructs \(R_0, R_1\) as follows:
      \(R_i = \{ r_j\in R\colon B_j = i\}\), where \(i\in \{0, 1\}\).
      The challenger then runs \(C\gets \SplitGroup[R_0, R_1]\) and gives \(C\) 
      to the adversary.

    \item[Phase 2] The adversary may continue to corrupt recipients and request 
      decryptions under the following conditions:
      she cannot corrupt any \(r\in R\) and not request decryptions for 
      subciphertexts of the challenge ciphertext \(C\).

    \item[Guess] The adversary outputs \(\hat{B}\) and wins if \(\hat{B} 
        = B\).
  \end{description}
  We define the adversary's advantage as \[\Adv{\text{Split-IND-CCA}}{\A, 
      \SplitGroupOp}[1^\lambda] = \mleft|\Pr[ \hat{B} 
    = B ] - \frac{1}{2^n}\mright|.\]
\end{definition}

Now we prove the following theorem about the security of the split-group 
algorithm.

\begin{theorem}\label{SplitGroupIsSecure}
  The algorithm \(\SplitGroupOp\) is Split-IND-CCA if the encryption algorithm 
  is ANO-IND-CCA secure.
\end{theorem}

\begin{proof}[sketch]
  The proof is by contradiction and follows that of Thm.~2 and Lem.~4 and 5 of 
  the full version of~\cite{ANOBE}, we refer to those for details.
  Assume that there exists an adversary \(\A\) that can win the Split-IND-CCA 
  game with non-negligible probability.
  We will now describe how to construct an algorithm \(\A*\) that can win the 
  IND-CCA game using \(\A\).

  We construct a sequences of games \(0\leq k\leq n\) with the following idea.
  In game \(0\) all recipients are in \(R_0\) and none in \(R_1\), while in 
  game \(n\) all recipients are in \(R_1\) and none in \(R_0\).
  If \(\A*\) uses the public key \(\PubKey{*}\) and challenge ciphertext 
  \(c_*\) (which is either \(i_0\) or \(i_1\)) from its own challenger as one 
  of the possible recipients, then for some games, say \(k\) and \(k+1\), 
  \(\A*\) will not know whether it is game \(k\) or game \(k+1\) that is 
  currently played (if \(c_* = i_0\) it is game \(k\), and game \(k+1\) if 
  \(c_* = i_1\)).
  If \(\A\) can distinguish between games \(k\) and \(k+1\), then \(\A\) knows 
  whether \(c_* = i_0\) or \(c_* = i_1\) and thus \(\A*\) can distinguish 
  between IND-CCA ciphertexts.
%  Game \(k\) is as follows:
%  \begin{description}
%    \item[Setup] \(\A*\) will generate \(n-1\) key-pairs itself and get one 
%      public key from its challenger, \(\PubKey{*}\).
%      \(\A*\) gives all public keys to \(\A\).
%
%    \item[Phase 1] \(\A*\) will just forward decryption requests from \(\A\) to 
%      its own oracle.
%      If \(\A\) wants the private key for \(\PubKey{*}\), then \(\A*\) quits 
%      with failure.
%
%    \item[Challenge] When \(\A\) gives \(\A*\) its recipient set \(R\), \(\A*\)
%      will construct \(R_0, R_1\) in the following way.
%      If the recipient corresponding to \(\PubKey{*}\) is not in \(R\), then 
%      \(\A*\) quits with failure.
%      Otherwise, \(\A*\) sets \(B = (1, \ldots, 1, 0, \ldots, 0)\) with \(k\) 
%      leading \(1\)'s in game \(k\).
%      For some game \(k^\prime\) \(\A*\) will set the subciphertext \(c_k\) to 
%      its own challenge ciphertext, for which \(\A*\) does not know the 
%      plaintext.
%
%    \item[Phase 2] \(\A*\) continues to forward the requests from \(\A\) to its
%      own oracles.
%      However, no requests for decryptions of any of the subciphertexts of the 
%      challenge ciphertext is allowed.
%
%    \item[Guess] When \(\A\) outputs its result, \(\A*\) can compute its guess 
%      as follows.
%      If recipient \(r_*\in R_0\) \(\A*\) outputs \(0\), otherwise if \(r_*\in 
%        R_1\) \(\A*\) outputs \(1\).
%  \end{description}
  It follows that there cannot be any \(\A\) that can win the game with 
  non-negligible advantage.
  \qed{}
\end{proof}

Now we know that Bob cannot distinguish whether everyone in the recipient sent 
received the same message or not.
We will add the split-group algorithm (\cref{SplitGroup}) as an available 
interface in the Pull Model (\cref{PullModel}) and we summarize this as the 
Extended Pull Model in the following definition.

\begin{definition}[Extended Pull Model]\label{ExtPullModel}
  Let \(\Pull[p, S]\) be an instance of the Pull Model as in \cref{PullModel}.
  Then we define the \emph{Extended Pull Model} to additionally provide the 
  following interfaces:
  \begin{itemize}
    \item \(\psplit[p, S]{p}[\cdot, \cdot]\) to split the subscriber set,
    \item A modified \(\pget[p, S]{s}[]\) to update \(i_p\) after a split,
  \end{itemize}
  as defined in \cref{ExtPullFunctions}.
\end{definition}

\begin{figure}
  \framebox{%
    \begin{minipage}[t]{0.53\textwidth}
      \begin{algorithmic}
        \Function{$\psplit[p, S]{p}$}{$R_0, R_1$}
          \State{$(i_0, \SignKey{0}), (i_1, \SignKey{1}), C\gets 
            \SplitGroup[R_0, R_1]$}
          \State{$\fsappend[i_p][(C, \Sign[\SignKey{p}]{C})]$}
          \State{\Return{$\Pull[p, R_0], \Pull[p, R_1]$}}
        \EndFunction{}
      \end{algorithmic}
    \end{minipage}

    \begin{minipage}[t]{0.43\textwidth}
      \begin{algorithmic}
        \Function{$\pget[p, S]{s}$}{}
          \State{$C = \fsread[i_p]$}
          \State{$M\gets \emptyset$}
          \For{$c\in C$}
            \State{$m_c = \Dec[\PriKey{s}, \VerifKey{p}]{c}$}
            \If{$m_c$ is new identifier}
              \State{$(i_p, \VerifKey{p}) \gets m_c$}
            \Else{}
              \State{$M\gets M\cup \{m_c\}$}
            \EndIf{}
          \EndFor{}
          \State{\Return{$M$}}
        \EndFunction{}
      \end{algorithmic}
    \end{minipage}
  }
  \caption{%
    The additional and modified interfaces of the Extended Pull Model.
    Note that we implicitly let the \(\SplitGroup\) algorithm have access to 
    the public keys of the recipients.
    We also assume there exists a coding that can differentiate an file 
    identifier fro an ordinary message.
  }\label{ExtPullFunctions}
\end{figure}

Note that the execution of \(\psplit[p, S]{p}[R_0, R_1]\) results in two new 
instances of the Pull Model, namely \(\Pull[p, R_0]\) and \(\Pull[p, R_1]\).
After this happens, Alice should no longer use \(\Pull[p, S]\), instead she 
should only use these two new instances.

On the subscribers' side, due to the construction of \(\pget[p, S]{s}\), their 
instance \(\Pull[p, S]\) is automatically turned into \(\Pull[p, R_i]\), for 
whichever \(i\in \{0, 1\}\) Alice put them in.

\subsection{Running Multiple Instances in Parallel}
\label{ParallelPull}

Now it remains for us to convince ourselves that Bob cannot distinguish between
different instances of the Pull Model: i.e.\ Alice posting to two both \(R_0\) 
and \(R_1\) or Alice posting to \(R_0\) and Carol to \(R_1\).
We have the following result.

\begin{theorem}
  Given two instances \(\Pull[p, S]\) and \(\Pull[p^\prime, S^\prime]\), \(p\) 
  and \(p^\prime\) are unlinkable for any adversary \(\A\).
\end{theorem}

\begin{proof}
  We will prove this by an information-theoretic argument.
  We have two cases: first, the two instances are unrelated, \(p\neq 
    p^\prime\); second, the two instances are either related or unrelated while 
  \(p = p^\prime\).
  We focus on the latter.
  In this case there are two possibilities, \(p\) created two instances and 
  they are thus unrelated.
  If they are related \(p\) must have used \(\psplit[p, S^*]{p}\) to form them 
  from some \(S^* = S\cup S^\prime\).
  In doing this, \(p\) randomly chooses \(i_0, i_1\in \{0, 1\}^\lambda\) and 
  \(\SignKey{0}, \SignKey{1}\rgets \Keygen{1^\lambda}\).
  This means that \((i_0, \SignKey{0})\) and \((i_1, \SignKey{1})\) are 
  information-theoretically independent.
  It follows from \cref{SplitGroupIsSecure} and the ANO-IND-CCA property of the 
  encryption scheme that even to \(s\in S\), nothing related to 
  \(\Pull[p^\prime, S^\prime]\) can be found.
  Thus we could equally likely have \(p\neq p^\prime\).
  \qed{}
\end{proof}



\section{Construction and Analysis of the Push Model}
\label{PushAnalysis}

The idea of the Push Model is for each subscriber to have an inbox in the file 
system --- as opposed to the Pull Model, where the publisher has an outbox.
This is simply a file object with a randomly chosen identifier.
The publisher then puts all published material in the inbox of each subscriber.

We can see that if we simply put the broadcast ciphertext from the Pull Model 
in all inboxes, then Eve can relate them since they contain identical 
ciphertexts.
We thus have to make some more modifications.
Assume that we instead encrypt one message (with unique ciphertext) per 
subscriber.
Now Eve cannot use the ciphertext to relate the recipients.
However, after a few published messages (she has oracle access to the 
publication interface) she has learned who has which inbox, thus she can easily 
win the ANO-IND-CCA game (\cref{ANO-IND-CCA}).

To deal with this problem, we will introduce a mix-net.

\begin{definition}[Mix-net]
  \dots
\end{definition}

The mix-net will not do much help in a single instance, with this construction 
we must run several instances in parallel to achieve any security.
Even then, we cannot achieve negligible probability for Eve to win (as for the 
Pull Model).
All we can achieve is a lower bound.
It is a lower bound since the result depends on the behaviour in the parallel 
instances --- which will not be uniformly random.
We base our analysis on the following definition of the Push Model.

\begin{definition}[Push Model]\label{PushModel}
  Let \(\mathcal{E} = (\KeygenOp, \EncOp, \DecOp)\) be an authenticated 
  encryption scheme and \(\FSop\) be a public file system.
  We denote by \(\Push[p, S]\) the \emph{push model protocol} implementing 
  a \((p, S)\)-communication model through the operations
  \begin{itemize}
    \item \(\pnsetup[p, S]{p}[\cdot]\) and \(\pnsetup[p, S]{s}[\cdot]\),
    \item \(\pnpub[p, S]{p}[\cdot][\cdot]\), and
    \item \(\pnget[p, S]{s}[\cdot]\)
  \end{itemize}
  as in \cref{PushFunctions}.
\end{definition}

\begin{figure}%[p]
  \framebox{%
  \begin{minipage}[t]{0.48\textwidth}
    \begin{algorithmic}
      \Function{$\pnsetup[p, S]{p}$}{$1^\lambda$}
        \For{$r\in S$}
          \State{$(\SignKey{r}, \VerifKey{r})\rgets \Keygen{1^\lambda}$}
          \State{Give $\VerifKey{r}$ to $\Push[p, S][r]$.}
        \EndFor{}
      \EndFunction{}

      \Statex{}

      \Function{$\pnpub[p, S]{p}$}{$R, m$}
        \For{$r\in R$}
          \State{$i_r^\prime\rgets \{0, 1\}^\lambda$}
          \State{$c_r\gets \Enc[\PubKey{r}, \SignKey{r}]{i_r^\prime, m}$}
          \State{$\fswrite[i_r][c_r]$}
          \State{$i_r\gets i_r^\prime$}
        \EndFor{}
      \EndFunction{}
    \end{algorithmic}
  \end{minipage}
  \begin{minipage}[t]{0.48\textwidth}
    \begin{algorithmic}
      \Function{$\pnsetup[p, S]{s}$}{$1^\lambda$}
        \State{$(\PubKey{s}, \PriKey{s})\rgets \Keygen{1^\lambda}$}
        \State{$i_s\rgets \{0, 1\}^{\lambda}$}
        \State{Give $(\PubKey{s}, i_s)$ to $\Push[p, S][p]$.}
      \EndFunction{}

      \Statex{}

      \Function{$\pnget[p, S]{s}$}{}
        \State{$C\gets \fsread[i_s]$}
        \If{$C = \bot$}
          \State{\Return{$\bot$}}
        \EndIf{}
        \State{$M\gets \emptyset$}
        \For{$c\in C$}
          \State{$d\gets \Dec[\PriKey{s}, \VerifKey{s}]{c}$}
          \If{$d\neq \bot$}
            \State{$(i_s^\prime, m_c)\gets d$}
            \State{$M\gets M\cup \{m_c\}$}
            \State{$i_s\gets i_s^\prime$}
            \State{$M\gets M\cup \pnget{s}[]$}
          \EndIf{}
        \EndFor{}
        \State{\Return{$M$}}
      \EndFunction{}
    \end{algorithmic}
  \end{minipage}
  }
  \caption{%
    Functions implementing the communication model for the Push Model protocol.
    The publisher's interface is to the left and the subscribers' to the right.
  }\label{PushFunctions}
\end{figure}

When Alice executes \(\pnsetup{p}\) the \(\KeygenOp\) algorithm generates 
a signature-verification key-pair \((\SignKey{p}, \VerifKey{p})\) and gives the 
verification key to all her friends.\footnote{%
  Since the scheme \(\mathcal{E}\) is an authenticated encryption scheme, the 
  key generation yields a public-private key-pair and a signature-verification 
  key-pair --- in Alice's case (conversely for her friends) we are only 
  interested in the authentication part, so we simply discard her encryption 
  keys in the definition.
}
Each of her friends \(s\in S\), when they execute \(\pnsetup{s}\) the 
\(\KeygenOp\) algorithm generates a public-private key-pair.
Additionally they randomly choose a string as an identifier.
They give the public key and the identifier to Alice.

When Alice wants to send a message \(m\) to a subset \(R\subseteq S\) of her 
friends, she uses \(\pnpub{p}\) to create the authenticated ciphertexts 
\(\sigma_s\gets \Enc[\PubKey{s}, \SignKey{p}]{m}\) for each friend \(s\in R\).
The encryption scheme simply combines the recipients public key \(\PubKey{s}\) 
and Alice's signature key \(\SignKey{p}\).
Then she uses \(\FS\) to append the ciphertext \(\sigma_s\) to the file with 
identifier \(i_s\).

The operation \(\pget{s}\) works similarly as in the Pull Model, however, it 
uses a different file and since the inbox contains a pointer to the next inbox 
it recursively calls itself upon finishing a message.
It starts by reading the inbox from the file system.
Then it iterates through the list of entries, decrypting each entry.
Each entry, if successfully decrypted, is a new inbox and a message.
It appends the message to the list of messages and sets the inbox to be the new
inbox.
Then it recursively calls itself to check the newly acquired inbox for 
messages.

Note that the authenticated encryption scheme \(\mathcal{E}\) can be 
a symmetric-key scheme although we use the notation of public-key cryptography 
in our abstraction.
We can simply let \(\PubKey{} = \PriKey{}\) and \(\SignKey{} = \VerifKey{}\) to 
achieve this.
We have thus not yet limited ourselves to any properties of \(\mathcal{E}\) 
(beyond authentication and confidentiality).

% XXX Write high-level analysis of push model
Some high-level analysis \dots
Point out what properties we need from \(\mathcal{E}\).
These will probably be very similar as for \ac{ANOBE}.
(Eve can correlate all inboxes by observing equal ciphertexts.)

\subsection{Running Multiple Instances in Parallel}
\label{ParallelPush}

Multiple instances of the Push model run in parallel \dots

\begin{theorem}
  Given two instances \(\Pull[p, S]\) and \(\Pull[p^\prime, S^\prime]\), \(p\) 
  and \(p^\prime\) are unlinkable for any adversary \(\A\).
\end{theorem}

\begin{proof}
  We will prove this by an information-theoretic argument.
  We have two cases: first, the two instances are unrelated, \(p\neq 
    p^\prime\); second, the two instances are either related or unrelated while 
  \(p = p^\prime\).
  We focus on the latter.
  In this case there are two possibilities, \(p\) created two instances and 
  they are thus unrelated.
  If they are related \(p\) must have used \(\psplit[p, S^*]{p}\) to form them 
  from some \(S^* = S\cup S^\prime\).
  In doing this, \(p\) randomly chooses \(i_0, i_1\in \{0, 1\}^\lambda\) and 
  \(\SignKey{0}, \SignKey{1}\rgets \Keygen{1^\lambda}\).
  This means that \((i_0, \SignKey{0})\) and \((i_1, \SignKey{1})\) are 
  information-theoretically independent.
  It follows from \cref{SplitGroupIsSecure} and the ANO-IND-CCA property of the 
  encryption scheme that even to \(s\in S\), nothing related to 
  \(\Pull[p^\prime, S^\prime]\) can be found.
  Thus we could equally likely have \(p\neq p^\prime\).
  \qed{}
\end{proof}



\section{Algorithmic Complexity}\label{AlgComplexity}
% XXX Actually do some measurements
% XXX Write an introduction to the section on complexities

We will now summarize the algorithmic complexity for our constructions.
The performance is interesting to evaluate from two perspectives: the 
publisher's (Alice in all examples) and the subscriber's (Bob in all examples).
From the publisher's perspective, it is interesting to investigate the needed 
space for key storage, communication complexity for publication and time 
complexity for encryption of new material.
From the subscriber's perspective, the complexity of key-storage size and the 
time-complexity of aggregating the newest published messages are the most 
interesting aspects.
An overview of the results is presented in \cref{Complexities}.

\begin{table}
  \centering
  \caption{%
    The storage, communication and time complexities in the two models.
    \(S\) is the set of all subscribers, \(R\) is the set of recipients of 
    a message.
    All values are \(O(\cdot)\).
  }\label{Complexities}
  \begin{tabular}{lrr}
    \toprule

    Publisher
    & Pull & Push \\
    
    \midrule

    Key-storage size
    & \(|S|\) & \(|S|\) \\

    \pause{}%
    Ciphertext size
    & \(|R|\) & \(|R|\) \\

    Encryption
    & \(1\) & \(|R|\) \\

    \pause{}%
    Communication
    & \(1\) & \(|R|\) \\

    \bottomrule

  \end{tabular}
  \begin{tabular}{lrr}
    \toprule

    Subscriber
    & Pull & Push \\
    
    \midrule

    Key-storage size
    & \(1\) & \(1\) \\

    \pause{}%
    Ciphertext size
    & \(|R|\) & \(1\) \\

    Decryption
    & \(1\) & \(1\) \\

    \pause{}%
    Communication
    & \(1\) & \(1\) \\

    \bottomrule

  \end{tabular}
\end{table}

The space complexity for the key management is the same for both the pull and 
push protocols.
If we have \(|S|\) subscribers, then we need to exchange and store \(O(|S|)\) 
keys:
we need one public key per friend.

The space complexity for the ciphertexts are \(O(|R|)\) for both models.
However, the pull protocol is slightly more space efficient since we need less 
signatures.
In the push protocol we require one signature per ciphertext, i.e.\ \(O(|R|)\), 
whereas for the pull protocol we only need one per message.

The time complexity for encryption depends on the underlying schemes.
But we can see that in the push protocol we get a factor \(|R|\) to that of the 
encryption scheme, whereas we have a constant factor in the pull protocol.
The time complexity for decryption on the other hand has a constant factor for 
both models.

Finally, we look at the communication complexity for the different protocols, 
which differ slightly.
If we look at one single instance, then we get a constant number of connections
for the subscribers.
However, most subscribers will have to pull from several publishers, this is 
the argued benefit of push model of communication --- there is only one inbox 
to read.

%\subsection{Extensions}\label{sec:Extensions}
%% XXX Write about possible extensions to improve the scheme
%% XXX Add that we can replace PKE with SKE in the Push Model
%For performance reasons, we also look into a trade-off between using a robust, 
%key-private IND-CCA2 \ac{PKE} scheme and a semantically secure symmetric 
%encryption scheme in the \ac{ANOBE} construction.
%The reason for this is that the symmetric operations are faster than the 
%asymmetric ones.
%This is an important factor for \acp{DOSN} during e.g.~news-feed aggregation 
%when a user comes online.
%In these situations we have to handle large amounts of data, which can lead to 
%performance problems.
%
%% XXX Write about MAC for group deniability
%If we replace the signatures by \acp{MAC}, then we can achieve group 
%deniability: anyone in the group could have posted the message.





\section{Related Work}\label{RelatedWork}

% XXX Formalize hidden policies, credentials and decisions --- but how?
%  - Hidden policies: policy-hiding ciphertext, how to formalize?
%    - Given two equally-sized ciphertexts the adversary cannot tell them 
%      apart?
%    - What about the size of the recipient set?  Padding and dummy entries 
%      should work?
%    - Probably the definition in [ANOBE] can help?
%    - Timing-attacks on the push model?  Alice writes n same-sized 
%      objects to n places within a short timespan.  How to formalize?
%  - Hidden credentials: the keys are by definition hidden, how to 
%    formalize?  In identity-based access control, is the identity itself 
%    the credential or whatever authenticates the identity?  Or both?
%  - Hidden decisions: the adversary should not be able to guess the 
%    outcome, an IND-style property.  This shouldn't be any problem to 
%    turn into a security game.

\citet{TowardsPPACwHPHCHD} identified three desirable and (conjectured 
sufficient) properties for a privacy-preserving \ac{AC} mechanism:
\begin{inparablank}
\item hidden policies,
\item hidden credentials, and
\item hidden decisions.
\end{inparablank}
The work in~\cite{TowardsPPACwHPHCHD} focused on \ac{FHE} and is thus not 
directly feasible for our purposes.
However, the properties are still relevant in our setting.

Hidden policies means that the access policy remains hidden from anyone but 
the owner and the subjects learn at most if they have access or not.
So the subjects cannot learn which other subjects can access the same object.
We arguably achieve this property in our constructions.
Furthermore, we also achieve something closer to \enquote{hidden 
  policy-changes} as well (Jealous Bob), which prevents previously-authorized 
subjects from determining whether they are no longer authorized to receive 
updates.

%Let us briefly illustrate why the hidden policies property is important.
%Bob is not allowed access to an object, but he can see that Alice is.
%Then Bob can go to Alice and either ask her about it, or otherwise force her to 
%reveal it to him --- e.g.\ by stealing her keys.
%Even if Bob was allowed access to the object it is important that he does not 
%learn the policy.
%If Bob knows that Alice also has access to the object, then he can reveal the 
%object's contents to Eve and tell Eve to blame Alice for leaking the data to 
%her.
%But if Bob did not know who else has access, then he might be the only one and 
%thus less prone to leak to Eve as he is the only one to blame.

Hidden credentials means that the subject never has to reveal the access 
credentials to anyone.
In our case this is a cryptographic key, and as a consequence we allow the 
subject to anonymously read the ciphertext from the storage node.
This means that the storage node cannot track which subjects are requesting 
access to which objects.
(If users are not anonymized, then the storage node can approximate the 
credential, i.e.\ the subject's identity.)

Hidden decisions means that no-one but the subject must learn the outcome of an 
access request.
This means that no-one should learn whether or not a subject could decrypt the 
ciphertext or not.
However, if everyone only requests ciphertexts that they know they can decrypt 
(which is the most efficient strategy), then the storage operator can easily 
guess the decision.
Most solutions probably suffer from this, including the constructions of this 
paper.
This decision together with non-anonymized users would allow the storage 
operator to infer parts of the policy, hence breaking the hidden policy 
property.
For this reason, in addition to anonymization, subjects must also request 
ciphertexts they cannot decrypt.
This can be done either by dummy requests (i.e.\ requesting objects at random) 
or by \ac{PIR} techniques, such as \ac{OT}.

We mentioned earlier the relation between the Pull Model and \ac{BE}.
The purpose of \ac{BE} is to develop methods to efficiently transmit data to 
dynamically changing target audiences \(S\subseteq U\) who are allowed to read 
the data, whereas the remaining users \(U\setminus S\) are not.
Due to the modularity of our constructions any \ac{BE} scheme fulfilling the 
properties could be plugged in and used instead of the \ac{ANOBE} scheme used 
above.
At the moment though, \ac{ANOBE} is the only known scheme that fulfils all 
requirements.
Also worthy of note is that none of the works in the \ac{BE} area has treated 
the problem of hidden policy-changes.
In fact, as is pointed out by \citet{ANOBE}, most research in \ac{BE} has been 
focused on efficiency and not privacy (beyond confidentiality).

There is also related work in the \ac{DOSN} community.
There are several proposals available for \acp{DOSN}, e.g.\ 
DECENT~\cite{DECENT}, Cachet~\cite{Cachet} and Persona~\cite{Persona}.
The \ac{AC} mechanisms in these proposals focus on providing confidentiality 
for the data.
E.g.\ Persona uses \ac{KP-ABE} to implement the \ac{AC} mechanism and
unfortunately, this yields lacking privacy: as this is not policy-hiding, 
anyone can read the \ac{AC} policies and see who may access what data.
There are also general cryptographic \ac{AC} schemes that focus on achieving 
policy-hiding ciphertexts, see the section of related work 
in~\cite{TowardsPPACwHPHCHD}.
E.g.\ \citet{PEAC} adapted \ac{PE} for the \ac{AC} mechanism in \acp{DOSN}.
Works in this area that have employed policy-hiding schemes for \acp{DOSN} have
also focused on solving the problem of re-encryption of old data upon group 
changes.
We do not solve this problem, but rather contribute the insight that it would 
violate our desired privacy properties.
So besides being more efficient in some cases, less efficient in others, they 
do not solve exactly the same problem.



\section{Conclusions}\label{Conclusions}

We wanted to achieve privacy-preserving access control enforcement in a public 
file system lacking built-in access control.
We presented two alternatives: the Pull Model and the Push Model.
Both implements the model of a publisher distributing a message to a set of 
subscribers.

The Pull Model achieves strong privacy properties.
The subscribers cannot learn who the other subscribers are even if they control 
the entire network or the whole storage system.
Further, if the publisher wishes to exclude any of the subscribers from future 
publications, then all subscribers learn that there was a policy update but no 
one learns how --- not even those excluded!

The Push Model is an interesting case.
Conceptually, the only difference between the Push and Pull models is that we 
distribute the message, instead of everyone fetching it.
This seems to yield better privacy properties at first glance, but it turns out
that we had to considerably weaken the adversary's capabilities before we could
get any security.
Hence the security guarantees are much weaker: for the Push Model the adversary 
can only control a small part of the storage system, or monitor only part of 
the network.
However, unlike in the Pull Model, when the subscriber wants to make a policy 
update, none of the subscribers will even be notified that there has been 
a policy update --- let alone determine if they have been excluded.

As was pointed out in \cref{RelatedWork}, we do not treat group management 
(i.e.\ revoking the credentials for subjects) as is done in other schemes.
If the publisher excludes a subscriber, it is only from future publications ---
not from past!
In fact, we can conclude from our treatment above that such functionality would 
actually violate the privacy properties.
Even if possible, any subject could have made a copy anyway, i.e.\ it is the 
problem of removing something from the Web.
However, other group changes are easily done, the most expensive one is to give
a new subscriber access to old publications --- this requires re-encrypting or 
resending all publications.

We can thus conclude that it is important that any optimizations to the 
constructions can be proven not to break any of the properties.
An interesting future direction would be to explore Eve's limitations in the 
Push Model a bit more.
For example, if Eve is not allowed to control all the \(k\) parallel games, 
i.e.\ that honest players play the other \(k-1\) games, would she still be able
to win?
At a first glance she would, but better understanding under what conditions 
might give us a better solution than to weaken her capabilities.

Other interesting directions are stronger deniability properties on the one 
hand, and stronger accountability on the other.
An example of improved deniability could be using a \ac{MAC} scheme for 
authentication of messages, then we no longer have the non-repudiation 
property.
Any subscriber could publish to all subscribers in the Pull Model.
An example of desired accountability would be that Bob wants to verify that 
Alice received the same message, if Eve told him that she sent a copy to Alice 
as well.


\subsubsection{Acknowledgements}

This work was inspired by work with Benjamin Greschbach and work of Oleksandr 
Bodriagov and Gunnar Kreitz.
We would like to thank the Swedish Foundation for Strategic Research for grant 
SSF FFL09-0086 and the Swedish Research Council for grant VR 2009-3793, which 
funded this work.
We would also like to thank the anonymous reviewers and especially Anja Lehmann 
for valuable feedback.


\printbibliography{}


\appendix
\subsection{\Acl{ANOBE}}\label{ANOBE}

The main idea of \ac{ANOBE} is to distribute a key \(k\) (i.e.\ 
key-encapsulation) to a subset of users, such that they can decrypt the 
broadcast message encrypted with \(k\) but no one else can.
The users who receive \(k\) should not be able to figure out who else received 
\(k\) and who did not.
Although it is a key-encapsulation scheme, we will use the term message to not 
confuse all the keys.

We will now describe how \ac{ANOBE} works.
\(\BE[Setup]\) generates a master public key \(MPK = (\S*, \{\PubKey{i}\}_{i\in
    U})\) and the master secret key \(MSK = \{\PriKey{i}\}_{i\in U}\), where 
\((\PubKey{i}, \PriKey{i})\rgets \E*[Keygen][1^\lambda]\).
An overview of the encryption function is given in \cref{EncANOBE} and an 
overview of the decryption function in \cref{DecANOBE}.
\(\BE[Keygen][MPK, MSK, i]\) simply returns \(\PriKey{i}\) from \(MSK\).

\subsubsection{Encryption}

We must first generate a one-time signature key-pair \((\SignKey{}, 
  \VerifKey{})\), then we choose a random permutation \(\pi\colon R\to R\).
Next we must encrypt the message \(m\) and the verification key \((m, 
  \VerifKey{})\) for every user \(i\in R\) in the recipient set \(R\subseteq 
  U\) under their respective public key, \(c_{i} = \Enc[\PubKey{i}]{m, 
    \VerifKey{}}\).
We let the \ac{ANOBE} ciphertext be the tuple \((\VerifKey{}, C, \sigma)\), 
where
\(C = ( c_{\pi(1)}, \ldots, c_{\pi({|S|})})\) and
\(\sigma = \Sign[\SignKey{}]{ C }\).
Note that the signatures does not authenticate the sender, it is used to verify 
correct or incorrect decryption.
We will get back to the details about this shortly.

\begin{frame}
  \begin{figure}
    \framebox{\begin{minipage}{0.96\textwidth}
    \begin{algorithmic}
      \Function{$\ANOBE[Enc]$}{$MPK, m, R$}
      \Comment{%
        Recipient set $R$,
        $m$ to be encrypted.
      }
        \State{%
          $(\SignKey{}, \VerifKey{})\rgets{\S*[Keygen][1^\lambda]}$
        }
        \Comment{%
          Signature-verification key-pair, security parameter $\lambda$
        }
        \State{%
          Choose a random permutation \(\pi\colon R\to R\).
        }

        \For{$i \in R$}
          \State{%
            $c_{i}\gets{ \E*[Enc][\PubKey{i}, m\concat \VerifKey{}] }$
          }
        \EndFor{}

        \State{%
          $C\gets{( c_{\pi(1)}, \ldots, c_{\pi({|S|})} )}$
        }
        \State{%
          $\sigma\gets{ \S*[Sign][\SignKey{}, C] }$
        }
        \State{%
          \Return{$(\VerifKey{}, C, \sigma)$}
        }
      \EndFunction{}
    \end{algorithmic}
    \end{minipage}}
    \caption{%
      An algorithmic overview of the encryption algorithm in the \ac{ANOBE} 
      scheme.
    }\label{EncANOBE}
  \end{figure}
\end{frame}

\subsubsection{Decryption}

We now have data which we parse as the tuple \((\VerifKey{}, C, \sigma)\).
If \(\S*[Verify][\VerifKey{}, C, \sigma ] = 0\), then we return \(\bot\) as the 
verification failed.
For each \(c\) in \(C\):
Compute \(M = \E*[Dec][\PriKey{}, c ]\).
If \(M \neq \bot\) and \(M = (m, \VerifKey{})\), then return \(m\).
Otherwise, try the next \(c\).
If there are no more \(c\) to try, then return \(\bot\).

\begin{frame}
  \begin{figure}
    \framebox{\begin{minipage}{0.96\textwidth}
    \begin{algorithmic}
      \Function{$\ANOBE[Dec]$}{$MPK, \PriKey{}, C_{\ANOBE}$}
      \Comment{%
        Private key \(\PriKey{}\),
        ciphertext $C_{\ANOBE} = (\VerifKey{}, C, \sigma)$.
      }

        \If{$\S*[Verify][\VerifKey{}, C, \sigma ] = 0$}
          \State{%
            \Return{$\bot$}
          }
        \EndIf{}

        \For{$c\in C$}
          \State{%
            $M\gets{\E*[Dec][\PriKey{}, c]}$
          }
          \Comment{%
            Try to decrypt
          }
          \If{$M = \bot$}
            \State{\Return{$\bot$}}
          \ElsIf{$M = (m, \VerifKey{})$}
            \State{\Return{m}}
          \EndIf{}
        \EndFor{}
        \State{\Return{$\bot$}}
      \EndFunction{}
    \end{algorithmic}
    \end{minipage}}
    \caption{%
      An algorithmic overview of the decryption algorithm in the \ac{ANOBE} 
      scheme.
    }\label{DecANOBE}
  \end{figure}
\end{frame}

To decrypt an \ac{ANOBE} ciphertext, we need a trial-and-error decryption 
procedure to decide if the ciphertext was indeed intended for us.
This is costly as it makes the decryption function complexity \(O(|S|)\).
\citet{ANOBE} presented a tag-hint system along with their \ac{ANOBE} scheme.
The tag-hint system reduced the complexity back to \(O(1)\).
As this is not relevant for our discussion, we refer the reader 
to~\cite{ANOBE}.


