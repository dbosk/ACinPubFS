\usepackage[utf8]{inputenc}
\usepackage[T1]{fontenc}
\usepackage{fixltx2e}
\usepackage[british]{babel}
\usepackage[defblank]{paralist}
\usepackage[fulladjust]{marginnote}
\usepackage[underline=false]{pgf-umlsd}
\usepackage{tikz}
\usepackage{algorithm}
\usepackage[noend]{algpseudocode}
\usepackage{graphicx}
\usepackage{booktabs}
\usepackage{subfig}
\usepackage{mleftright}

\usepackage{xparse}
\NewDocumentCommand{\head}{O{45} O{1em} m}{%
  \makebox[#2][l]{\rotatebox{#1}{#3}}
}

\usepackage{amssymb}
\usepackage{amsmath}
% Common algorithms
\DeclareMathOperator{\Aop}{\mathcal{A}}
\NewDocumentCommand{\A}{s o}{\ensuremath{%
    \Aop%
    \IfBooleanT{#1}{%
      ^*%
    }
    \IfValueT{#2}{%
      \mleft( #1 \mright)%
    }
  }}
\DeclareMathOperator{\Dop}{\mathcal{D}}
\NewDocumentCommand{\D}{s o}{\ensuremath{%
    \Dop%
    \IfBooleanT{#1}{%
      ^*%
    }
    \IfValue{#2}{%
      \mleft( #2 \mright)%
    }
  }}
%\DeclareMathOperator{\Fop}{\mathcal{F}}
%\NewDocumentCommand{\F}{o}{\ensuremath{%
%    \Fop%
%    \IfValueT{#1}{%
%      \mleft( #1 \mright)%
%    }
%  }}
\DeclareMathOperator{\M}{\mathcal{M}}

\NewDocumentCommand{\E}{s}{\ensuremath{%
    \IfBooleanTF{#1}{%
      \Scheme*{E}%
    }{%
      \Scheme{E}%
    }%
  }}
\let\parsign\S
\DeclareDocumentCommand{\S}{s}{\ensuremath{%
    \IfBooleanTF{#1}{%
      \Scheme*{S}%
    }{%
      \Scheme{S}%
    }%
  }}
\NewDocumentCommand{\BE}{s}{\ensuremath{%
    \IfBooleanTF{#1}{%
      \Scheme*{BE}%
    }{%
      \Scheme{BE}%
    }%
  }}
\NewDocumentCommand{\ANOBE}{s}{\ensuremath{%
    \IfBooleanTF{#1}{%
      \Scheme*{ANOBE}%
    }{%
      \Scheme{ANOBE}%
    }%
  }}


\DeclareMathOperator{\Sop}{\mathcal{S}}
%\NewDocumentCommand{\S}{s}{\ensuremath{%
%    \IfBooleanTF{#1}{\overline{\Sop}}{\Sop}%
%  }}

\DeclareMathOperator{\SplitGroupOp}{\mathsf{SplitGroup}}
\NewDocumentCommand{\SplitGroup}{o}{\ensuremath{%
    \SplitGroupOp%
    \IfValueT{#1}{%
      \mleft( #1 \mright)%
    }
  }}

% File system
\DeclareMathOperator{\FSop}{\mathcal{FS}}
\NewDocumentCommand{\FS}{o}{\ensuremath{%
    \FSop%
    \IfValueT{#1}{%
      \!.\mathsf{#1}%
    }
  }}
\NewDocumentCommand{\fsread}{o}{\ensuremath{%
    \FS[read]%
    \IfValueT{#1}{%
      \mleft( #1 \mright)%
    }
  }}
\NewDocumentCommand{\fswrite}{o o}{\ensuremath{%
    \FS[append]%
    \IfValueT{#1}{%
      \mleft( #1, \IfValueTF{#2}{#2}{\cdot} \mright)%
    }
  }}

% Communication model
\DeclareMathOperator{\Cop}{\mathcal{C}}
\NewDocumentCommand{\C}{o o o}{\ensuremath{%
    \Cop%
    \IfValueT{#1}{%
      _{#1}%
    }%
    \IfValueT{#2}{%
      \mleft[ #2 \mright]%
    }%
    \IfValueT{#3}{%
      \relax.\mathsf{#3}%
    }
  }}
\NewDocumentCommand{\csetup}{o m o}{\ensuremath{%
    \C[#1][#2][setup]%
    \IfValueT{#3}{%
      \mleft( #3 \mright)%
    }
  }}
\NewDocumentCommand{\cpub}{o m o o}{\ensuremath{%
    \C[#1][#2][publish]%
    \IfValueT{#3}{%
      \mleft( #3, \IfValueTF{#4}{#4}{\cdot} \mright)%
    }
  }}
\NewDocumentCommand{\cget}{o m o}{\ensuremath{%
    \C[#1][#2][get]%
    \IfValueT{#3}{%
      \mleft( #3 \mright)%
    }
  }}

% Pull Model
\NewDocumentCommand{\PullOp}{}{\Cop^{\mathsf{Pull}}}
\NewDocumentCommand{\Pull}{o o o}{\ensuremath{%
    \PullOp%
    \IfValueT{#1}{_{#1}}%
    \IfValueT{#2}{%
      \mleft[ #2 \mright]%
    }%
    \IfValueT{#3}{%
      \relax.\mathsf{#3}%
    }
  }}
\NewDocumentCommand{\psetup}{o m o}{\ensuremath{%
    \IfValueTF{#1}{%
      \Pull[#1][#2][setup]%
    }{%
      \Pull[][#2][setup]%
    }
    \IfValueT{#3}{%
      \mleft( #3 \mright)%
    }
  }}
\NewDocumentCommand{\ppub}{o m o o}{\ensuremath{%
    \IfValueTF{#1}{%
      \Pull[#1][#2][publish]%
    }{%
      \Pull[][#2][publish]%
    }
    \IfValueT{#3}{%
      \mleft( #3, \IfValueTF{#4}{#4}{\cdot} \mright)%
    }
  }}
\NewDocumentCommand{\pget}{o m o}{\ensuremath{%
    \IfValueTF{#1}{%
      \Pull[#1][#2][fetch]%
    }{%
      \Pull[][#2][fetch]%
    }
    \IfValueT{#3}{\mleft( #3 \mright)}
  }}
\NewDocumentCommand{\psplit}{o m o}{\ensuremath{%
    \IfValueTF{#1}{%
      \Pull[#1][#2][split]%
    }{%
      \Pull[][#2][split]%
    }
    \IfValueT{#3}{\mleft( #3 \mright)}
  }}

% Push Model
\NewDocumentCommand{\PushOp}{}{\Cop^{\mathsf{Push}}}
\NewDocumentCommand{\Push}{o o o}{\ensuremath{%
    \PushOp%
    \IfValueT{#1}{_{#1}}%
    \IfValueT{#2}{%
      \mleft[ #2 \mright]%
    }%
    \IfValueT{#3}{%
      \relax.\mathsf{#3}%
    }
  }}
\NewDocumentCommand{\pnsetup}{o m o}{\ensuremath{%
    \IfValueTF{#1}{%
      \Push[#1][#2][setup]%
    }{%
      \Push[][#2][setup]%
    }
    \IfValueT{#3}{%
      \mleft( #3 \mright)%
    }
  }}
\NewDocumentCommand{\pnpub}{o m o o}{\ensuremath{%
    \IfValueTF{#1}{%
      \Push[#1][#2][publish]%
    }{%
      \Push[][#2][publish]%
    }
    \IfValueT{#3}{%
      \mleft( #3, \IfValueTF{#4}{#4}{\cdot} \mright)%
    }
  }}
\NewDocumentCommand{\pnget}{o m o}{\ensuremath{%
    \IfValueTF{#1}{%
      \Push[#1][#2][fetch]%
    }{%
      \Push[][#2][fetch]%
    }
    \IfValueT{#3}{\mleft( #3 \mright)}
  }}
\NewDocumentCommand{\pnsplit}{o m o}{\ensuremath{%
    \IfValueTF{#1}{%
      \Push[#1][#2][split]%
    }{%
      \Push[][#2][split]%
    }
    \IfValueT{#3}{\mleft( #3 \mright)}
  }}


\usepackage[strict]{csquotes}
%\MakeBlockQuote{|}{§}{|}
%\EnableQuotes{}

\usepackage[lncs,crypto,stdmaths,algebra,ac,surveillance,std]{libbib}

\usepackage[natbib,style=lncs,backend=biber]{biblatex}
\addbibresource{anon.bib}
\addbibresource{crypto.bib}
\addbibresource{meta.bib}
\addbibresource{osn.bib}
\addbibresource{surveillance.bib}
\addbibresource{ac.bib}
\addbibresource{be.bib}
\addbibresource{rfc.bib}

\usepackage[binary,amssymb]{SIunits}
\usepackage{cleveref}
