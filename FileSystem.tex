% XXX Formalize the setting, the system model:
%  - A file system: read, write, append.
%  - Adversary is storage node, operates file system.
%  - Adversary participates as user.

\subsection{A General File System}\label{FileSystem}

We will model our system as an abstract file system with the operations create, 
read and append.
The operations are defined as we intuitively expect: we can create an object, 
then read it and also append to it.
The file system itself provides no access control.

\begin{definition}[Public file system]\label{FS}
  A \emph{file} \(f = (i, m)\) consists of the \emph{identifier} \(i\) and the 
  \emph{content} \(m\).
  We define the set of files \(F = \{(i, m_i)\}\) together with the following 
  operations to be the \emph{public file system} \(\FS\):
  \begin{itemize}
    \item \(\fswrite[i][m_i]\) will set \(F\gets F\cup \{(i, m_i)\}\) if 
      \(\{(i^\prime, m^\prime)\in F \mid i^\prime = i\} = \emptyset\), i.e.\ 
      the file \(i\) does not yet exist.
      Otherwise this operation will set \((i, m)\gets (i, m\concat m^\prime)\) 
      for \((i, m)\in F\).

    \item \(\fsread[i] = m_i\) if \((i, m_i)\in F\), otherwise 
      \(\fsread[i^\prime] = \bot\).

  \end{itemize}
\end{definition}

We can see in the definition that anyone can read any object in the file 
system.
Anyone can also create new files and append to existing files.

% XXX Formalize assumptions:
%  - Storage node cannot tell the origin of two requests apart.
%  - Eve controls the entire storage system?
%    - Here she can do an exhaustive search.
%  - Eve only controls part of the network?
%    - Can Eve do an exhaustive search in the storage?
%
%Assuming that Eve controls the entire storage system should be more 
%interesting as it's a stronger adversary.  There shouldn't be any need 
%for the weaker adversary who controls only a part of the storage system.  
%We'll have to remove some randomness re-use from ANOBE for the 
%implementation, but that should suffice.

We will let the adversary Eve operate the file system \(\FS\) defined in 
\cref{FS}.
By this we mean that Eve has access to the internal state of \(\FS\) directly, 
i.e.\ the set of files \(F\).
She can thus read all files (same as everyone else), but she can also do 
arbitrary searches since she does not have to use \(\fsread{\cdot}\).
She can also modify the files (including deleting them).

To be consistent with \cref{FS}, Eve cannot distinguish between Alice's and 
Bob's requests to the operations of \(\FS\) --- in the definition there is 
nothing to identify who used any of the operations.
But she can record the times at which they occurred with her own clock, since 
she executes them.


