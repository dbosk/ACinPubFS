% XXX Formalize what we want to do:
%  - Publisher makes a message m asynchronously available to n recipients 
%    while hiding the recipient set:
%    - Pull model: publish m as object o.
%    - Push model: publish m as objects o_1, ..., o_n.
%  - Each recipient reads m:
%    - Pull model: read object o.
%    - Push model: read object o_i.
%  - The publisher should be unidentifiable for a non-recipient.
%
% XXX Top-down approach:
%  - First, analyse high-level properties in pull and push models.
%    - Motivation for hidden policies, credentials, decisions.
%    - Alice removes Bob from recipient set: Bob shouldn't distinguish this or 
%      if Alice stopped posting to all.
%    - Bob shouldn't be able to monitor Alice's activities unrelated to Bob.
%    - Eve knows Alice's and Bob's inboxes, for any message m in Alice's, she 
%      shouldn't be able to tell if it's also in Bob's.
%  - Second, suggest an example of possible implementation using (modified) 
%    ANOBE.
%  - Third, complexity for solution.
%
\section{The Ideal Communication Model}\label{IdealCommunication}

There are several ways Alice can implement the message passing with her 
friends.
We will start by presenting an ideal model of communication 
(\cref{CommunicationModel}) whose properties Alice wants to achieve.
Then we will proceed to the details of two alternative protocols 
(\cref{PullModel,PushModel}) that yields the properties of the ideal model.

% XXX Shall we have an adversary interface for C?
\begin{definition}[Communication model]\label{CommunicationModel}
  Let \(\C[p, S]\) be a \emph{\((p, S)\)-communication model} with 
  a \emph{publisher} \(p\) and a \emph{set of subscribers} \(S\).
  Then we have the following operations defined on \(\C[p, S]\):
  \begin{itemize}
    \item The publisher first runs \(\csetup[p, S]{p}[1^\lambda]\) and each 
      subscriber \(s\in S\) runs \(\csetup[p, S]{s}[1^\lambda]\), where 
      \(\lambda\) is the security parameter.
    \item The publisher \(p\) uses \(\cpub[p, S]{p}[R][m]\) to publish the 
      message \(m\) to the designated \emph{recipient set} \(R\) by making it 
      available to all subscribers \(s\in R\subseteq S\).
    \item Each subscriber \(s\in S\) uses \(\cget[p, S]{s}[]\) to get the set 
      \(M\) of published messages \(m\in M\) for which \(s\) was in the 
      recipient set.
  \end{itemize}
\end{definition}

Whenever \(p, S\) are clear from the context, we will simply omit them, e.g.\ 
\(\csetup{p}\) and \(\csetup{s}\).

There are two ways the adversary can gain information.
The first is from corrupted subscribers and the second is from the public file 
system over which the communication model is implemented.
Hence the second is not visible in \cref{CommunicationModel}, but will be in 
\cref{PullModel,PushModel}.

In our desired scenario, Alice acts as a publisher and her friends as the 
subscribers.
We want the following properties:
\begin{description}
  \item[Message privacy]
    No subscriber \(s^\prime\in S\) can use \(\cget{s}\) for \(s^\prime\neq s\).
    I.e.\ no subscriber can read the messages of any other subscriber, we call 
    this property \emph{message privacy}.

  \item[Hidden policies]
    No subscriber \(s_i\) must learn which \(s_1, \ldots, s_n\in R\) beyond 
    that \(i\in \{1, \ldots, n\}\) and \(s_i\in R\) for a given message \(m\).
    I.e.\ no subscriber must know who else received a message, we call this 
    property \emph{hidden policy}.
    Thus if Alice publishes a message, none of her friends know who else 
    received it and none can read the others' received messages (possibly to 
    check if they have received the same message).

% XXX Formalize other required (more high-level) properties:
%  - Alice removes Bob as a friend: Bob shouldn't distinguish this or if 
%    Alice stopped posting to all.
%  - Bob shouldn't be able to monitor Alice's activities.
%  - Eve knows Alice's and Bob's inboxes, for any message m in Alice's, 
%    she shouldn't be able to tell if it's also in Bob's.

  \item[Hidden policy-updates]
    In addition to the hidden-policy property, we want something we call 
    \emph{hidden policy-updates}.
    Consider the following example:
    Bob might become jealous if he realizes that Alice no longer includes him 
    in the recipient set for her messages.
    As such Bob should only be able to determine that Alice publishes content 
    by being in the recipient set himself.
\end{description}

There are several ways to implement the message-passing protocol for the 
communication model in \cref{CommunicationModel}.
We will focus on two alternative protocols, one using the pull model and the 
other using the push model for communication.
The push model is analogous to the subscription of magazines:
a subscriber contacts the publisher and signs a subscription, whenever the 
publisher issues a new magazine it sends a copy to the subscriber's mailbox.
The pull model is the converse of the push model.
It is analogous to the selling of magazines in kiosks:
the publisher issues magazines and the \enquote{subscribers} come to the kiosk 
and buy them whenever they want.
We can see that our ideal communication model allows both a pull and push 
strategy for implementation.
We will describe and analyse the pull construction in \cref{PullAnalysis} and 
the push construction in \cref{PushAnalysis}.
First we need to review our needed building blocks.
