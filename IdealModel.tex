% XXX Formalize what we want to do:
%  - Publisher makes a message m asynchronously available to n recipients 
%    while hiding the recipient set:
%    - Pull model: publish m as object o.
%    - Push model: publish m as objects o_1, ..., o_n.
%  - Each recipient reads m:
%    - Pull model: read object o.
%    - Push model: read object o_i.
%  - The publisher should be unidentifiable for a non-recipient.
%
% XXX Top-down approach:
%  - First, analyse high-level properties in pull and push models.
%    - Motivation for hidden policies, credentials, decisions.
%    - Alice removes Bob from recipient set: Bob shouldn't distinguish this or 
%      if Alice stopped posting to all.
%    - Bob shouldn't be able to monitor Alice's activities unrelated to Bob.
%    - Eve knows Alice's and Bob's inboxes, for any message m in Alice's, she 
%      shouldn't be able to tell if it's also in Bob's.
%  - Second, suggest an example of possible implementation using (modified) 
%    ANOBE.
%  - Third, complexity for solution.
%
\section{The Ideal Communication Model}\label{IdealCommunication}

There are several ways Alice can implement the message passing with her 
friends.
We will start by presenting an ideal model of communication 
(\cref{CommunicationModel}) whose properties Alice wants to achieve.
Then we will proceed to the details of two alternative protocols 
(\cref{PullModel,PushModel}) that yields the properties of the ideal model.

% XXX Shall we have an adversary interface for C?
\begin{definition}[Communication model]\label{CommunicationModel}
  Let \(\C[p, S]\) be a \emph{\((p, S)\)-communication model} with 
  a \emph{publisher} \(p\) and a \emph{set of subscribers} \(S\).
  Then we have the following operations defined on \(\C[p, S]\):
  \begin{itemize}
    \item Before start the publisher runs \(\csetup[p, S]{p}[1^\lambda]\) and 
      each subscriber \(s\in S\) runs \(\csetup[p, S]{s}[1^\lambda]\), where 
      \(\lambda\) is the security parameter.
    \item The publisher \(p\) uses \(\cpub[p, S]{p}[R][m]\) to publish the 
      message \(m\) to the designated \emph{recipient set} \(R\) by making it 
      available to all subscribers \(s\in R\subseteq S\).
    \item Each subscriber \(s\in S\) uses \(\cget[p, S]{s}[]\) to get the set 
      \(M\) of published messages \(m\in M\) for which \(s\) was in the 
      recipient set.
  \end{itemize}
\end{definition}

Whenever \(p, S\) are clear from the context, we will simply omit them, e.g.\ 
\(\csetup{p}\) and \(\csetup{s}\).

There are two ways the adversary can gain information.
The first is from corrupted subscribers and the second is from the public file 
system over which the communication model is implemented.
Hence the second is not visible in \cref{CommunicationModel}, but will be in 
\cref{PullModel,PushModel}.

In our desired scenario, Alice acts as a publisher and her friends as the 
subscribers.
We note that no subscriber \(s\in S\) can use \(\cpub{s}[R][m]\) to publish 
a message \(m\).
The communication model is strictly unidirectional from publisher to 
subscribers and they cannot access the publisher's interface.
In the same sense, no subscriber \(s^\prime\in S\) can use \(\cget{s}\) for 
\(s^\prime\neq s\).
It follows too that no subscriber \(s_i\) can learn which \(s_1, \ldots, s_n\in 
  R\) beyond that \(i\in \{1, \ldots, n\}\) and \(s_i\in R\) for a given 
message \(m\).
This is captured formally in the following security game, which is adapted from 
Def.\ 2 of~\cite{ANOBE} (ANO-IND-CCA).

\begin{definition}[ANO-IND-CCA]\label{ANO-IND-CCA}
  We define the ANO-IND-CCA game as follows:
  \begin{description}
    \item[Setup] The challenger runs \(\csetup{s}[1^\lambda]\) for all \(s\in 
        S\) and \(\csetup{p}[1^\lambda]\).
      The adversary is given oracle access to the \(\cpub{p}\) interface.

    \item[Phase 1] The adversary \(\A\) may corrupt any \(s\in S\).
      This means that the adversary gains the entire state: private keys 
      generated during the set-up phase etc.
      Additionally \(\A\) has access to a decryption oracle to decrypt 
      arbitrary ciphertexts for any \(s\in S\).

    \item[Challenge] The adversary chooses messages \(m_0, m_1\) such that 
      \(|m_0| = |m_1|\) and recipient sets \(R_0, R_1\) such that \(|R_0| 
        = |R_1|\) and for all corrupted \(s\in S\) we have \(s\notin S_0\cup 
        S_1\setminus (S_0\cap S_1)\).
      If \(s\in S_0\cap S_1\) we require \(m_0 = m_1\).
      Then \(\A\) gives them to the challenger.
      The challenger randomly chooses \(b\in \{0, 1\}\) and executes 
      \(\cpub{p}[R_b][m_b]\).

    \item[Phase 2] The adversary may continue to corrupt \(s\notin S_0\cup 
        S_1\setminus (S_0\cap S_1)\).
      \(\A\) may corrupt \(s\in S_0\cap S_1\) only if \(m_0 = m_1\).
      The adversary is not allowed to use the decryption oracle on the 
      challenge ciphertext.

    \item[Guess] \(\A\) outputs a bit \(\hat{b}\) and wins the game if 
      \(\hat{b} = b\).

  \end{description}
  We define the adversary's advantage \(\Adv{\text{ANO-IND-CCA}}{\A, 
      \Cop}[1^\lambda] = |\Pr[ \hat{b} = b ] - \frac{1}{2}|\).
\end{definition}

We also provide the following, slightly weaker, version.
The difference is that the adversary no longer has access to the decryption 
oracle.

\begin{definition}[ANO-IND-CPA]\label{ANO-IND-CPA}
  We define the ANO-IND-CPA game as follows:
  \begin{description}
    \item[Setup] The challenger runs \(\csetup{s}[1^\lambda]\) for all \(s\in 
        S\) and \(\csetup{p}[1^\lambda]\).
      The adversary is given oracle access to the \(\cpub{p}\) interface.

    \item[Phase 1] The adversary \(\A\) may do publications and observe the 
      outcomes and perform arbitrary computations on the resulting ciphertexts.

    \item[Challenge] The adversary chooses messages \(m_0, m_1\) such that 
      \(|m_0| = |m_1|\) and recipient sets \(R_0, R_1\) such that \(|R_0| 
        = |R_1|\) and for all corrupted \(s\in S\) we have \(s\notin S_0\cup 
        S_1\setminus (S_0\cap S_1)\).
      If \(s\in S_0\cap S_1\) we require \(m_0 = m_1\).
      Then \(\A\) gives them to the challenger.
      The challenger randomly chooses \(b\in \{0, 1\}\) and executes 
      \(\cpub{p}[R_b][m_b]\).

    \item[Phase 2] The adversary may continue to do publications and arbitrary 
      computations on the ciphertexts.

    \item[Guess] \(\A\) outputs a bit \(\hat{b}\) and wins the game if 
      \(\hat{b} = b\).

  \end{description}
  We define the adversary's advantage \(\Adv{\text{ANO-IND-CPA}}{\A, 
      \Cop}[1^\lambda] = |\Pr[ \hat{b} = b ] - \frac{1}{2}|\).
\end{definition}

% XXX Formalize other required (more high-level) properties:
%  - Alice removes Bob as a friend: Bob shouldn't distinguish this or if 
%    Alice stopped posting to all.
%  - Bob shouldn't be able to monitor Alice's activities.
%  - Eve knows Alice's and Bob's inboxes, for any message m in Alice's, 
%    she shouldn't be able to tell if it's also in Bob's.

However, there are some desirable privacy properties that are not captured by 
these games.
Bob might become jealous if he realizes that Alice no longer includes him in 
the recipient set for her messages.
As such Bob should only be able to determine that Alice publishes content by 
being in the recipient set himself.
We treat this problem in later sections, until then we will refer to this as 
the problem of \enquote{Jealous Bob}.

There are several ways to implement the message-passing protocol for the 
communication model in \cref{CommunicationModel}.
We will focus on two alternative protocols which uses \(\FS\) from \cref{FS} 
above, these are the Push Model and the Pull Model.
The Push Model is analogous to the subscription of magazines:
a subscriber contacts the publisher and signs a subscription, whenever the 
publisher issues a new magazine it sends a copy to the subscribers mailbox.
The Pull Model the converse of the Push Model.
It is analogous to the selling of magazines in kiosks:
the publisher issues magazines and the \enquote{subscribers} come to the kiosk 
and buy them whenever they want.
We note that essentially all \ac{BE} schemes, including \ac{ANOBE}, are 
designed for the Pull Model.
It is therefore natural to start our analysis with the Pull Model.


