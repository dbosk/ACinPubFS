\section{Construction and Analysis of the Pull Model}
\label{PullAnalysis}

In this model, each publisher has an \enquote{outbox} file in the file system.
This is simply a file object with a randomly chosen identifier.
The publisher adds new publications to the outbox while subscribers pull new 
content from the outbox.
In our analysis we will use the following definition for the Pull Model.

\begin{definition}[Pull Model]\label{PullModel}
  Let \(\mathcal{E} = (\KeygenOp, \EncOp, \DecOp)\) be an authenticated 
  broadcast-encryption scheme and \(\FS\) be a public file system.
  We denote by \(\Pull[p, S]\) the \emph{pull model protocol} implementing 
  a \((p, S)\)-communication model through the operations
  \begin{itemize}
    \item \(\psetup[p, S]{p}[\cdot]\) and \(\psetup[p, S]{s}[\cdot]\) for each
      \(s\in S\),
    \item \(\ppub[p, S]{p}[\cdot][\cdot]\), and
    \item \(\pget[p, S]{s}[\cdot]\) for each \(s\in S\)
  \end{itemize}
  as in \cref{PullFunctions}.
\end{definition}

\begin{figure}%[p]
  \framebox{%
  \begin{minipage}[t]{0.48\textwidth}
    \begin{algorithmic}
      \Function{$\psetup[p, S]{p}$}{$1^\lambda$}
        \State{$(\SignKey{p}, \VerifKey{p})\rgets \Keygen{1^\lambda}$}
        \State{$i_p\rgets \{0, 1\}^{\lambda}$}
        \State{$\forall s\in S$: give $(\VerifKey{p}, i_p)$ to $\Pull[p, 
          S][s]$.}
      \EndFunction{}

      \Statex{}

      \Function{$\ppub[p, S]{p}$}{$R, m$}
        \State{$c\gets \Enc[\{\PubKey{s}\}_{s\in R}, \SignKey{p}]{m}$}
        \State{$\fswrite[i_p][c]$}
      \EndFunction{}
    \end{algorithmic}
  \end{minipage}
  \begin{minipage}[t]{0.48\textwidth}
    \begin{algorithmic}
      \Function{$\psetup[p, S]{s}$}{$1^\lambda$}
        \State{$(\PubKey{s}, \PriKey{s})\rgets \Keygen{1^\lambda}$}
        \State{Give $\PubKey{s}$ to $\Pull[p, S][p]$.}
      \EndFunction{}

      \Statex{}

      \Function{$\pget[p, S]{s}$}{}
        \State{$C = \fsread[i_p]$}
        \State{$M\gets \emptyset$}
        \For{$c\in C$}
          \State{$m_c = \Dec[\PriKey{s}, \VerifKey{p}]{c}$}
          \State{$M\gets M\cup \{m_c\}$}
        \EndFor{}
        \State{\Return{$M$}}
      \EndFunction{}
    \end{algorithmic}
  \end{minipage}
  }
  \caption{%
    Functions implementing the communication model for the Pull Model protocol.
    The publisher's interface is to the left and the subscribers' to the right.
  }\label{PullFunctions}
\end{figure}

When Alice executes \(\psetup{p}\) the \(\KeygenOp\) algorithm generates 
a signature-verification key-pair \((\SignKey{p}, \VerifKey{p})\).\footnote{%
  Since the scheme \(\mathcal{E}\) is an authenticated \ac{BE} scheme, the key 
  generation yields a proper broadcast keys for encryption and decryption in 
  addition to a signature-verification key-pair --- in Alice's case (conversely 
  for her friends) we are only interested in the authentication part, so we 
  simply discard her encryption keys in the definition.
}
She also randomly chooses an identifier.
Then the verification key and identifier are both given to all her friends.
Each of her friends \(s\in S\), when they execute \(\psetup{s}\) the 
\(\KeygenOp\) algorithm generates a public-private key-pair.\footnote{%
  Actually this is ju the notation we use.
  It might just as well be Alice's execution that generates the broadcast keys 
  and distributes them to her friends, while her friends' \(\psetup{s}\) does 
  nothing.
}
They give their respective public key to Alice.

When Alice wants to publish a message \(m\) to the recipient set \(R\subseteq 
  S\), she runs \(\ppub{p}[R][m]\).
This operation simply creates an authenticated broadcast ciphertext 
\(\sigma\gets \Enc[\{\PubKey{s}\}_{s\in R}, \SignKey{p}]{m}\) by combining the 
broadcast key \(\{\PubKey{s}\}_{s\in R}\) and the signature key 
\(\SignKey{p}\).
She appends the ciphertext to her outbox file.
Each friend can then use \(\pget{s}\) to retrieve the ciphertexts from the file 
system and decrypt them.

As we mentioned earlier (and in \cref{PullModel}), \ac{BE} schemes are designed 
for the Pull Model.
We wanted to ensure that no subscriber should be able to learn who else is 
a subscriber.
We know from the analysis by \citet{ANOBE} that Eve has negligible advantage in 
the game in \cref{ANO-IND-CCA} when the \ac{ANOBE} scheme is used in 
\cref{PullModel}.
(We will not restate their proof here, but refer to the full version of their 
paper~\cite{ANOBE}.)

However, \ac{ANOBE} used in the Pull Model cannot immediately yield negligible 
advantage to Jealous Bob.
Bob can read Alice's outbox, conclude that there are entries which he cannot 
decrypt and thus he can become jealous.
(In \cref{PullFunctions} we see that Bob can count the number of \(\bot\) in 
the output of \(\pget{\cdot}\).)
One approach to prevent this is that Alice uses a unique outbox per friend, so 
Bob has his own outbox.
This actually reduces the problem to the Push Model.
Unfortunately this is not entirely trivial, which we will see in our analysis 
in \cref{PushModel}.
We will instead analyse a simpler solution for the Pull Model.

\subsection{Changing Recipient Set in the Pull Model}
\label{ChangingPullRecipientSet}

We will now propose and analyse a solution to the problem of Jealous Bob.
This solution, however, is based on the intricacies of the \ac{ANOBE} 
construction, and thus require us to use \ac{ANOBE} in the Pull Model above.
(For convenience the \ac{ANOBE} scheme is described in \cref{ANOBE}.)

When Bob get jealous he goes to Eve to ask her help, this means that we must 
prevent Eve from learning how Alice changed her policy too.
When Alice wants to change her policy and remove Jealous Bob from the 
subscriber set \(S\), she will update the identifier of her outbox file.
Essentially she will divide \(S\) into two new sets \(S_0, S_1\subset S\) such 
that \(S = S_0\cup S_1\), where Jealous Bob will be alone in one of them.
Then she will generate two new outbox identifiers \(i_0, i_1\) --- one for the 
recipients in \(S_0\) and one those in \(S_1\).
She will then create a broadcast ciphertext which decrypts to \(i_0\) for 
recipients in \(S_0\) and to \(i_1\) for recipients in \(S_1\).
From them on Alice will publish to the two outbox files \(i_0\) and \(i_1\) 
separately.
If she no longer wants to post to Bob, she will simply fade out her 
publications in her new Bob-specific outbox.
We summarize this algorithm in \cref{SplitGroup}.

\begin{figure}
  \framebox{%
    \begin{minipage}{0.96\textwidth}
      \begin{algorithmic}
        \Function{$\SplitGroupOp$}{$S_0, S_1$}
          \Comment{%
            Recipient sets $S_0, S_1$ such that $S = S_0\cup S_1$ and $S_0\cap 
            S_1 = \emptyset$.
          }
          \State{%
            $i_0\rgets \{0, 1\}^\lambda$,
            $(\SignKey{0}, \VerifKey{0})\rgets \Keygen{1^\lambda}$
          }
          \Comment{Generate new outbox and key-pair.}
          \State{%
            $i_1\rgets \{0, 1\}^\lambda$,
            $(\SignKey{1}, \VerifKey{1})\rgets \Keygen{1^\lambda}$
          }
          \Comment{One per new group.}
          \State{%
            $(\SignKey{}, \VerifKey{})\gets{\Keygen{1^\lambda}}$
          }
          \Comment{%
            One-time signature-verification key-pair.
          }
          \State{%
            Choose a random permutation $\pi\colon S\to S$.
          }

          \For{$s \in S$}
            \If{$s\in S_0$}
              \State{%
                $c_{s}\gets{ \Enc[\PubKey{s}]{i_0, \VerifKey{0}, \VerifKey{}}}$
              }
            \Else\Comment{$s\in S_1$}
              \State{%
                $c_{s}\gets{ \Enc[\PubKey{s}]{i_1, \VerifKey{1}, \VerifKey{}}}$
              }
            \EndIf{}
          \EndFor{}

          \State{%
            $C\gets{\mleft( c_{\pi(s)}\mright)_{s\in S}}$
          }
          \Comment{Put all subciphertexts in random order.}
          \State{%
            $\sigma\gets{ \Sign[\SignKey{}]{C} }$
          }
          \State{%
            \Return{$(i_0, \SignKey{0}), (i_1, \SignKey{1}), (\VerifKey{}, C, 
              \sigma)$}
          }
        \EndFunction{}
      \end{algorithmic}
    \end{minipage}
  }
  \caption{%
    An algorithm splitting a subscriber set \(S\) into two new \(S_0, S_1\).
  }\label{SplitGroup}
\end{figure}

The property we want from this algorithm is that Bob cannot determine how the 
subscribers are divided.
Thus he cannot know whether he is removed or not.
We capture this in the following game.

\begin{definition}[Split-group indistinguishability]\label{SplitGroupSecurity}
  Let \(\mathcal{E} = (\SetupOp, \KeygenOp, \EncOp, \DecOp)\) be an \ac{ANOBE} 
  scheme.
  We define the security of the split-group algorithm in \cref{SplitGroup} as 
  follows:
  \begin{description}
    \item[Setup] The challenger runs \(\Setup{1^\lambda}\) of the \ac{ANOBE} 
      scheme.
      This generates the public and private keys for all recipients.
      The public keys are given to the adversary.

    \item[Phase 1] The adversary may corrupt recipients (request secret keys) 
      and request decryptions of ciphertexts for any recipient (without 
      corruption).

    \item[Challenge] The adversary gives the challenger a set of uncorrupted 
      recipients \(R = \{r_0, \ldots, r_n\}\).
      The challenger randomly chooses \(B\rgets (B_0, \ldots, B_n) \in \{0, 
        1\}^n\) and constructs \(R_0, R_1\) as follows:
      \(R_i = \{ r_j\in R\colon B_j = i\}\), where \(i\in \{0, 1\}\).
      The challenger then runs \(C\gets \SplitGroup[R_0, R_1]\) and gives \(C\) 
      to the adversary.

    \item[Phase 2] The adversary may continue to corrupt recipients and request 
      decryptions under the following conditions:
      she cannot corrupt any \(r\in R\) and not request decryptions for 
      subciphertexts of the challenge ciphertext \(C\).

    \item[Guess] The adversary outputs \(\hat{B}\) and wins if \(\hat{B} 
        = B\).
  \end{description}
  We define the adversary's advantage as \[\Adv{\text{Split-IND-CCA}}{\A, 
      \SplitGroupOp}[1^\lambda] = \mleft|\Pr[ \hat{B} 
    = B ] - \frac{1}{2^n}\mright|.\]
\end{definition}

Now we prove the following theorem about the security of the split-group 
algorithm.

\begin{theorem}\label{SplitGroupIsSecure}
  The algorithm \(\SplitGroupOp\) is Split-IND-CCA if the encryption algorithm 
  is ANO-IND-CCA secure.
\end{theorem}

\begin{proof}[sketch]
  The proof is by contradiction and follows that of Thm.~2 and Lem.~4 and 5 of 
  the full version of~\cite{ANOBE}, we refer to those for details.
  Assume that there exists an adversary \(\A\) that can win the Split-IND-CCA 
  game with non-negligible probability.
  We will now describe how to construct an algorithm \(\A*\) that can win the 
  IND-CCA game using \(\A\).

  We construct a sequences of games \(0\leq k\leq n\) with the following idea.
  In game \(0\) all recipients are in \(R_0\) and none in \(R_1\), while in 
  game \(n\) all recipients are in \(R_1\) and none in \(R_0\).
  If \(\A*\) uses the public key \(\PubKey{*}\) and challenge ciphertext 
  \(c_*\) (which is either \(i_0\) or \(i_1\)) from its own challenger as one 
  of the possible recipients, then for some games, say \(k\) and \(k+1\), 
  \(\A*\) will not know whether it is game \(k\) or game \(k+1\) that is 
  currently played (if \(c_* = i_0\) it is game \(k\), and game \(k+1\) if 
  \(c_* = i_1\)).
  If \(\A\) can distinguish between games \(k\) and \(k+1\), then \(\A\) knows 
  whether \(c_* = i_0\) or \(c_* = i_1\) and thus \(\A*\) can distinguish 
  between IND-CCA ciphertexts.
%  Game \(k\) is as follows:
%  \begin{description}
%    \item[Setup] \(\A*\) will generate \(n-1\) key-pairs itself and get one 
%      public key from its challenger, \(\PubKey{*}\).
%      \(\A*\) gives all public keys to \(\A\).
%
%    \item[Phase 1] \(\A*\) will just forward decryption requests from \(\A\) to 
%      its own oracle.
%      If \(\A\) wants the private key for \(\PubKey{*}\), then \(\A*\) quits 
%      with failure.
%
%    \item[Challenge] When \(\A\) gives \(\A*\) its recipient set \(R\), \(\A*\)
%      will construct \(R_0, R_1\) in the following way.
%      If the recipient corresponding to \(\PubKey{*}\) is not in \(R\), then 
%      \(\A*\) quits with failure.
%      Otherwise, \(\A*\) sets \(B = (1, \ldots, 1, 0, \ldots, 0)\) with \(k\) 
%      leading \(1\)'s in game \(k\).
%      For some game \(k^\prime\) \(\A*\) will set the subciphertext \(c_k\) to 
%      its own challenge ciphertext, for which \(\A*\) does not know the 
%      plaintext.
%
%    \item[Phase 2] \(\A*\) continues to forward the requests from \(\A\) to its
%      own oracles.
%      However, no requests for decryptions of any of the subciphertexts of the 
%      challenge ciphertext is allowed.
%
%    \item[Guess] When \(\A\) outputs its result, \(\A*\) can compute its guess 
%      as follows.
%      If recipient \(r_*\in R_0\) \(\A*\) outputs \(0\), otherwise if \(r_*\in 
%        R_1\) \(\A*\) outputs \(1\).
%  \end{description}
  It follows that there cannot be any \(\A\) that can win the game with 
  non-negligible advantage.
  \qed{}
\end{proof}

Now we know that Bob cannot distinguish whether everyone in the recipient sent 
received the same message or not.
We will add the split-group algorithm (\cref{SplitGroup}) as an available 
interface in the Pull Model (\cref{PullModel}) and we summarize this as the 
Extended Pull Model in the following definition.

\begin{definition}[Extended Pull Model]\label{ExtPullModel}
  Let \(\Pull[p, S]\) be an instance of the Pull Model as in \cref{PullModel}.
  Then we define the \emph{Extended Pull Model} to additionally provide the 
  following interfaces:
  \begin{itemize}
    \item \(\psplit[p, S]{p}[\cdot, \cdot]\) to split the subscriber set,
    \item A modified \(\pget[p, S]{s}[]\) to update \(i_p\) after a split,
  \end{itemize}
  as defined in \cref{ExtPullFunctions}.
\end{definition}

\begin{figure}
  \framebox{%
    \begin{minipage}[t]{0.53\textwidth}
      \begin{algorithmic}
        \Function{$\psplit[p, S]{p}$}{$R_0, R_1$}
          \State{$(i_0, \SignKey{0}), (i_1, \SignKey{1}), C\gets 
            \SplitGroup[R_0, R_1]$}
          \State{$\fswrite[i_p][(C, \Sign[\SignKey{p}]{C})]$}
          \State{\Return{$\Pull[p, R_0], \Pull[p, R_1]$}}
        \EndFunction{}
      \end{algorithmic}
    \end{minipage}

    \begin{minipage}[t]{0.43\textwidth}
      \begin{algorithmic}
        \Function{$\pget[p, S]{s}$}{}
          \State{$C = \fsread[i_p]$}
          \State{$M\gets \emptyset$}
          \For{$c\in C$}
            \State{$m_c = \Dec[\PriKey{s}, \VerifKey{p}]{c}$}
            \If{$m_c$ is new identifier}
              \State{$(i_p, \VerifKey{p}) \gets m_c$}
            \Else{}
              \State{$M\gets M\cup \{m_c\}$}
            \EndIf{}
          \EndFor{}
          \State{\Return{$M$}}
        \EndFunction{}
      \end{algorithmic}
    \end{minipage}
  }
  \caption{%
    The additional and modified interfaces of the Extended Pull Model.
    Note that we implicitly let the \(\SplitGroup\) algorithm have access to 
    the public keys of the recipients.
    We also assume there exists a coding that can differentiate an file 
    identifier fro an ordinary message.
  }\label{ExtPullFunctions}
\end{figure}

Note that the execution of \(\psplit[p, S]{p}[R_0, R_1]\) results in two new 
instances of the Pull Model, namely \(\Pull[p, R_0]\) and \(\Pull[p, R_1]\).
After this happens, Alice should no longer use \(\Pull[p, S]\), instead she 
should only use these two new instances.

On the subscribers' side, due to the construction of \(\pget[p, S]{s}\), their 
instance \(\Pull[p, S]\) is automatically turned into \(\Pull[p, R_i]\), for 
whichever \(i\in \{0, 1\}\) Alice put them in.

\subsection{Running Multiple Instances in Parallel}
\label{ParallelPull}

Now it remains for us to convince ourselves that Bob cannot distinguish between
different instances of the Pull Model: i.e.\ Alice posting to two both \(R_0\) 
and \(R_1\) or Alice posting to \(R_0\) and Carol to \(R_1\).
We have the following result.

\begin{theorem}
  Given two instances \(\Pull[p, S]\) and \(\Pull[p^\prime, S^\prime]\), \(p\) 
  and \(p^\prime\) are unlinkable for any adversary \(\A\).
\end{theorem}

\begin{proof}
  We will prove this by an information-theoretic argument.
  We have two cases: first, the two instances are unrelated, \(p\neq 
    p^\prime\); second, the two instances are either related or unrelated while 
  \(p = p^\prime\).
  We focus on the latter.
  In this case there are two possibilities, \(p\) created two instances and 
  they are thus unrelated.
  If they are related \(p\) must have used \(\psplit[p, S^*]{p}\) to form them 
  from some \(S^* = S\cup S^\prime\).
  In doing this, \(p\) randomly chooses \(i_0, i_1\in \{0, 1\}^\lambda\) and 
  \(\SignKey{0}, \SignKey{1}\rgets \Keygen{1^\lambda}\).
  This means that \((i_0, \SignKey{0})\) and \((i_1, \SignKey{1})\) are 
  information-theoretically independent.
  It follows from \cref{SplitGroupIsSecure} and the ANO-IND-CCA property of the 
  encryption scheme that even to \(s\in S\), nothing related to 
  \(\Pull[p^\prime, S^\prime]\) can be found.
  Thus we could equally likely have \(p\neq p^\prime\).
  \qed{}
\end{proof}


