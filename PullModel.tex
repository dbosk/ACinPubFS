\section{Construction and Analysis of the Pull Protocol}
\label{PullAnalysis}

In this model, each publisher has an \enquote{outbox} file in the file system.
This is simply a file object with a randomly chosen identifier.
The publisher adds new publications to the outbox while subscribers pull new 
content from the outbox.
In our analysis we define the protocol for the pull model as follows.

\begin{definition}[Pull Protocol]\label{PullModel}
  Let \(\BE = (\BE[Setup], \BE[Keygen], \BE[Enc], \BE[Dec])\) be an ANO-IND-CCA 
  \ac{BE} scheme,
  \(\S = (\S[Keygen], \S[Sign], \S[Verify])\) be a signature scheme and \(\FS\) 
  be a public file system.
  We denote by \(\Pull[p, S]\) the \emph{pull protocol} implementing a \((p, 
    S)\)-communication model through the operations
  \begin{itemize}
    \item \(\psetup[p, S]{p}[\cdot]\) and \(\psetup[p, S]{s}[\cdot]\) for each
      \(s\in S\),
    \item \(\ppub[p, S]{p}[\cdot][\cdot]\), and
    \item \(\pget[p, S]{s}[\cdot]\) for each \(s\in S\)
  \end{itemize}
  as in \cref{PullFunctions}.
\end{definition}

\begin{figure}%[p]
  \framebox{%
  \begin{minipage}[t]{0.51\textwidth}
    \begin{algorithmic}
      \Function{$\psetup[p, S]{p}$}{$1^\lambda$}
        \State{$i_p\rgets \{0, 1\}^{\lambda}$}
        \State{$(\SignKey{p}, \VerifKey{p})\rgets \S[Keygen][1^\lambda]$}
        \State{$(MPK, MSK)\rgets \BE[Setup][1^\lambda, |S|]$}
        \For{$s\in S$}
          \State{$\PriKey{s}\gets \BE[Keygen][MPK, MSK, s]$}
          \State{Give $(\PriKey{s}, \VerifKey{p}, i_p)$ to $\Pull[p, S][s]$.}
        \EndFor{}
      \EndFunction{}

      \Statex{}

      \Function{$\ppub[p, S]{p}$}{$R, m$}
        \State{$c\gets \BE[Enc][MPK, R, m]$}
        \State{$\sigma\gets \S[Sign][\SignKey{p}, c]$}
        \State{$\fswrite[i_p][c\concat \sigma]$}
      \EndFunction{}
    \end{algorithmic}
  \end{minipage}
  \begin{minipage}[t]{0.45\textwidth}
    \begin{algorithmic}
      \Function{$\psetup[p, S]{s}$}{$1^\lambda$}
        \State{Receive from \(p\).}
      \EndFunction{}

      \Statex{}

      \Function{$\pget[p, S]{s}$}{}
        \State{$C = \fsread[i_p]$}
        \State{$M\gets \emptyset$}
        \For{$(c, \sigma)\in C$}
          \If{$\S[Verify][\VerifKey{p}, c, \sigma]\neq 1$}
            \State{Continue with next.}
          \EndIf{}
          \State{$m_c\gets \BE[Dec][MPK, \PriKey{s}, c]$}
          \If{$m_c\neq \bot$}
            \State{$M\gets M\cup \{m_c\}$}
          \EndIf{}
        \EndFor{}
        \State{\Return{$M$}}
      \EndFunction{}
    \end{algorithmic}
  \end{minipage}
  }
  \caption{%
    Functions implementing the communication model for the pull protocol.
    The publisher's interface is to the left and the subscribers' to the right.
  }\label{PullFunctions}
\end{figure}

When Alice executes \(\psetup{p}\) she will create all needed keys.
She also randomly chooses an identifier.
Then each respective secret key, the verification key and identifier are given 
to all her friends.

When Alice wants to publish a message \(m\) to the recipient set \(R\subseteq 
  S\), she runs \(\ppub{p}[R][m]\).
This operation simply creates a \ac{BE} ciphertext and Alice appends the 
ciphertext to her outbox file.
Each friend can then use \(\pget{s}\) to retrieve the ciphertexts from the file 
system and decrypt them.

Note that we can have symmetric authentication, e.g.\ \iac{MAC} by replacing 
\(\S\) with \iac{MAC} scheme, although we use the notation of asymmetric 
authentication.
This would yield different privacy properties, i.e.\ we remove the 
non-repudiation which \(\S\) brings.
However, this is not at the centre of our discussion, it just illustrates the 
modularity of the scheme.

We wanted no recipient to know who else received the message, the ANO-IND-CCA 
property of the \ac{BE} scheme gives us this.
So we achieve the hidden policy that we wanted.

However, we cannot immediately achieve the hidden policy-update property.
Bob can read Alice's outbox, conclude that there are entries which he cannot 
decrypt and thus he can become jealous.
(In \cref{PullFunctions} we see that Bob can count the number of \(\bot\) in 
the output of \(\pget{\cdot}\).)
One approach to prevent this is that Alice uses a unique outbox per friend, so 
Bob has his own outbox.
This actually reduces the problem to our push construction, which we will cover 
later.
We will instead analyse a simpler solution for the pull protocol now.

\subsection{Changing Recipient Set in the Pull Protocol}
\label{ChangingPullRecipientSet}

We will now propose and analyse a solution to the problem of hidden 
policy-updates, i.e.\ Jealous Bob.
This solution, however, is based on the intricacies of the \ac{ANOBE} 
construction, and thus require us to use \ac{ANOBE}.
In essence, the \ac{ANOBE} construction, described in \cref{ANOBE}, allows us 
to create a ciphertext which decrypts to different messages.\footnote{%
  If this is not possible for \iac{BE} scheme under consideration, then there 
  is always the possibility to simply create new instance.
  However, this might be more costly, since it requires a separate secure 
  channel.
  But we are not bound to the \ac{ANOBE} scheme, it simply provides some extra 
  convenience in this case.
}
We will use this to encrypt a special message that updates the outbox 
identifier \(i_p\).
Half of the recipients will receive \(i_p^\prime\) and the other half will 
receive \(i_p^{\prime\prime}\).
We summarize this algorithm in \cref{SplitGroup}.
Alice could divide her subscribers into an arbitrary number of parts, even 
\(|S|\) for having each subscriber in their own group.
But for simplicity we describe the algorithm for dividing the subscriber set 
into two parts.

\begin{figure}
  \framebox{%
    \begin{minipage}{0.96\textwidth}
      \begin{algorithmic}
        \Function{$\SplitGroupOp$}{$MPK, S_0, S_1$}
          \Comment{%
            Recipient sets $S_0, S_1$ such that $S = S_0\cup S_1$ and $S_0\cap 
            S_1 = \emptyset$.
          }
          \State{%
            $i_0\rgets \{0, 1\}^\lambda$,
            $(\SignKey{0}, \VerifKey{0})\rgets \S*[Keygen][1^\lambda]$
          }
          \Comment{Generate new outbox and key-pair.}
          \State{%
            $i_1\rgets \{0, 1\}^\lambda$,
            $(\SignKey{1}, \VerifKey{1})\rgets \S[Keygen][1^\lambda]$
          }
          \Comment{One per new group.}
          \State{%
            $(\SignKey{}, \VerifKey{})\gets{\S*[Keygen][1^\lambda]}$
          }
          \Comment{%
            One-time signature-verification key-pair.
          }
          \State{%
            Choose a random permutation $\pi\colon S\to S$.
          }

          \For{$s \in S$}
            \If{$s\in S_0$}
              \State{%
                $c_{s}\gets{ \E*[Enc][\PubKey{s}, i_0\concat 
                  \VerifKey{0}\concat \VerifKey{}]}$
              }
            \Else\Comment{$s\in S_1$}
              \State{%
                $c_{s}\gets{ \E*[Enc][\PubKey{s}, i_1\concat 
                  \VerifKey{1}\concat \VerifKey{}]}$
              }
            \EndIf{}
          \EndFor{}

          \State{%
            $C\gets{\mleft( c_{\pi(s)}\mright)_{s\in S}}$
          }
          \Comment{Put all subciphertexts in random order.}
          \State{%
            $\sigma\gets{ \S*[Sign][\SignKey{}, C] }$
          }
          \State{%
            \Return{$(i_0, \SignKey{0}), (i_1, \SignKey{1}), (\VerifKey{}, C, 
              \sigma)$}
          }
        \EndFunction{}
      \end{algorithmic}
    \end{minipage}
  }
  \caption{%
    An algorithm splitting a subscriber set \(S\) into two new \(S_0, S_1\).
  }\label{SplitGroup}
\end{figure}

We note that this solution does not hide the fact that Alice updated her 
policy, we just hide from Bob whether he was excluded or not.
When Bob get jealous he goes to Eve to ask her help, this means that we must 
prevent Eve from learning how Alice changed her policy too --- so the 
construction is secure against this strong adversary.

The property we want from this algorithm is that Bob cannot determine how the 
subscribers are divided.
Thus he cannot know whether he is removed or not.
This follows from the ANO-IND-CCA property of the \ac{ANOBE} scheme.
We refer the reader to the proof of Thm.~1 in~\cite{ANOBE}.

Now we know that Bob cannot distinguish whether everyone in the recipient sent 
received the same message or not.
We will add the split-group algorithm (\cref{SplitGroup}) as extension to the 
pull protocol of \cref{PullModel} and we summarize this as the Extended Pull 
Protocol in the following definition.

\begin{definition}[Extended Pull Protocol]\label{ExtPullModel}
  Let \(\Pull[p, S]\) be an instance of the Pull Model as in \cref{PullModel}.
  Then we define the \emph{Extended Pull Model} to additionally provide the 
  following interfaces:
  \begin{itemize}
    \item \(\psplit[p, S]{p}[\cdot, \cdot]\) to split the subscriber set,
    \item A modified \(\pget[p, S]{s}[]\) to update \(i_p\) after a split,
  \end{itemize}
  as defined in \cref{ExtPullFunctions}.
\end{definition}

\begin{figure}
  \framebox{%
    \begin{minipage}[t]{0.53\textwidth}
      \begin{algorithmic}
        \Function{$\psplit[p, S]{p}$}{$R_0, R_1$}
          \State{$(i_0, \SignKey{0}), (i_1, \SignKey{1}), C\gets 
            \SplitGroup[MPK, R_0, R_1]$}
          \State{$\sigma\gets \S[Sign][\SignKey{p}, C]$}
          \State{$\fswrite[i_p][C\concat \sigma]$}
          \State{\Return{$\Pull[p, R_0], \Pull[p, R_1]$}}
        \EndFunction{}
      \end{algorithmic}
    \end{minipage}

    \begin{minipage}[t]{0.43\textwidth}
      \begin{algorithmic}
        \Function{$\pget[p, S]{s}$}{}
          \State{$C = \fsread[i_p]$}
          \State{$M\gets \emptyset$}
          \For{$(c\concat \sigma)\in C$}
            \If{$\S*[Verify][\VerifKey{p}, c, \sigma]\neq 1$}
              \State{Continue with next.}
            \EndIf{}
            \State{$m_c = \E*[Dec][\PriKey{s}, c]$}
            \If{$m_c$ is new identifier}
              \State{$(i_p, \VerifKey{p}) \gets m_c$}
            \Else{}
              \State{$M\gets M\cup \{m_c\}$}
            \EndIf{}
          \EndFor{}
          \State{\Return{$M$}}
        \EndFunction{}
      \end{algorithmic}
    \end{minipage}
  }
  \caption{%
    The additional and modified interfaces of the Extended Pull Protocol.
    We assume there exists a coding that can differentiate a file identifier 
    from an ordinary message.
  }\label{ExtPullFunctions}
\end{figure}

Note that the execution of \(\psplit[p, S]{p}[R_0, R_1]\) results in two new 
instances of the pull protocol, namely \(\Pull[p, R_0]\) and \(\Pull[p, R_1]\).
After this happens, Alice should no longer use \(\Pull[p, S]\), instead she 
should only use these two new instances.

We could instead incorporate \(\psplit{p}\) into the \(ppub{p}\) interface and 
let it keep track of the different groups (instances) in the state, this would 
make it more similar to the definition of the ideal model 
(\cref{CommunicationModel}).
However, for presentation purposes, many simpler algorithms are better than 
fewer that are more complex.

On the subscribers' side, due to the construction of \(\pget[p, S]{s}\), their 
instance \(\Pull[p, S]\) is automatically turned into \(\Pull[p, R_i]\), for 
whichever \(i\in \{0, 1\}\) Alice put them in.

\subsection{Running Multiple Pull Instances in Parallel}
\label{ParallelPull}

Now it remains for us to convince ourselves that Bob cannot distinguish between
different instances of the pull protocol: i.e.\ Alice posting to two both 
\(R_0\) and \(R_1\) or Alice posting to \(R_0\) and Carol to \(R_1\).

We will use an information-theoretic argument.
We can see that \(\Pull[p_0, S_0]\) is information-theoretically independent 
from both \(\Pull[p_1, S_1^0]\) and \(\Pull[p_1, S_1^1]\).
Although \(\Pull[p_1, S_1^0]\) and \(\Pull[p_1, S_1^1]\) are coming from 
a split, we can see in \cref{SplitGroup} that the two instances only depend 
on the randomly chosen identifiers and verification keys, and the sets of 
users.
The public keys can remains the same, the key-privacy property of \(\E*\) will 
ensure this.
It follows that \(\Pull[p_1, S_1^0]\) is as indistinguishable from \(\Pull[p_1, 
  S_1^1]\) as \(\Pull[p_0, S_0]\) is indistinguishable from \(\Pull[p_1, 
  S_1^i]\), for \(i\in \{0, 1\}\).
Thus Bob's only chance is by distinguishing who received what new outbox in the 
split message, but this is difficult given the ANO-IND-CCA property.

