\section{Construction and Analysis of the Push Model}
\label{PushAnalysis}

The idea of the Push Model is for each subscriber to have an inbox in the file 
system --- as opposed to the Pull Model, where the publisher has an outbox.
This is simply a file object with a randomly chosen identifier.
The publisher then puts all published material in the inbox of each subscriber.

We can see that if we simply put the broadcast ciphertext from the Pull Model 
in all inboxes, then Eve can relate them since they contain identical 
ciphertexts.
We thus have to make some more modifications.
We will base our analysis on the following definition.

\begin{definition}[Push Model]\label{PushModel}
  Let \(\mathcal{E} = (\KeygenOp, \EncOp, \DecOp)\) be an authenticated 
  encryption scheme and \(\FSop\) be a public file system.
  We denote by \(\Push[p, S]\) the \emph{push model protocol} implementing 
  a \((p, S)\)-communication model through the operations
  \begin{itemize}
    \item \(\pnsetup[p, S]{p}[\cdot]\) and \(\pnsetup[p, S]{s}[\cdot]\),
    \item \(\pnpub[p, S]{p}[\cdot][\cdot]\), and
    \item \(\pnget[p, S]{s}[\cdot]\)
  \end{itemize}
  as in \cref{PushFunctions}.
\end{definition}

\begin{figure}%[p]
  \framebox{%
  \begin{minipage}[t]{0.48\textwidth}
    \begin{algorithmic}
      \Function{$\pnsetup[p, S]{p}$}{$1^\lambda$}
        \For{$r\in S$}
          \State{$(\SignKey{r}, \VerifKey{r})\rgets \Keygen{1^\lambda}$}
          \State{Give $\VerifKey{r}$ to $\Push[p, S][r]$.}
        \EndFor{}
      \EndFunction{}

      \Statex{}

      \Function{$\pnpub[p, S]{p}$}{$R, m$}
        \For{$r\in R$}
          \State{$c_r\gets \Enc[\PubKey{r}, \SignKey{r}]{m}$}
          \State{$\fswrite[i_r][c_r]$}
        \EndFor{}
      \EndFunction{}
    \end{algorithmic}
  \end{minipage}
  \begin{minipage}[t]{0.48\textwidth}
    \begin{algorithmic}
      \Function{$\pnsetup[p, S]{s}$}{$1^\lambda$}
        \State{$(\PubKey{s}, \PriKey{s})\rgets \Keygen{1^\lambda}$}
        \State{$i_s\rgets \{0, 1\}^{\lambda}$}
        \State{Give $(\PubKey{s}, i_s)$ to $\Push[p, S][p]$.}
      \EndFunction{}

      \Statex{}

      \Function{$\pnget[p, S]{s}$}{}
        \State{$C\gets \fsread[i_s]$}
        \If{$C = \bot$}
          \State{\Return{$\emptyset$}}
        \EndIf{}
        \State{$M\gets \emptyset$}
        \For{$c\in C$}
          \State{$m_c\gets \Dec[\PriKey{s}, \VerifKey{s}]{c}$}
          \If{$m_c\neq \bot$}
            \State{$M\gets M\cup \{m_c\}$}
          \EndIf{}
        \EndFor{}
        \State{\Return{$M$}}
      \EndFunction{}
    \end{algorithmic}
  \end{minipage}
  }
  \caption{%
    Functions implementing the communication model for the Push Model protocol.
    The publisher's interface is to the left and the subscribers' to the right.
  }\label{PushFunctions}
\end{figure}

When Alice executes \(\pnsetup{p}\) the \(\KeygenOp\) algorithm generates 
a signature-verification key-pair \((\SignKey{s}, \VerifKey{s})\) for every 
subscriber \(s\in S\) and gives the verification key to all her 
friends.\footnote{%
  Since the scheme \(\mathcal{E}\) is an authenticated encryption scheme, the 
  key generation yields a public-private key-pair and a signature-verification 
  key-pair --- in Alice's case (conversely for her friends) we are only 
  interested in the authentication part, so we simply discard her encryption 
  keys in the definition.
}
Each of her friends \(s\in S\), when they execute \(\pnsetup{s}\) the 
\(\KeygenOp\) algorithm generates a public-private key-pair.
Additionally they randomly choose a string as an identifier for their inbox.
They give the public key and the identifier to Alice.

When Alice wants to send a message \(m\) to a subset \(R\subseteq S\) of her 
friends, she uses \(\pnpub{p}\) to create the authenticated ciphertexts 
\(c_s\gets \Enc[\PubKey{s}, \SignKey{s}]{m}\) for each friend \(s\in R\).
The encryption scheme simply combines the recipients public key \(\PubKey{s}\) 
and Alice's friend-specific signature key \(\SignKey{s}\).
Then she uses \(\FS\) to append the ciphertext \(c_s\) to the file with 
identifier \(i_s\).

The operation \(\pget{s}\) works similarly as in the Pull Model, however, it 
uses a different file.
It starts by reading the inbox from the file system.
Then it iterates through the list of entries, decrypting each entry.
Each entry, if successfully decrypted, is a new message.
It appends the message to the list of messages which it returns upon 
finishing.

Note that, similar to the Pull Model, the authenticated encryption scheme 
\(\mathcal{E}\) can be a symmetric-key scheme although we use the notation of 
public-key cryptography in our abstraction.
We can simply let \(\PubKey{} = \PriKey{}\) and \(\SignKey{} = \VerifKey{}\) to 
achieve this.
This makes the Push Model modular as well.

% XXX Write high-level analysis of push model
Let us now look at what Eve can do in this model.
We start by looking at the ANO-IND-CCA game (\cref{ANO-IND-CCA}).
In the Push Model setting Eve can actually win this game with non-negligible 
advantage:
She starts by publishing a message (through the publication oracle) to each 
user.
Then she selects two sets of users, gives those to the challenger.
When the challenger publishes the message, Eve just observes which inboxes are 
written to and then she knows with absolute certainty which set was used.

To deal with this problem, we will introduce a mix-net.

\begin{definition}[Mix-net]\label{MixNet}
  Let \(m_1, \ldots, m_k\) be messages from senders \(1, \ldots, k\).
  A mix-net is a functionality \(\M\) such that on inputs \(m_1, \ldots, m_k\) 
  the outputs \(\M[1](m_1), \ldots, \M[k](m_k)\) are unlinkable to its senders.
  More specifically
  \begin{equation*}
    \Pr[i\mid \M[i](m_i)] = \Pr[i] = \frac{1}{k}.
  \end{equation*}
\end{definition}

The mix-net will not do much help in a single instance.
If all inputs to the mix-net are related, then so will the outputs and Eve will 
still not have any problem winning the ANO-IND-CCA game.
We must have one instance per input, and thus, with this construction we must 
run several instances in parallel to achieve any security.

\subsection{Running Multiple Push Instances in Parallel}
\label{ParallelPush}

There are two issues we must consider when running the Push Model in parallel.
First, as in the Pull Model, we must convince ourselves that Eve cannot 
distinguish between parallel instances of the Push Model.
Thereafter we can continue our discussion of using the mix-net to mix different
instances together.

We will now clarify what we mean by two instances being indistinguishable.
Given two ciphertexts, the adversary should not be able to tell if they are 
from the same or two different instances.
More formally:

\begin{definition}[Push-IND-CCA]
  We have the following game:
  \begin{description}
    \item[Setup] The challenger creates two instances \(\Push[p_0, S_0]\) and 
      \(Push[p_1, S_1]\).
      The adversary is given the public keys and verification keys for all 
      \(s\in S_0\cup S_1\).

    \item[Phase 1] The adversary may request encryptions and decryptions for 
      arbitrary plaintexts and ciphertexts for any user through oracle access.

    \item[Challenge] The adversary gives two messages \(m_0\) and \(m_1\) to 
      the challenger.
      The challenger then chooses \(b\in \{0, 1\}\).
      If \(b = 1\) the challenger will randomly choose two recipients \(r, 
        r^\prime\) from the same set \(S_i\), otherwise \(r\) from \(S_0\) and 
      \(r^\prime\) from \(S_1\).
      The challenger encrypts \(m_b\) for \(r\) and \(r^\prime\) (as in 
      \(\pnpub{p}\)) and gives the ciphertexts \(c, c^\prime\) to the 
      adversary.

    \item[Phase 2] The adversary is allowed to continue querying the oracles, 
      except for the challenge ciphertexts \(c\) and \(c^\prime\).

    \item[Guess] The adversary outputs \(\hat{b}\) and wins if \(\hat{b} = b\).
  \end{description}
  We define the advantage as \(\Adv{\text{Push-IND-CCA}}{\A}[1^\lambda] 
    = |\Pr[\hat{b} = b] - \frac{1}{2}|\).
\end{definition}

To convince ourselves that two instances indeed are indistinguishable we give 
the following theorem and proof.

\begin{theorem}
  The Push Model (\cref{PushModel}) yields Push-IND-CCA security if the 
  underlying encryption scheme is IND-CCA secure.
  More specifically,
  \begin{equation*}
    \Adv{\text{Push-IND-CCA}}{\A}[1^\lambda]\leq \Adv{\text{IND-CCA}}{\A, 
      \mathcal{E}}[1^\lambda],
  \end{equation*}
  where \(\mathcal{E}\) is the encryption scheme used.
\end{theorem}

\begin{proof}[sketch]
  We will give an information-theoretic argument for this.
  Each encryption in \cref{PushFunctions} depends on the following:
  the public key of the recipient, the signature key for the authentication, 
  and the message.
  Both the public key and the signature key is unique per recipient.
  It follows that two encryptions of the same message is equally likely done 
  within the same instance as it is in two different instances.
  Thus, if the adversary has non-negligible advantage, it must be able to 
  distinguish \(m_0\) from \(m_1\).
  \qed{}
\end{proof}

Now we know that it makes sense to use the mix-net, because the messages 
themselves will not reveal which instance they belong to.
We can thus resume our discussion of using the mix-net to achieve something 
similar to the ANO-IND-CCA property for the Pull Model.
We already concluded that we cannot achieve security in the Push Model without 
running parallel instances.
We will thus do that, and as a consequence we will also run the security games 
in parallel.

\begin{definition}[Parallel games]
  Let \(\M\) be a mix-net with \(k\) input slots.
  We can run \(k\) games in parallel if each game uses the mix-net and is 
  assigned one input slot.
\end{definition}

We note that by definition the mix-net will not output anything until the last
input slot has received input, thus synchronizing the games and Eve cannot 
force one game to proceed without the other \(k-1\) games.

The game we are interested in running in parallel is ANO-IND-CPA from 
\cref{ANO-IND-CPA}.
We will now assume that the calls to \(\FS\) in \cref{PushFunctions} are 
replaced with \(\M_w[\cdot](\fswrite[\cdot][\cdot])\) and 
\(\M_r[\cdot](\fsread[\cdot])\), where \(\M_w\) and \(\M_r\) are two mix-nets 
for writing and reading operations, respectively.
The input slot is unique to each game.
Then we get the following result.

\begin{theorem}
  \(k\) instances of the Push Model run in parallel using a mix-net, cannot 
  yield ANO-IND-CPA security.
\end{theorem}

\begin{proof}
  In phase 1 the adversary selects one target user in each instance.
  She publishes any message in each instance (through the mix-net) to the 
  selected users and notes the result.
  Now the adversary replaces one user and repeats the process.
  The difference in result (or number of messages in the inboxes) will reveal 
  the inboxes of the new and replaced user.
  This can be repeated until all inboxes known.
  Then the adversary can easily win the game.
  \qed{}
\end{proof}

\begin{corollary}
  The Push Model cannot yield ANO-IND-CCA security.
\end{corollary}

The corollary follows from the same argument, only here the decryption oracle 
can be used to make Eve more efficient:
Eve can publish a unique message to one user in each instance, then she simply 
requests a decryption of the resulting ciphertexts to see which inboxes belongs 
to which instance.

