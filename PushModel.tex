\section{Construction and Analysis of the Push Protocol}
\label{PushAnalysis}

The idea of the push model is for each subscriber to have an inbox in the file 
system --- as opposed to the pull model, where the publisher has an outbox.
This is simply a file object with a randomly chosen identifier.
The publisher then puts all published material in the inbox of each subscriber.

We can see that if we simply put the broadcast ciphertext from the pull 
protocol in all inboxes, then Eve can relate them since they contain identical 
ciphertexts.
We thus have to make some more modifications.
We will use the protocol in the next definition in our analysis.

\begin{definition}[Push Protocol]\label{PushModel}
  Let \(\E = (\E[Keygen], \E[Enc], \E[Dec])\) be an AI-CCA encryption scheme,
  \(\S = (\S[Keygen], \S[Sign], \S[Verify])\) be a strongly unforgeable 
  signature scheme,
  and \(\FSop\) be a public file system.
  We denote by \(\Push[p, S]\) the \emph{push model protocol} implementing 
  a \((p, S)\)-communication model through the operations in 
  \cref{PushFunctions}.
%  \begin{itemize}
%    \item \(\pnsetup[p, S]{p}[\cdot]\) and \(\pnsetup[p, S]{s}[\cdot]\),
%    \item \(\pnpub[p, S]{p}[\cdot][\cdot]\), and
%    \item \(\pnget[p, S]{s}[\cdot]\)
%  \end{itemize}
\end{definition}

\begin{figure}%[p]
  \framebox{%
  \begin{minipage}[t]{0.48\textwidth}
    \begin{algorithmic}
      \Function{$\pnsetup[p, S]{p}$}{$1^\lambda$}
        \For{$r\in S$}
          \State{$(\SignKey{r}, \VerifKey{r})\rgets \S[Keygen][1^\lambda]$}
          \State{Give $\VerifKey{r}$ to $r$.}
        \EndFor{}
      \EndFunction{}

      \Statex{}

      \Function{$\pnpub[p, S]{p}$}{$R, m$}
        \For{$r\in R$}
          \State{$c_r\gets \E[Enc][\PubKey{r}, m]$}
          \State{$\sigma_r\gets \S[Sign][\SignKey{p}, c_r]$}
          \State{$\fswrite[i_r][c_r\concat \sigma_r]$}
        \EndFor{}
      \EndFunction{}
    \end{algorithmic}
  \end{minipage}
  \begin{minipage}[t]{0.48\textwidth}
    \begin{algorithmic}
      \Function{$\pnsetup[p, S]{r}$}{$1^\lambda$}
      \Comment{$r\in S$}
        \State{$(\PubKey{r}, \PriKey{r})\rgets \E[Keygen][1^\lambda]$}
        \State{$i_r\rgets \{0, 1\}^{\lambda}$}
        \State{Give $(\PubKey{r}, i_r)$ to $p$.}
      \EndFunction{}

      \Statex{}

      \Function{$\pnget[p, S]{r}$}{}
      \Comment{$r\in S$}
        \State{$C\gets \fsread[i_r]$}
        \If{$C = \bot$}
          \State{\Return{$\emptyset$}}
        \EndIf{}
        \State{$M\gets \emptyset$}
        \For{$(c, \sigma)\in C$}
          \If{$\S[Verify][\VerifKey{r}, c, \sigma]\neq 1$}
            \State{Continue with next.}
          \EndIf{}
          \State{$m_c\gets \E[Dec][\PriKey{r}, c]$}
          \If{$m_c\neq \bot$}
            \State{$M\gets M\cup \{m_c\}$}
          \EndIf{}
        \EndFor{}
        \State{\Return{$M$}}
      \EndFunction{}
    \end{algorithmic}
  \end{minipage}
  }
  \caption{%
    Functions implementing the communication model for the push model protocol.
    The publisher's interface is to the left and the subscribers' to the right.
  }\label{PushFunctions}
\end{figure}

When Alice executes \(\pnsetup{p}\) it generates a signature-verification 
key-pair \((\SignKey{r}, \VerifKey{r})\) for every subscriber \(r\in S\) and 
gives the respective verification keys to each subscriber.
Each subscriber \(r\in S\), when they execute \(\pnsetup{s}\) it generates 
a public-private key-pair.
Additionally they randomly choose a string as an identifier for their inbox.
They give the public key and the identifier to Alice.

When Alice wants to send a message \(m\) to a subset \(R\subseteq S\) of her 
subscribers, she uses \(\pnpub{p}\) to create a ciphertext \(c_r\) with 
signature \(\sigma_r\) for each recipient \(r\in R\).
Then she uses \(\FS\) to append \(c_r\concat \sigma_r\) to the file with 
identifier \(i_r\).
The operation \(\pget{r}\) works similarly as in the pull protocol, however, it 
uses a different file.
It starts by reading the inbox from the file system.
Then it iterates through the list of entries, decrypting each entry.
Each entry, if successfully decrypted, is a new message.
It appends the message to the list of messages which it returns upon 
finishing.

Note that, similarly as for the pull protocol, the authentication scheme \(\S\) 
can be symmetric.
Here, the encryption scheme \(\E\) can also be a symmetric-key scheme, provided 
it is key-private.
We can simply let \(\PubKey{} = \PriKey{}\) and \(\SignKey{} = \VerifKey{}\) to 
achieve this.
This would yield private group communication with few key exchanges.
However, we use the notation of public-key cryptography in our abstraction.
This makes the push protocol modular as well.
%Furthermore, the encryption and authentication can be replaced by authenticated
%encryption if desired, however, to keep the similarity to the pull protocol we 
%keep encryption and authentication separate.

Before we continue our analysis of what Eve can do against the push protocol, 
let us first look back at the pull protocol.
Imagine that Eve gets access to a publication oracle for an instance of the 
pull protocol, and remember that she controls the entire file system.
It is not hard to convince ourselves that Eve cannot learn anything more from 
the file system than she can from the ciphertext alone (when playing the 
ANO-IND-CCA game): each publication just appends the new ciphertext to the same 
file in the
file system --- no matter how she changes the recipient set.

Let us now look at what Eve can do in the push protocol.
Here she can actually learn information by observing the file system.
Whenever a message is published she learns which inboxes are used to get 
messages from the publisher, thus she can relate several inboxes.
Hence, the security analysis of this protocol is only interesting when the 
protocol is run in parallel.

\subsection{Running Multiple Push Instances in Parallel}
\label{ParallelPush}

There is one issue we must consider before continuing our discussion of the 
security of the push protocol.
For the security discussion to make sense we must first, as for the pull 
protocol, convince ourselves that Eve cannot distinguish between parallel 
instances of the push protocol.
Thereafter we can continue our discussion of the security of the push protocol 
run in parallel.

We will give an information-theoretic argument for the indistinguishability.
Each encryption in \cref{PushFunctions} depends on the following:
the public key of the recipient, the signature key for the authentication, 
and the message.
Both the public key and the signature key is unique per recipient and instance.
It follows that two encryptions of the same message are equally likely done 
within the same instance as in two different instances.
This follows from the security of the encryption scheme \(\E\).
Thus, Eve's only chance is to decide from the ciphertexts that they contain the 
same message, but since \(\E\) provides IND-CCA security this is possible only 
with negligible advantage.

Now we know that it makes sense to talk of the security when running parallel 
instances.
Assume that we run \(n\) instances of the push protocol in parallel.
It is not too difficult for Eve to distinguish which inboxes are related.
Eventually one instance will publish when no other instance is publishing.
Then that instance can be distinguished from others when publishing at the same
time, hence Eve can learn more and more over time --- even without using 
anything like the publication oracle mentioned earlier.
To deal with this problem, we will introduce a mix-net.

\begin{definition}[Mix-net]\label{MixNet}
  Let \(m_1, \ldots, m_k\) be messages from senders \(1, \ldots, k\).
  A mix-net is a functionality \(\M\) such that on inputs \(m_1, \ldots, m_k\) 
  the outputs \(\M[1](m_1), \ldots, \M[k](m_k)\) are unlinkable to its senders.
  More specifically \(\Pr[i\mid \M[i](m_i)] = \Pr[i] = \frac{1}{k}\).
\end{definition}

We note that by definition the mix-net will not output anything until the last
input slot has received input.

We will now assume that the calls to \(\FS\) in the push protocol 
(\cref{PushFunctions}) are replaced with 
\(\M_w[\cdot](\fswrite[\cdot][\cdot])\) and \(\M_r[\cdot](\fsread[\cdot])\), 
where \(\M_w\) and \(\M_r\) are two mix-nets for writing and reading 
operations, respectively.
The input slots are unique to each instance of the push protocol, instance 
\(i\) has input slot \(\M[i](\cdot)\).

Remember that the inbox identifier is randomly chosen, this means that the 
probability for an inbox being used by two instances is low (\(\approx 
  1/2^{\lambda/2}\)).
Without the mix-net Eve can also distinguish which inboxes are related by 
looking at the distribution of messages in the files in the file system.
It is unlikely in practice that the behaviour of all publishers will be 
uniform, so Eve will eventually learn the related inboxes.
However, due to the assumed mix-net, it will enforce a uniform number of 
messages, \(n\) messages in yield \(n\) messages out.
There are mix-nets that add dummy messages to improve throughput --- which will
be needed for efficiency reasons --- but we refer to that literature for 
a discussion on its security.

Another technique to distort the distribution is that each subscriber can reuse 
the inbox across instances for different publishers --- this is actually 
desirable for the push protocol also for efficiency reasons, the subscriber has 
only \emph{one inbox for all} publishers.
However, this allows the publisher to determine that a subscriber subscribes to 
other publishers, i.e.\ a scenario related to Jealous Bob, but this time 
a Jealous Alice.
This in turn can be solved by adding noise to each inbox, thus making real 
messages indistinguishable from noise.
A more efficient solution would be for Bob to share an inbox with an 
\emph{unknown} other subscriber.
Bob can pick an inbox which already contains messages and use the same 
identifier.
This would make fetching new messages less efficient, but will reduce the 
amount of noise needed.

Finally we note that if Eve did not control the entire file system, she would 
not be able to relate inboxes that are not in the part she controls.
She would be able to read them, but the accuracy of her timing attacks would be
reduced to how often she could read the files to detect change.
She could still do the message distribution attacks though.
This change of the adversary takes us from a potentially centralized setting to 
a distributed setting where there is no network wide adversary, but an 
adversary controlling only a part of the network of the file system nodes.

