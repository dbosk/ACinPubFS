\author{%
  Daniel Bosk \and
  Oleksandr Bodriagov \and
  Gunnar Kreitz \and
  Sonja Buchegger
}
\institute{%
    School of Computer Science and Communication\\
    KTH Royal Institute of Technology\\
    Email: \email{\{dbosk,obo,gkreitz,buc\}@kth.se}%
}
\title{%
  Access Control in
  Decentralized Online Social Networks using
  Anonymous Broadcast Encryption
}
%\date{17th February 2015}

\maketitle
\begin{abstract}
  \dots
\end{abstract}

\acresetall
\section{Introduction}

Centralized \ac{OSN} collect and store private data.
They are thus a risk for privacy violations, e.g.~data mining for 
advertisements or transferring data to third parties.
We have learned that the spy agencies, like the \ac{NSA}, are very interested 
in the data found in such centralized storage \cite{prism}.

As a response to this \acp{DOSN} have been suggested.
Among the benefits of these are that users can keep control of their data, 
there is no central provider where third parties easily can access it, and it 
is more difficult to censor.

As such, \ac{AC} is an important part in these systems.
There are several \acp{DOSN} available, both as research projects and 
implementations.
Currently their \ac{AC} mechanisms focus on providing confidentiality for the 
data.
However, there is a lack of privacy for the \ac{AC} policies, meaning that 
anyone can see who may access what data.
\citet{predicateac} adapted predicate encryption for \ac{AC} and measured its 
performance.
In the present work, we continue along this line by exploring the benefits of 
using \ac{ANOBE}.

\subsection{\acs{DOSN} model}

Distributed untrusted storage.
The storage nodes run on top of Tor \cite{tor} as hidden services.

All encrypted objects are world-readable.

The profile-owner generates and distributes keys to all friends.

A user can decrypt a ciphertext if her key satisfies the access policy.

Each user has a private key for signing.

\subsection{Our contributions}

\dots

\subsection{Paper outline}

\dots


\section{Related work}

\dots


\section{Anonymous broadcast encryption}

The main idea of \ac{ANOBE} is to distribute keys to to a subset of users, such 
that they can decrypt the broadcast message but no one else can.
This is a suitable mechanism for \ac{DOSN} as user updates are basically 
broadcasts to all or a subset of friends in the network.

\ac{ANOBE} was specifically designed to hide the recipient set.
This is an important property as all ciphertexts are world-readable.
If we did not have this property someone who is not an intended recipient can 
learn who is a valid recipient, perhaps making that subset of users targets to 
learn the broadcast data.

The \ac{ANOBE} scheme of \cite{anobe} \dots


\section{Use in a \acs{DOSN}}

There are several uses of \ac{ANOBE} in \acp{DOSN}.
First, there is the pull model: all friends pull updates from a user's profile.
Second, there is the push model: the user pushes changes to all friends' 
inboxes whenever something is updated in the profile.

The \ac{PE} variant constructed in \cite{predicateac} is suitable for the pull 
model, but not for the push model.
However, an \ac{ANOBE} scheme can probably more easily be used in the push 
model.
(This relates at bit to FAFA.)


\section{Performance evaluation}

\dots


\section{Conclusion}

\dots


\section*{Acknowledgements}

This research was funded by the Swedish Foundation for Strategic Research grant 
SSF FFL09-0086 and the Swedish Research Council grand VR 2009-3793.


\printbibliography
