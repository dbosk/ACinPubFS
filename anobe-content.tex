\title{%
  Privacy-Preserving Access Control in
  Decentralized Online Social Networks using
  Anonymous Broadcast Encryption
}
\author{%
  Daniel Bosk \and
  Sonja Buchegger
}
\institute{%
  School of Computer Science and Communication\\
  KTH Royal Institute of Technology, Stockholm, Sweden\\
  Email: \email{\{dbosk,buc\}@kth.se}%
}
%\date{12th May 2015}

\maketitle

Online social networks collect and store large amounts of private data.
Trusting too much data with third parties is a privacy risk.
For this reason \acp{DOSN} was proposed.
Among the benefits of \acp{DOSN} are that users can keep control of their data, 
there is no central provider through which third parties (by force) can access 
it, and it is more difficult to censor.
However, these require more research to ensure security and privacy, as the 
decentralized structure opens up for other risks---even when data is encrypted.

The current research on access control in \acp{DOSN} has focused on 
efficiently achieving confidentiality for data.
Although the data receives privacy, the access-control structures and patterns 
do not.
We focus on achieving efficient privacy-preserving access control mechanisms 
for the \ac{DOSN} setting, i.e.~hidden policies, hidden credentials and 
hidden decisions.
%This means that no attacker should learn anything about who may access what 
%(the policy), who wants to access an object (the credential) or whether 
%a subject is granted access to an object or not (the decision).

We use the \ac{DOSN} setting to adapt \acl{ANOBE} for better performance.
There are some properties of the underlying distributed storage structure which 
can be utilized for better efficiency and privacy properties.
However, there is usually a trade-off between privacy and efficiency.
In this work we evaluate the privacy properties and the complexity of different 
trade-offs.
One trade-off we explore is to use the push model for a replacement of the 
storage expensive anonymous tag-hint system suggested for \acl{ANOBE}.
This way a subscriber knows which ciphertext is meant for her, since all 
ciphertexts pushed to her inbox are for her.
This has the further advantage of hiding the size of the recipient set.

Another trade-off investigated is to use a semantically secure symmetric cipher 
instead of an IND-CCA2 \acl{PKE}.
The original \acl{ANOBE} scheme was constructed for an IND-CCA2 \acl{PKE} 
scheme, but to scale better for \acp{DOSN} we look into using a less 
computationally complex scheme, like a semantically secure symmetric cipher.

There are also other problems to consider.
For example, \acl{BE} was designed to broadcast a message to a dynamically 
changing group.
As such, changing access policies for already existing objects requires only 
one re-encryption of the object and a re-broadcast to the new set of authorized 
subjects.
This requires the subjects to keep track of those keys.
On the other hand, binding many objects to the same key results in the need for 
re-encryption of many objects in the case of a change of policy.


\printbibliography
