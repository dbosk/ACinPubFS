\section{Algorithmic Complexity}\label{AlgComplexity}
% XXX Actually do some measurements
% XXX Write an introduction to the section on complexities

We will now summarize the algorithmic complexity for our constructions.
The performance is interesting to evaluate from two perspectives: the 
publisher's (Alice in all examples) and the subscriber's (Bob in all examples).
From the publisher's perspective, it is interesting to investigate the needed 
space for key storage, communication complexity for publication and time 
complexity for encryption of new material.
From the subscriber's perspective, the complexity of key-storage size and the 
time-complexity of aggregating the newest published messages are the most 
interesting aspects.
An overview of the results is presented in \cref{Complexities}.

\begin{table}
  \centering
  \caption{%
    The storage, communication and time complexities in the two models.
    \(S\) is the set of all subscribers, \(R\) is the set of recipients of 
    a message.
    All values are \(O(\cdot)\).
  }\label{Complexities}
  \begin{tabular}{lrr}
    \toprule

    Publisher
    & Pull & Push \\
    
    \midrule

    Key-storage size
    & \(|S|\) & \(|S|\) \\

    \pause{}%
    Ciphertext size
    & \(|R|\) & \(|R|\) \\

    Encryption
    & \(1\) & \(|R|\) \\

    \pause{}%
    Communication
    & \(1\) & \(|R|\) \\

    \bottomrule

  \end{tabular}
  \begin{tabular}{lrr}
    \toprule

    Subscriber
    & Pull & Push \\
    
    \midrule

    Key-storage size
    & \(1\) & \(1\) \\

    \pause{}%
    Ciphertext size
    & \(|R|\) & \(1\) \\

    Decryption
    & \(1\) & \(1\) \\

    \pause{}%
    Communication
    & \(1\) & \(1\) \\

    \bottomrule

  \end{tabular}
\end{table}

The space complexity for the key management is the same for both the pull and 
push protocols.
If we have \(|S|\) subscribers, then we need to exchange and store \(O(|S|)\) 
keys:
we need one public key per friend.

The space complexity for the ciphertexts are \(O(|R|)\) for both models.
However, the pull protocol is slightly more space efficient since we need less 
signatures.
In the push protocol we require one signature per ciphertext, i.e.\ \(O(|R|)\), 
whereas for the pull protocol we only need one per message.

The time complexity for encryption depends on the underlying schemes.
But we can see that in the push protocol we get a factor \(|R|\) to that of the 
encryption scheme, whereas we have a constant factor \(1\) in the pull 
protocol.
The time complexity for decryption on the other hand has a constant factor for 
both models.

Finally, the communication complexity for the different models differs 
slightly.
If we look at one single instance, then we get a constant number of connections
for the subscribers.
However, most subscribers will have to pull from several publishers, this is 
the argued benefit of push model of communication --- there is only one inbox 
to read.

%\subsection{Extensions}\label{sec:Extensions}
%% XXX Write about possible extensions to improve the scheme
%% XXX Add that we can replace PKE with SKE in the Push Model
%For performance reasons, we also look into a trade-off between using a robust, 
%key-private IND-CCA2 \ac{PKE} scheme and a semantically secure symmetric 
%encryption scheme in the \ac{ANOBE} construction.
%The reason for this is that the symmetric operations are faster than the 
%asymmetric ones.
%This is an important factor for \acp{DOSN} during e.g.~news-feed aggregation 
%when a user comes online.
%In these situations we have to handle large amounts of data, which can lead to 
%performance problems.
%
%% XXX Write about MAC for group deniability
%If we replace the signatures by \acp{MAC}, then we can achieve group 
%deniability: anyone in the group could have posted the message.


